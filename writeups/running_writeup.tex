    \documentclass[11pt,
        usenames, % allows access to some tikz colors
        dvipsnames % more colors: https://en.wikibooks.org/wiki/LaTeX/Colors
    ]{article}
    \usepackage{
        amsmath,
        amssymb,
        fouriernc, % fourier font w/ new century book
        fancyhdr, % page styling
        lastpage, % footer fanciness
        hyperref, % various links
        setspace, % line spacing
        amsthm, % newtheorem and proof environment
        mathtools, % \Aboxed for boxing inside aligns, among others
        float, % Allow [H] figure env alignment
        enumerate, % Allow custom enumerate numbering
        graphicx, % allow includegraphics with more filetypes
        wasysym, % \smiley!
        upgreek, % \upmu for \mum macro
        listings, % writing TrueType fonts and including code prettily
        tikz, % drawing things
        booktabs, % \bottomrule instead of hline apparently
        cancel % can cancel things out!
    }
    \usepackage[margin=1in]{geometry} % page geometry
    \usepackage[
        labelfont=bf, % caption names are labeled in bold
        font=scriptsize % smaller font for captions
    ]{caption}
    \usepackage[font=scriptsize]{subcaption} % subfigures

    \newcommand*{\scinot}[2]{#1\times10^{#2}}
    \newcommand*{\dotp}[2]{\left<#1\,\middle|\,#2\right>}
    \newcommand*{\rd}[2]{\frac{\mathrm{d}#1}{\mathrm{d}#2}}
    \newcommand*{\pd}[2]{\frac{\partial#1}{\partial#2}}
    \newcommand*{\rtd}[2]{\frac{\mathrm{d}^2#1}{\mathrm{d}#2^2}}
    \newcommand*{\ptd}[2]{\frac{\partial^2 #1}{\partial#2^2}}
    \newcommand*{\md}[2]{\frac{\mathrm{D}#1}{\mathrm{D}#2}}
    \newcommand*{\pvec}[1]{\vec{#1}^{\,\prime}}
    \newcommand*{\svec}[1]{\vec{#1}\;\!}
    \newcommand*{\bm}[1]{\boldsymbol{\mathbf{#1}}}
    \newcommand*{\ang}[0]{\;\text{\AA}}
    \newcommand*{\mum}[0]{\;\upmu \mathrm{m}}
    \newcommand*{\at}[1]{\left.#1\right|}

    \newtheorem{theorem}{Theorem}[section]

    \let\Re\undefined
    \let\Im\undefined
    \DeclareMathOperator{\Res}{Res}
    \DeclareMathOperator{\Re}{Re}
    \DeclareMathOperator{\Im}{Im}
    \DeclareMathOperator{\Log}{Log}
    \DeclareMathOperator{\Arg}{Arg}
    \DeclareMathOperator{\Tr}{Tr}
    \DeclareMathOperator{\E}{E}
    \DeclareMathOperator{\Var}{Var}
    \DeclareMathOperator*{\argmin}{argmin}
    \DeclareMathOperator*{\argmax}{argmax}
    \DeclareMathOperator{\sgn}{sgn}
    \DeclareMathOperator{\diag}{diag\;}

    \DeclarePairedDelimiter\bra{\langle}{\rvert}
    \DeclarePairedDelimiter\ket{\lvert}{\rangle}
    \DeclarePairedDelimiter\abs{\lvert}{\rvert}
    \DeclarePairedDelimiter\ev{\langle}{\rangle}
    \DeclarePairedDelimiter\p{\lparen}{\rparen}
    \DeclarePairedDelimiter\s{\lbrack}{\rbrack}
    \DeclarePairedDelimiter\z{\lbrace}{\rbrace}

    % \everymath{\displaystyle} % biggify limits of inline sums and integrals
    \tikzstyle{circ} % usage: \node[circ, placement] (label) {text};
        = [draw, circle, fill=white, node distance=3cm, minimum height=2em]
    \definecolor{commentgreen}{rgb}{0,0.6,0}
    \lstset{
        basicstyle=\ttfamily\footnotesize,
        frame=single,
        numbers=left,
        showstringspaces=false,
        keywordstyle=\color{blue},
        stringstyle=\color{purple},
        commentstyle=\color{commentgreen},
        morecomment=[l][\color{magenta}]{\#}
    }

\begin{document}

\def\Snospace~{\S{}} % hack to remove the space left after autorefs
\renewcommand*{\sectionautorefname}{\Snospace}
\renewcommand*{\appendixautorefname}{\Snospace}
\renewcommand*{\figureautorefname}{Fig.}
\renewcommand*{\equationautorefname}{Eq.}
\renewcommand*{\tableautorefname}{Tab.}

\onehalfspacing

\pagestyle{fancy}
\rfoot{Yubo Su}
\rhead{}
\cfoot{\thepage/\pageref{LastPage}}

\section{Equations}

We begin with the fully compressible fluid equations and general equation of
state
\begin{subequations}\label{se:comp}
    \begin{align}
        \rd{\rho}{t} + \rho \p*{\vec{\nabla} \cdot \vec{u}} &= 0,
            \label{eq:density}\\
        \rd{S}{t} &= 0,\label{eq:entropy}\\
        \rd{\vec{u}}{t} + \frac{\vec{\nabla}P}{\rho} + \vec{g} &= 0,
            \label{eq:velocity}\\
        P &= P(\rho, S).\label{eq:eos}
    \end{align}
\end{subequations}
We notate $\rd{}{t} = \pd{}{t} + \p*{\vec{u} \cdot \vec{\nabla}}$. Note this is
$5$ equations (in 2D) for 5 variables $(\rho, P, S, \vec{u})$ and so is properly
closed.

\subsection{Current Implementation}

We assume $P(\rho, S) = P(\rho)$ a barotropic equation of state, such that $S$
is entirely decoupled from the dynamical equations. We thus drop
\autoref{eq:entropy} of the full fluid equations \autoref{se:comp}. Then, we
will decompose dynamical variables into background and fluctuation quantities $P
= P_0 + P_1, \rho = \rho_0 + \rho_1$, where background quantities are
time-independent. Assuming no background velocities, we write $\vec{u}$ to
denote the fluctuation velocity with no ambiguity.

Furthermore, we will implement incompressibility by mandating
$\at{\pd{P(\rho)}{\rho}}_{ad} \to \infty$ adiabatic derivative. This forces
$\Delta P \gg \Delta \rho$ within a displaced fluid parcel, or $\rd{\rho}{t} =
0$ comoving derivative. This allows us to rewrite \autoref{eq:density}, and so
we arrive at the system of equations
\begin{subequations}\label{se:pseudo_mid}
    \begin{align}
        \vec{\nabla} \cdot \vec{u} &= 0, \label{eq:mid_density}\\
        \rd{\vec{u}}{t} + \frac{\vec{\nabla}P}{\rho} + \vec{g} &= 0,
            \label{eq:mid_velocity}\\
        \rd{\rho}{t} = \rd{\rho_1}{t} + u_z \rd{\rho_0}{z} &= 0.
            \label{eq:mid_eos}
    \end{align}
\end{subequations}

We now assume $P_0 = \rho_0 c_s^2 \propto e^{-z/H}$ isothermal exponential
stratification. At hydrostatic equilibrium, \autoref{eq:mid_velocity} forces
$\vec{\nabla}P_0 = -\rho_0 \vec{g}$. We next claim that $\frac{\rho_1}{\rho_0}
\sim \p*{\frac{u}{c_s}}^2 = \mathrm{Ma}^2 \ll 1$\footnote{I guess this doesn't
have to be true if $\frac{P_0}{\rho_0} = c_s^2 \neq \at{\pd{P}{\rho}}_{ad}$
which is the velocity we are sending to infinity.}. This may be verified to be
true by looking briefly at the momentum equation: $\omega u +
k\p*{\frac{P}{\rho} - \frac{P_0}{\rho_0}} = 0$, which then dividing through by
$P_0 / \rho_0 = c_s^2$ shows that $\frac{P/P_0}{\rho/\rho_0} - 1 \sim
\mathrm{Ma}^2$.

Given this, we expand \autoref{eq:mid_velocity} by $\frac{\vec{\nabla}P}{\rho}
\approx \frac{\vec{\nabla}P_0}{\rho_0} + \frac{\vec{\nabla}P_1}{\rho_0} -
\frac{\rho_1 \vec{\nabla}P_0}{\rho_0^2} + \dots$ and obtain the full system of
equations I have been using
\begin{subequations}\label{se:pseudo_fin}
    \begin{align}
        \vec{\nabla} \cdot \vec{u} &= 0, \label{eq:pseudo_density}\\
        \pd{\vec{u}}{t} + \p*{\vec{u} \cdot \vec{\nabla}}\vec{u}
            + \frac{\vec{\nabla}P_1}{\rho_0}
            + \frac{\rho_1 \vec{g}}{\vec{_0}} - \frac{\rho_1 \vec{\nabla}P_1}{
                \rho_0^2} &= 0,
            \label{eq:pseudo_velocity}\\
        \pd{\rho_1}{t} + \p*{\vec{u} \cdot \vec{\nabla}}\rho_1
            - \frac{u_z \rho_0}{H} &= 0.
            \label{eq:pseudo_eos}
    \end{align}
\end{subequations}
These are complete up to numerical and forcing terms.

\subsection{Anelastic}

We can argue for this in a very formal/nondimensionalized way, but we can also
see this comparatively simply (albeit less rigorously). We will start from
\autoref{se:comp}. Expand all variables in terms of $\epsilon \equiv \mathrm{Ma}
= \frac{u}{c_s}$ where $u$ is the characteristic velocity of the flows we want
to study, then
\begin{align*}
    P &= P_0 + \epsilon P_1 + \dots,& \rho &= \rho_0 + \epsilon \rho_1 + \dots,&
        S &= S_0 + \epsilon S_1 + \dots,& u &\sim \mathcal{O}(\epsilon^1),\\
    \vec{\nabla} &\sim k \sim \mathcal{O}(\epsilon^0) ,&
        \partial_t &\approx uk \sim \mathcal{O}\p*{\epsilon^1}.
\end{align*}
Note that we pick a particular length scale with $k$ and with typical flow
velocity $u$, then $\partial_t = uk$ is the correct order to examine
at\footnote{More formally, we can probably do $\partial_t = \partial_{t_0} +
\partial_{t_1/\epsilon} + \dots$}.

With this dictionary, we may consider the incompressible equations at:
\begin{itemize}
    \item $\mathcal{O}(\epsilon^0)$, which is just
        $\frac{\vec{\nabla}P_0}{\rho_0} + \vec{g} = 0$ hydrostatic equilibrium.

        At this point, we may make the same argument as in the preceeding
        section: $\rd{}{t} \sim \mathcal{O}(\epsilon^2)$, so
        $\frac{\vec{\nabla}P}{\rho} + \vec{g} = \frac{\vec{\nabla}P}{\rho} -
        \frac{\vec{\nabla}P_0}{\rho_0} \sim \mathcal{O}(\epsilon^2)$ as well.
        Thus, $P_1, \rho_1 = 0$ necessarily, otherwise
        $\frac{\vec{\nabla}P}{\rho} - \frac{\vec{\nabla}P_0}{\rho_0}$ must have
        $\mathcal{O}(\epsilon^1)$ terms that are not cancelled by
        $\rd{\vec{u}}{t}$ which is prohibited.

    \item $\mathcal{O}(\epsilon^1)$. Since $P_1 = \rho_1 = 0$, the only terms at
        this order are $\vec{\nabla} \cdot \p*{\rho_0 \vec{u}} = 0$ and
        $\p*{\vec{u}_1 \cdot \vec{\nabla}}S_0 = 0$.

    \item $\mathcal{O}(\epsilon^2)$. This is where the momentum equation gets
        most of its terms, it becomes the familiar
        \begin{equation}
            \rd{\vec{u}}{t} + \frac{\vec{\nabla}P_2}{\rho_0} + \frac{\rho_2
                \vec{g}}{\rho_0} = 0.
        \end{equation}
        Now we haven't argued for $S_1 \neq 0$, so we must permit it as well, so
        the entropy equation becomes
        \begin{equation}
            \rd{S_1}{t} + \p*{\vec{u} \cdot \vec{\nabla}}S_0 = 0.
        \end{equation}

    \item Higher order expansions are not hard to find, and relax $\vec{\nabla}
        \cdot \p*{\rho_0 \vec{u}} = 0$.
\end{itemize}
Thus, to order $\mathrm{Ma}^2$, our equations are
\begin{subequations}\label{se:anel_0}
    \begin{align}
        \vec{\nabla} \cdot \p*{\rho_0 \vec{u}} &= 0,\\
        \rd{S_1}{t} + \vec{u} \cdot \vec{\nabla}S_0 &= 0,\\
        \rd{\vec{u}}{t} + \frac{\vec{\nabla}P_2}{\rho_0}
            + \frac{\rho_2 \vec{g}}{\rho_0} &= 0,\\
        P &= P(\rho, S).
    \end{align}
\end{subequations}
The last step is just to remove $\rho_2$ from the equations using the equation
of state so that the first three equations of \autoref{se:anel_0} above can form
a complete system. We simply follow Glatzmaier and rewrite
\begin{subequations}\label{se:anel_f}
    \begin{align}
        \vec{\nabla} \cdot \p*{\rho_0 \vec{u}} &= 0,\\
        \pd{S_1}{t} + \p*{\vec{u} \cdot \vec{\nabla}}S_1 + \vec{u} \cdot
            \vec{\nabla}S_0 &= 0,\\
        \pd{\vec{u}}{t} + \p*{\vec{u} \cdot \vec{\nabla}}\vec{u} +
            \vec{\nabla}\frac{P_2}{\rho_0} - \frac{S_1\vec{g}}{S_0} &= 0.
    \end{align}
\end{subequations}
Solving this system gives buoyancy waves with frequency $N^2 =
\frac{g}{H_S} = \frac{g}{H} \at{\pd{\ln S_0}{\ln \rho_0}}_P \ll \frac{g}{H}$ as
we obtain $\pd{\ln S_0}{z} = -\frac{1}{H_S}$ stratification using the EoS.

\end{document}

