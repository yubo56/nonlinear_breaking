    \documentclass[
        fleqn,
        usenatbib,
        referee,
    ]{mnras}
    \usepackage{
        amsmath,
        amssymb,
        newtxtext,
        newtxmath,
        ae, aecompl,
        graphicx,
        booktabs,
    }
    \usepackage[T1]{fontenc}
    \usepackage[
        labelfont=bf, % caption names are labeled in bold
        font=scriptsize % smaller font for captions
    ]{caption}
    \usepackage[caption=false]{subfig} % subfigures

    \newcommand*{\scinot}[2]{#1\times10^{#2}}
    \newcommand*{\rd}[2]{\frac{\mathrm{d}#1}{\mathrm{d}#2}}
    \newcommand*{\rtd}[2]{\frac{\mathrm{d}^2#1}{\mathrm{d}#2^2}}
    \newcommand*{\pd}[2]{\frac{\partial#1}{\partial#2}}
    \newcommand*{\ptd}[2]{\frac{\partial^2#1}{\partial#2^2}}
    % inline
    \newcommand*{\mdil}[2]{\mathrm{D}#1/\mathrm{D}#2}
    \newcommand*{\pdil}[2]{\partial#1/\partial#2}
    \newcommand*{\rdil}[2]{\mathrm{d}#1/\mathrm{d}#2}
    \newcommand*{\md}[2]{\frac{\mathrm{D}#1}{\mathrm{D}#2}}
    \newcommand*{\at}[1]{\left.#1\right|}
    \newcommand*{\abs}[1]{\left|#1\right|}
    \newcommand*{\ev}[1]{\left\langle#1\right\rangle}
    \newcommand*{\p}[1]{\left(#1\right)}
    \newcommand*{\s}[1]{\left[#1\right]}
    \newcommand*{\z}[1]{\left\{#1\right\}}
    \newcommand*{\bm}[1]{\mathbf{#1}}
    \newcommand*{\uv}[1]{\hat{\mathbf{#1}}}
    \DeclareMathOperator*{\argmin}{argmin}
    \DeclareMathOperator*{\argmax}{argmax}
    \DeclareMathOperator*{\med}{med}
    \DeclareMathOperator*{\erf}{erf}

\title[Internal Gravity Wave Breaking]{Physics of Tidal Dissipation:
Hydrodynamical Simulations of Internal Gravity Wave Breaking in Stellar
Envelopes and Massive Stellar Binaries}
\author[Y. Su et\ al.]{
Yubo Su,$^1$,
Daniel Lecoanet,$^2$
Dong Lai,$^1$
\\
$^1$ Cornell Center for Astrophysics and Planetary Science, Department of
Astronomy, Cornell University, Ithaca, NY 14853, USA
\\
$^2$ Princeton Center for Theoretical Science, Princeton University, Princeton,
NJ 08544, USA
}

\date{Accepted XXX\@. Received YYY\@; in original form ZZZ}

\pubyear{2019}

\begin{document}\label{firstpage}
\pagerange{\pageref{firstpage}--\pageref{lastpage}}
\maketitle

% cp ../sims/agg.png plots; cp ../sims/2d_4_fourier/snapshots_lin_0_masked/f_amps.png plots/lin_amps.png; cp ../sims/2d_4_fourier/snapshots_lin_0_masked/fluxes.png plots/lin_fluxes.png; cp ../sims/2d_4_fourier/snapshots_yubo_nu1_vhres/f_amps2.png plots/nl_f_amps2.png; cp ../sims/2d_4_fourier/snapshots_yubo_nu1_vhres/f_amps.png plots/nl_f_amps.png; cp ../sims/2d_4_fourier/snapshots_yubo_nu1_vhres/fluxes.png plots/nl_fluxes.png; cp ../sims/2d_4_fourier/snapshots_yubo_nu1_vhres/f_refl.png plots/nl_f_refl.png; cp ../sims/2d_4_fourier/snapshots_yubo_nu1_vhres/f_ri.png plots/nl_f_ri.png; cp ../sims/2d_4_fourier/snapshots_yubo_nu1_vhres/front.png plots/nl_front.png;

\begin{abstract}
    In sufficiently compact white dwarf binaries, dynamical tides raise a train
    of internal gravity waves that propagate towards the surface and dissipate
    via nonlinear wave breaking. We perform 2D hydrodynamical simulations of
    this wave breaking in an incompressible, isothermal atmosphere. After an
    initial transient phase, we find that these waves generate a sharp
    transition, a critical layer, between the non-rotating core and
    synchronously rotating envelope. We find evidence that indefinite steepening
    of this critical layer is suppressed by the Kelvin-Helmholtz instability. We
    study the absorption and reflection of incident waves off the critical layer
    and provide analytical formulae describing its evolution. We infer
    necessary conditions to resolving momentum transfer within the critical
    layer. Finally, we speculate on the application of our model to tidal
    synchronization and heating in astrophysical systems.
\end{abstract}

\begin{keywords}
white dwarfs -- hydrodynamics -- binaries:close -- waves % chktex 8
\end{keywords}

\section{Introduction}\label{s:intro}

Compact white dwarf (WD) binary systems, with orbital periods in the range of
minutes to hours, are important for a range of astrophysical problems. They are
the most important sources of gravitational waves (GWs) for the Laser
Interferometric Space Antenna (LISA) \citep{lisa}. They are also thought to
produce interesting optical transients such as underluminous supernovae
\citep{underlum}, Ca-rich fast transients \citep{carich}, and tidal novae
\citep{tidal_novae}. Most importantly, they have been proposed as the likely
progenitors of type Ia supernovae (e.g. \citet{Ia0,webbink} or more recently
\citet{Ia1,Ia2}). While presently only a few tens of compact WD binaries are
known \citep{lsst_wd}, \emph{Gaia} (currently gathering data) is expected to
expand the catalog to a few hundreds \citep{lsst_wd} \citep[results based on
\emph{Gaia}'s second data release have already begun to
appear][]{gaiaDD,gaiaDD2}, and the Large Synoptic Survey Telescope (LSST, first
light scheduled for 2020) will likely detect a few thousand more
\citep{lsst_wd}. These observations will significantly advance the understanding
of WD binaries and their evolution.

In spite of the broad importance of WD binaries, the evolution of these systems
prior to their final mergers is not well understood. Much of this uncertainty
comes from our imprecise understanding of tidal interactions, which play an
important role during a compact WD binary's inspiral \citep{fullerII}. Previous
studies have shown that these interactions manifest as tidal excitation of
internal gravity waves (IGW), waves in the WD fluid restored by the buoyancy
force due to density stratification \citep{fullerI}. As these waves propagate
outwards towards the WD surface, they grow in amplitude until they break, as do
ocean waves on a shore, and transfer both energy and angular momentum from the
binary orbit to the outer envelope of the WD \citep{fullerI,fullerII}.

Previous works have found that the dissipation of IGW can generate significantly
more energy than thermal radiation from the isolated WD surface and is thus a
major contributor to the WD energy budget \citep{fullerII,fullerIV}. However,
these works parameterized the wave breaking process in an ad hoc manner. The
details of dissipation, namely the location and spatial extent of the wave
breaking, affect the observable outcome: dissipation near the surface of the WD
can be efficiently radiated away and simply brightens the WD, while dissipation
deep in the WD envelope causes an energy buildup that results in energetic
flares \citep{tidal_novae}. Works in other fields based on numerical simulations
show that strongly nonlinear wave breaking behaves differently than predictions
based in linear and weakly nonlinear theory \citep{winters1994,barker_ogilvie}.
Such fully nonlinear numerical simulations have not been performed for WDs.

In this paper, we perform numerical simulations of tidally-excited IGWs
undergoing wave breaking in an idealized WD fluid. The primary aim of the paper
is to thoroughly characterize the angular momentum transfer within a WD by
purely hydrodynamical effects. As a first approximation, we analyze the problem
in a 2D Cartesian geometry, and consider IGW propagating into a fluid initially
at rest. We find that, after an initial transient phase, the fluid develops two
distinct zones: (i) a lower zone that has no horizontal mean flow, and (ii) an
upper zone with significant horizontal mean flow.

These two zones are separated by a \emph{critical layer}. The interaction of IGW
and critical layers plays an important role in the tidal synchronization process
of stellar binaries \citep{zahn75,gn89} as well as in terrestrial phenomena such
as the quasi-biennial oscillation \citep{lindzen_qbo}. The majority of our paper
is dedicated to characterizing the behavior of the critical layer when
interacting with a continuous train of IGW excited from the bottom of the fluid.
IGW are generally \emph{anti}-diffusive, in that they steepen shear flows
\citep{lindzen_qbo,lecoanet_meanflow}, and act to narrow the critical layer. We
find this steepening is suppressed by the Kelvin-Helmholtz instability and
turbulence within the very narrow critical layer. By careful accounting of the
momentum flux budget about the critical layer, we are able to model the
propagation of the critical layer, reflection of the incident IGW, and other
momentum redistribution behaviors. Our results are numerically converged for
sufficiently large resolutions.

In Section~\ref{s:equations}, we will describe the system of equations we will
use to analyze IGW breaking. In Section~\ref{s:theory}, we review existing
understanding of wave breaking and present new analytical results. In
Section~\ref{s:numerics}, we describe our numerical setup, which in
Section~\ref{s:weak_sim} we validate in the weak forcing limit against linear
theory. In Section~\ref{s:sim}, we present the results of simulation of IGW
breaking and our characterization of the critical layer. Finally, we summarize
and conclude in Section~\ref{s:discussion}.

\section{Problem Setup}\label{s:equations}

We consider a stratified, incompressible, isothermal fluid representing a
stellar envelope or atmosphere. We use barotropic equation of state $P(\rho, T)
= P(\rho)$ as a first approximation. We study dynamics in 2D, so that fluid
variables depend only on the Cartesian coordinates $x$ and $z$. While it is
well-known wave breaking is a 3D process \citep{klostermeyer,winters1994}, the
dynamical effect of the breaking process is likely to be similar in 2D
\citep{barker_ogilvie}. We approximate the gravitational field as uniform,
pointing in the $(-\uv{z})$ direction. The plane-parallel approximation is
justified since wave breaking generally occurs near the stellar surface. The
background density stratification is given by
\begin{equation}
    \overline{\rho} = \overline{\rho}_0 e^{-z/H},
\end{equation}
with $\overline{\rho}_0$ some reference density (we generally denote background
quantities with overbars and perturbation quantities with primes).

The Euler equations for an incompressible, barotropic fluid in a uniform
gravitational field are
\begin{subequations}\label{se:nl_orig}
    \begin{align}
        \bm{\nabla} \cdot \bm{u} &= 0,\label{eq:nl_incomp}\\
        \md{\rho}{t} &= 0 ,\label{eq:nl_density}\\
        \md{\bm{u}}{t} + \frac{\bm{\nabla}P}{\rho} + g\uv{z} &=
            0\label{eq:nl_mom},
    \end{align}
\end{subequations}
where $\mdil{}{t} = \pdil{}{t}\,+\,\p{\bm{u} \cdot \bm{\nabla}}$ is the
Lagrangian or material derivative, and $\bm{u}, \rho, P$ denote the velocity
field, density and pressure respectively. We denote $-g\uv{z}$ constant
gravitational acceleration. Note that hydrostatic equilibrium ($\pdil{}{t} = 0$)
of the background implies $\bm{\nabla}\overline{P} = -\overline{\rho} g\uv{z}$,
so that $\overline{P} = \overline{\rho} gH$. A vertically-stratified shear flow
$\overline{u}_x(z)\uv{x}$ is permitted in hydrostatic equilibrium, but we will
assume no such background flow, so $\bm{u} = \bm{u}'$. Physically, this
assumption corresponds to a non-rotating star, or going to the corotating frame
of a rigidly rotating star.

For convenience, we introduce the dimensionless density variable $\Upsilon$ and
the reduced pressure $\varpi$ \citep[e.g.][]{lecoanet_anel} via
\begin{align}
    \Upsilon &\equiv \ln \frac{\rho}{\bar{\rho}},\\
    \varpi &\equiv \frac{P}{\rho}.
\end{align}
These variables automatically enforce $\rho > 0$ and eliminate the stiff term
$\vec{\nabla} P / \rho$ in the Euler equation. In terms of $\Upsilon$ and
$\varpi$, the second two equations in~\eqref{se:nl_orig} become
\begin{subequations}\label{se:nl_upsilon}
    \begin{align}
        \md{\Upsilon}{t} + u_z \pd{\ln \overline{\rho}}{z} &= 0
            ,\label{eq:nl_up_density} \\
        \md{\bm{u}}{t} + \bm{\nabla}\varpi + \varpi\bm{\nabla}\Upsilon
            - \frac{\varpi}{H}\uv{z} + g\uv{z} &= 0\label{eq:nl_upsilon_u}.
    \end{align}
\end{subequations}
Hydrostatic equilibrium corresponds to $\Upsilon = 0, \overline{\varpi} = gH$.

\section{Internal Gravity Waves: Theory}\label{s:theory}

\subsection{Linear Analysis}\label{ss:lin_analysis}

In the small perturbation limit, we may linearize
Eqs.~\eqref{se:nl_upsilon}. The solution in this linear regime is given by
\citep{drazin,sutherland0}
\begin{equation}
    u_z'\p{x, z, t} = Ae^{z/2H}\cos\p{k_{x}x + k_{z}z - \omega t},
        \label{eq:lin_sol}
\end{equation}
where $A$ is a constant amplitude, and the frequency $\omega$ and the wave
number $\p{k_x, k_z}$ satisfy the dispersion relation
\begin{equation}
    \omega^2 = \frac{N^2k_{x}^2}{k_{x}^2 + k_{z}^2 + \p{2H}^{-2}}.
        \label{eq:disp_rel}
\end{equation}
Our equations are valid in the limit of large sound speed ($c_s \to \infty$), in
which the \emph{Brunt-V\"ais\"al\"a frequency} $N$ is given by
\begin{equation}
    N^2 \equiv g^2\p{\rd{\rho}{P} - \frac{1}{c_s^2}} = \frac{g}{H},
\end{equation}
is constant. Other dynamical quantities are simply related to $u'_z$.

In the short-wavelength/WKB limit ($\abs{k_{z}H} \gg 1$), the solution exhibits
the following characteristics:
\begin{enumerate}
    \item The amplitude of the wave grows with $z$ as $e^{z/2H}$. Thus, the
        linear approximation always breaks down for sufficiently large $z$.

    \item The phase and group velocities are given by:
        \begin{align}
            \bm{c}_{p} &=
                \p{k_{x}\uv{x} + k_{z}\uv{z}}\frac{\omega}
                {k_{x}^2 + k_{z}^2 + \p{2H}^{-2}},\\
            \bm{c}_{g} &= N\frac{\s{k_{z}^2 + \p{2H}^{-2}}\uv{x}
                - \p{k_{x}k_{z}\uv{z}}}
                {\s{k_{x}^2 + k_{z}^2 + \p{2H}^{-2}}^{3/2}}.\label{eq:vg}
        \end{align}
        We note $\bm{c}_{p} \cdot \bm{c}_g = \mathcal{O}\s{\p{k_{z}H}^{-2}}
        \approx 0$. In the Boussinesq approximation where terms
        $\mathcal{O}\p{H^{-2}}$ are ignored, the phase and group velocities are
        exactly orthogonal \citep{drazin,sutherland1}. We use the convention
        where upward propagating IGW have $c_{g, z} > 0$, $k_z < 0, k_x > 0$.

    \item The averaged horizontal momentum flux $F$ (in the $+\uv{z}$
        direction) carried by the IGW is defined by
        \begin{equation}
            F(z, t) \equiv \ev{\rho u_{x}' u_{z}'}_x \equiv
                \frac{1}{L_x}\int_0^{L_x}\limits \rho u_{x}'u_{z}'\;\mathrm{d}x.
                    \label{eq:F_def}
        \end{equation}
        The notation $\ev{\dots}_x$ denotes averaging over the $x$ domain. For
        the linear solution (Eq.~\eqref{eq:lin_sol}), this evaluates to
        \begin{equation}
            F \approx -\frac{A^2}{2}\overline{\rho}_0\frac{k_{z}}{k_{x}},
                    \label{eq:S_lin}
        \end{equation}
        Thus, indeed $F > 0$ for an upward propagating IGW ($c_{g, z} > 0$).
\end{enumerate}

\subsection{Wave Generation}

To model continuous excitation of IGWs deep in the stellar envelope propagating
towards the surface, we use a volumetric forcing term to excite IGW near the
bottom of the simulation domain. Our forcing excites both IGWs
propagating upwards, imitating a wave tidally excited deeper in the star, and
downwards, which are not physically meaningful, but are dissipated by a
damping zone described in Section~\ref{ss:damping}.

As not to interfere with the incompressibility constraint, we force the system
on the density equation. We implement forcing with strength $C$ localized around
$z_0$ with small width $\sigma$ by replacing Eq.~\eqref{eq:nl_up_density}
with
\begin{equation}
    \md{\Upsilon}{t} + u_{z}\pd{\ln \overline{\rho}}{z}
        = Ce^{-\frac{(z - z_0)^2}{2\sigma^2}}
            \cos \p{k_{x}x - \omega t}.\label{eq:vol_drive}
\end{equation}
Using a narrow Gaussian profile excites a broad $z$ power spectrum, but only the
$k_{z}$ satisfying dispersion relation (Eq.~\eqref{eq:disp_rel}) for the given
$k_{x}$ and $\omega$ will propagate.

In the linearized system, the effect of this forcing can be solved exactly
(see Appendix~\ref{s:force_solved}). If we approximate $\abs{k_zH} \gg 1, \sigma
\ll H$, the solution can be approximated as two plane waves propagating away
from the forcing zone
\begin{align}
    u_{z}&(x, z, t) \approx{} \frac{C}{2k_z}\frac{gk_x^2}{\omega^2}
        \exp\p{-\frac{k_z^2\sigma^2}{2}}
        \sqrt{2\pi \sigma^2} \nonumber\\
        &{}\times\begin{cases}
        e^{\frac{z - z_0}{2H}}\sin\p{k_{x}x + k_{z}(z - z_0) - \omega t
            + \frac{k_z\sigma^2}{2H}}
            & \text{for }z > z_0,\\[5pt]
        e^{\frac{z - z_0}{2H}}\sin\p{k_{x}x - k_{z}(z - z_0) - \omega t
            + \frac{k_z\sigma^2}{2H}}
            & \text{for }z < z_0.\\
    \end{cases}\label{eq:uz_lin}
\end{align}
The $z > z_0$ region models an upward propagating IGW wavetrain. The $x$
component of the velocity can be obtained by the incompressibility constraint
(Eq.~\eqref{eq:nl_incomp}).

\subsection{Wave Breaking Height}\label{ss:wave_breaking}

As the upward propagating IGW grows in amplitude ($\abs{\bm{u}} \propto
e^{z/2H}$), it is expected to break due to nonlinear effects. We can estimate
the height of wave breaking as when $\abs{\bm{u}} \sim \omega / \abs{\bm{k}}$.
This can be rewritten using the Lagrangian displacement $\bm{\xi} = \bm{u} /
\p{-i\omega}$:
\begin{equation}
    \abs{\xi_z k_z} \gtrsim 1.\label{eq:nl}
\end{equation}

\citet{drazin, klostermeyer, winters1994} describe the onset of wave breaking in
some detail. At intermediate amplitudes, wave breaking occurs via triadic
resonances which transfer energy from the ``parent’’ IGW to ``daughter’’ waves
on smaller length scales that efficiently damp. The horizontal momentum flux
decreases from $F$ to $0$ over this breaking region. The lost flux is deposited
into a horizontal mean flow
\begin{equation}
    \overline{U}(z, t) \equiv \ev{u_x}_x.\label{eq:mean_flow}
\end{equation}
As the mean flow grows, a \emph{critical layer} may form, as discussed below.

\subsection{Critical Layers}\label{ss:crit_layer}

A horizontal shear flow $\overline{U}(z, t)\uv{x}$ enters the fluid equations
via the Lagrangian derivative, which can be decomposed as
\begin{equation}
    \md{}{t} = \pd{}{t} + \overline{U} \pd{}{x} + \p{\bm{u}' \cdot \vec{\nabla}},
\end{equation}
where $\bm{u}'$ is the velocity field \emph{without} the shear flow. Thus,
$\overline{U}$ has the effect of Doppler shifting the time derivative into the
frame comoving with the mean flow. If $\overline{U}$ is roughly constant, then
the behavior of a linear plane-wave perturbation satisfies the modified
dispersion relation
\begin{equation}
    \p{\omega - \overline{U}k_x}^2 =
        \frac{N^2k_{x}^2}{k_{x}^2 + k_{z}^2 + \p{2H}^{-2}}.
        \label{eq:disp_rel_U}
\end{equation}
This is just Eq.~\eqref{eq:disp_rel} with $\omega \to \omega - \overline{U}
k_x$. It is apparent that if $\overline{U} = \overline{U}_c$, where
\begin{equation}
    \overline{U}_c \equiv \frac{\omega}{k_x},\label{eq:u_crit}
\end{equation}
then the dispersion relation is singular and the linear solution breaks down.
Physically, this corresponds to the Doppler-shifted frequency of the IGW being
zero. Anywhere $\overline{U} = \overline{U}_c$ is called a \emph{critical
layer}.

The behavior of an IGW incident upon a critical layer was first studied in the
inviscid, linear regime in \citep{booker_bretherton}, which found nearly
complete absorption of the IGW\@. The amplitude
reflection and transmission coefficients are given by
\begin{align}
    \mathcal{R} &= \exp\p{-2\pi \sqrt{\mathrm{Ri} - \frac{1}{4}}}, &
    \mathcal{T} &= \exp\p{-\pi \sqrt{\mathrm{Ri} - \frac{1}{4}}},
        \label{eq:crit_coeffs}
\end{align}
where $\mathrm{Ri}$ is the local Richardson number evaluated at the critical
layer height $z_c$:
\begin{equation}
    \mathrm{Ri} \equiv \at{\frac{N^2}{\p{\pdil{\overline{U}}{z}}^2}}_{z_c}.
        \label{eq:ri_def}
\end{equation}
In the $\mathrm{Ri} \gg 1$ limit, $\mathcal{R}, \mathcal{T} \ll 1$ and the
incident wave is almost completely absorbed. This result also applies to viscous
fluids \citep{hazel}. However, weakly nonlinear theory \citep{brown_stewartson}
and numerical simulations \citep{winters1994} suggest that nonlinear effects may
significantly enhance reflection and transmission.

Instead, we consider the horizontal momentum transfer at the critical layer. Any
incident horizontal momentum flux $F_a(t)$ absorbed by the fluid must manifest
as additional horizontal momentum of the shear flow. Since $\overline{U}$ cannot
exceed $\overline{U}_c$, the critical layer will instead propagate downward in
response to the incident momentum flux. The horizontal momentum of the shear
flow satisfies
\begin{equation}
    \pd{}{t}\int\limits \overline{\rho}(z) \overline{U}(z, t)\;\mathrm{d}z
        - F_a(t) = 0.
\end{equation}
Assuming $\overline{U}(z > z_c) \approx \overline{U}_c$ and $\overline{U}(z <
z_c) \approx 0$ are both approximately constant, this condition becomes
\begin{equation}
    -\overline{\rho}(z_c) \overline{U}_c\rd{z_c}{t} = F_a(t).\label{eq:zc_anal}
\end{equation}
If $F_a$ is constant in time, $z_c(t)$ has as analytical solution
\begin{equation}
    z_c(t) = -H\ln \s{\exp\p{-\frac{z_c(t = 0)}{H}} +
        \frac{tF_a}{\overline{U}_c H\overline{\rho}_0}},\label{eq:zc_sol}
\end{equation}
where $z_c(t = 0)$ is the initial critical layer height.

\section{Numerical Simulation Setup}\label{s:numerics}

We use the pseudo-spectral code Dedalus \citep{dedalus,dedalus2} to simulate
the excitation and propagation of IGWs (Section~\ref{s:weak_sim}) as well as
their nonlinear breaking and the formation of a critical layer
(Section~\ref{s:sim}).

\subsection{Parameter Choices}\label{ss:params}

We solve Eqs.~\eqref{eq:nl_incomp},~\eqref{eq:nl_upsilon_u},
and~\eqref{eq:vol_drive} in a Cartesian box with size $L_x, L_z$. We choose
periodic boundary conditions in both the $x$ and $z$ direction. We damp
perturbations to zero near the top/bottom of the domain using damping zones
(see Section~\ref{ss:damping}). We expand all variables as Fourier series
with $N_x$ and $N_z$ modes, and use the $3/2$ dealiasing rule to avoid aliasing
errors in the nonlinear terms \citep{boyd}.

The geometry of our simulation domain is fixed by one further parameter, $z_0$,
the forcing location. We choose $L_z = 12.5H$, and the lower and upper damping
zones are located at $z < 1.5H$ and $z > 10.5H$ respectively. The forcing (see
Eq.~\eqref{eq:vol_drive}) is at $z_0 = 3H$ with width $\sigma = 0.078H$,
sufficiently far from the lower damping zone and permitting sufficient room for
the upward propagating wave to grow as $\propto e^{z/2H}$. Finally, we want
similar grid spacing in the $x$ and $z$ directions (i.e.\ $L_x / N_x \sim L_z
/N_z$), guided by the intuition that turbulence generated by wave breaking is
approximately isotropic, so we use $L_x = 4H$ and $N_z / N_x = 4$.

The time integration uses a split implicit-explicit third-order scheme where
certain terms are treated implicitly and the remaining terms are treated
explicitly. A third-order, four-stage DIRK-ERK scheme \citep{ascher} is used
with adaptive timesteps computed from advective Courant-Friedrichs-Lewy (CFL)
time. Specifically, we use $\Delta t = 0.7 \min(\Delta x / u_x,\Delta z /
u_{z})$, where the minimum is taken over every grid point in the domain, and
$\Delta x \equiv L_x / N_x$ and $\Delta z \equiv L_z / N_z$ are the grid
spacings in the $x$ and $z$ directions respectively.

We non-dimensionalize the problem such that $H = N = \rho_0 = 1$. The physics of
the simulation is then fixed by the four remaining parameters $k_{x}$, $\omega$,
$C$, and the viscosity $\nu$. We describe our choices for these parameters
below:
\begin{enumerate}
    \item $k_{x}$: Tidally excited waves in stars generally have $\ell = 2$,
        corresponding to a horizontal wavenumber $k_\perp\sim 1/R$, where $R$ is
        the radius of the star. We use the smallest wavenumber in our
        simulation, $k_x=2\pi/L_x$.

    \item $\omega$: We choose $\omega$ by evaluating the dispersion relation
        $\omega(k_x, k_z)$ for a desired $k_z$ (see Eq.~\eqref{eq:disp_rel}). We
        pick $\abs{k_z H} = 2\pi$ to ensure the waves are very well resolved in
        all of our simulations. Note however that tidally forced IGWs typically
        have $\omega \ll N$, or equivalently $k_r/k_\perp \sim k_r R \gg 1$.
        This requires $\abs{k_z H} \gtrsim 1$, which is only marginally
        satisfied in our simulations.

    \item $C$: In our weak forcing simulations, we first choose the forcing
        strength $C$ (see Eq.~\eqref{eq:vol_drive}) such that $\abs{\xi_z k_z}
        \ll 1$ is satisfied everywhere in the simulation domain. This constrains
        $C$ by Eq.~\eqref{eq:uz_lin}.

    \item $\nu$: Nonlinear effects transfer wave energy from the injection
        wavenumber $\bm{k}$ to larger wavenumbers. Our spectral method does not
        have any numerical viscosity, so diffusivity must be introduced into the
        equations to regularize the systems at large wavenumbers. We add
        viscosity and diffusivity to the system in a way that conserves
        horizontal momentum (see Appendix~\ref{se:strat_impl} for details). To
        quantify $\nu$, we define the dimensionless Reynolds number
        \begin{equation}
            \mathrm{Re}  \equiv \frac{\omega}{\nu k_{z}^2}. \label{eq:re_def}
        \end{equation}
        When the forcing is weak, we may set $\nu \approx 0$, or $\mathrm{Re}
        \gg 1$ because there is negligible energy transfer into higher-order
        modes. Note that $\nu = 0$ changes the order of the PDE system, as the
        highest-order derivatives appear only in the dissipitive terms, so a
        nonzero $\nu$ must be chosen even in weakly forced simulations.
\end{enumerate}

Finally, we use initial conditions $\bm{u}(x, z, 0) = \Upsilon(x, z, 0) = 0$ and
$\varpi(x, z, 0) = 1$, corresponding to hydrostatic equilibrium and no initial
fluid motion.

\subsection{Damping Layers}\label{ss:damping}

We aim to damp disturbances that reach the vertical boundaries of the simulation
domain without inducing nonphysical reflection. To do so, we replace material
derivatives in Eq.~\eqref{se:nl_upsilon} with:
\begin{align}
    \md{}{t} &\to \md{}{t} + \Gamma(z),\\
    \Gamma(z) &= \frac{1}{2\tau}\s{2 + \tanh \frac{z - z_T}{\Delta z}
        + \tanh \frac{z_B - z}{\Delta z}},\label{eq:Gamma}
\end{align}
where $z_B = 1.5H$ and $z_T = 10.5H$ are the boundaries of the lower and upper
damping zones respectively. This damps perturbations below $z_B$ and
above $z_T$ with damping time $\tau$ and negligibly affects the dynamics between
$z_B$ and $z_T$. We choose the transition width $\Delta z = 0.25H$ and damping
time $\tau = 1 / (15N)$. This prescription is similar to \citet{lecoanet_damp}
and has the advantage of being smooth, important for spectral methods. Further
details of our implementation of the fluid equations in Dedalus are described in
Appendix~\ref{se:strat_impl}.

\section{Weakly Forced Numerical Simulation}\label{s:weak_sim}

To test our numerical code and implementation, we carry out a simulation in the
weakly forced regime with $C = \scinot{1.64}{-7}$. According to the linear
solution (Eq.~\eqref{eq:uz_lin}), this generates IGW with $\abs{\xi_z k_z}
\approx \scinot{5}{-5}$ just above the forcing zone. This amplitude is
sufficiently small to ensure $\abs{\xi_z k_z} \ll 1$ in the entire simulation
domain. We include a nonzero $\nu$ corresponding to $\mathrm{Re} = 10^7$.

We expect the waves to follow the analytical solution given by
Eq.~\eqref{eq:uz_lin} and the corresponding $u_x(x, z, t)$; we denote this
solution $\bm{u}_{al}(x, z, t)$. The amplitude of the observed IGW in the
simulation field $\bm{u}$ relative to analytical solution $\bm{u}_{al}$ over
some region $z \in [z_b, z_t]$ can be estimated from
\begin{equation}
    A_i(t) = \frac{\int\limits_{z_b}^{z_t}\int\limits_0^{L_x}
        \overline{\rho}\p{\bm{u} \cdot \bm{u}_{al}}\;\mathrm{d}x\mathrm{d}z}
        {\int\limits_{z_b}^{z_t}\int\limits_0^{L_x}
        \overline{\rho}\abs{\bm{u}_{al}}^2\;\mathrm{d}x\mathrm{d}z}.
        \label{eq:ahat_def}
\end{equation}
The subscript $i$ denotes the incident wave. If $\bm{u} = \bm{u}_{al}$, then
$A_i(t) = 1$. The normalization in Eq.~\eqref{eq:ahat_def} is chosen such that
the overlap between $\bm{u}, \bm{u}_{al} \propto e^{z/2H}$ is evenly weighted
throughout the integration region.

For the weakly forced simulation, we expect
$A_i(t) = 1$ when integrated between the forcing and damping zones, i.e.\
$z_b \gtrsim z_0$ and $z_t \lesssim z_T$ ($z_0, z_T$ are defined in
Eq.~\eqref{eq:vol_drive} and Eq.~\eqref{eq:Gamma} respectively). For consistency
with the nonlinear case later, we choose $z_b = z_0 + 3\sigma$ and $z_t = z_b +
H$. Note that using a larger integration domain by choosing $z_t = z_T - \Delta
z$ just below the upper damping zone instead does not change the measured $A_i$.
The resulting measurement of $A_i(t)$ is shown in Fig.~\ref{fig:lin_amps}, and
indeed $A_i \approx 1$ after the initial transient.
\begin{figure}
    \centering
    \includegraphics[width=0.9\columnwidth]{plots/lin_amps.png}
    \caption{Amplitude of the excited IGW over time (in units of $N^{-1}$) in
    the weakly forced simulation, computed using Eq.~\eqref{eq:ahat_def}.
    $A_i(t) = 1$ corresponds to perfect agreement with the analytical estimate.
    After an initial transient phase, we observe $A_i(t)$ asymptotes to $\approx
    1$, implying continuous excitation of identical IGW with the expected
    amplitude. The small deviation of $A_i(t)$ from unity may be due to
    timestepping errors, as a relatively large fixed step size $\Delta t =
    0.1/N$ was used for this simulation.}\label{fig:lin_amps}
\end{figure}

The analytical theory (Section~\ref{ss:lin_analysis}) also predicts that the
horizontal momentmum flux $F(z, t)$ is independent of $z$ between the forcing
zone where the wave is generated and the damping zone where it is dissipated.
The expected horizontal momentum flux carried by the excited IGW in the linear
theory can be computed by simply evaluating Eq.~\eqref{eq:F_def} for
$\bm{u}_{al}$ and is a constant:
\begin{equation}
    F_{al} \equiv \ev{\rho u_{al, x} u_{al, z}}_x.\label{eq:F_al}
\end{equation}
Denote the momentum flux measured in the simulation by $F(z, t)$
(use Eq.~\eqref{eq:F_def} with velocities taken from the simulation), then we
expect $F(z, t) = F_{al}$ between $z_0$ and $z_T$. Fig.~\ref{fig:lin_fluxes}
shows agreement with this prediction.
\begin{figure}
    \centering
    \includegraphics[width=0.9\columnwidth]{plots/lin_fluxes.png}
    \caption{$F/F_{al}$ as a function of $z$ at select times $t$ (in units of
    $N^{-1}$). As the initial transient dies out, $F / F_{al} \approx 1$ to a
    good approximation above the forcing zone $z > z_0 = 2H$ and below the
    damping zone $z \lesssim z_T = 9.5H$. The horizonal momentum flux excited in
    the forcing zone is transported without loss to the top of the domain, where
    it is dissipated by the damping zone (see Section~\ref{ss:damping}) without
    reflection.}\label{fig:lin_fluxes}
\end{figure}

% GOT HERE

\section{Numerical Simulations of Wave Breaking}\label{s:sim}

% times on fluxes.png are 054, 086, 153, 451

To perform simulations of wave breaking phenomena, we use the same setup as
described in Section~\ref{s:numerics} and Section~\ref{s:weak_sim} except for
different values of $C$ and $\nu$. In particular, we choose $C$ such that
$\abs{\xi_z k_z} = 0.1$ in the forcing zone ($z = z_0$). The linear solution
predicts $\abs{\xi_z k_z} \sim 4.25$ at the upper damping zone $z_T$. We choose
the viscosity $\nu$ so that $\mathrm{Re}$ is as large as possible across the
various resolutions. A table of our simulations can be found in
Table~\ref{tab:params}.
\begin{table}
    \centering
    \begin{tabular}{l c c c}
        Resolution & $\mathrm{Re}$\\\bottomrule
        $1024 \times 4096$ & $2048$\\
        $768 \times 3072$ & $1024$\\
        $512 \times 2048$ & $512$\\
        $256 \times 1024$ & $341$\\
        $256 \times 1024$ & $205$\\
        $256 \times 1024$ & $146$\\
    \end{tabular}
    \caption{Table of simulation resolutions with wave
    breaking.}\label{tab:params}
\end{table}

\subsection{Numerical Simulation Results}\label{ss:nl_ns}

A full video of our simulation with $N_x = 768$, $N_z = 3072$, $\mathrm{Re} =
1024$ is available
online\footnote{http://www.princeton.edu/~lecoanet/data/breaking\_wave.mov}. We
take this to be our fiducial simulation for the remainder of this paper, though
other simulations show qualitatively similar behavior.

% convert yubo_000054.png -crop 2000x2000+150+250 out.png
In Fig.~\ref{fig:snapshots}, we present snapshots of $u_x$ and $\Upsilon$ at
various phases of the simulation. The flow evolves through several distinct
stages:
\begin{enumerate}
    \item At early times (top left panel), the flow resembles a linear IGW lower
        in the simulation domain but breaks down into smaller-scale features at
        higher $z$. Some characteristic swirling motion can be seen in the
        advected scalar $\Upsilon$, indicating Kelvin-Helmholtz instabilities.

    \item At a slightly later time (top right panel), the mean flow in
        $u_x$ becomes much more prominent and the critical layer $z_c$ has
        become much more definite. Small-scale fluctuations are still present in
        $u_x$ but at smaller amplitudes due to being in a denser region of the
        fluid.

    \item In the bottom left panel, the critical layer transition becomes very
        sharp, and small swirls of limited vertical extent in $\Upsilon$ at the
        location of the critical layer suggest that the Kelvin-Helmholtz
        instability is responsible for regulating the width of this transition.
        More discussion can be found in Section~\ref{ss:khi}.

    \item At the end of the simulation (bottom right panel), the flow shows very
        few significant qualitative differences from the previous snapshot,
        suggesting that the latter phase of the simulation has reached a steady
        state.
\end{enumerate}

\begin{figure*}
    \includegraphics[width=0.45\textwidth]{plots/yubo_000054.png}\hfil
    %
    \includegraphics[width=0.45\textwidth]{plots/yubo_000086.png}

    \includegraphics[width=0.45\textwidth]{plots/yubo_000153.png}\hfil
    %
    \includegraphics[width=0.45\textwidth]{plots/yubo_000451.png}
    \caption{Snapshots of $u_x$ (in units of $HN$) and $\Upsilon \equiv
    \ln\p{\rho / \bar{\rho}}$ in the fiducial simulation illustrating distinct
    phases of the evolution of the flow. Note that $\overline{U}_c \approx
    0.16HN$ (see Eq.~\eqref{eq:u_crit}) for the parameters used. The four times
    depicted are $t = 413.4/N$ (top left), $t = 658.5/N$ (top right), $t =
    1171.4/N$ (bottom left), and $t = 3437.8/N$ (bottom right). The four
    snapshots illustrate (i) the initial transient wave breaking phase, (ii)
    formation of a distinct critical layer, (iii) steepening of the critical
    layer, and (iv) downward advance of the critical
    layer.}\label{fig:snapshots}
\end{figure*}

In Fig.~\ref{fig:nl_fluxes}, we plot the mean horizontal flow velocity
$\overline{U}$ (Eq.~\eqref{eq:mean_flow}) and the dimensionless momentum flux $F
/ F_{al}$ (Eqs.~\eqref{eq:F_def} and~\eqref{eq:F_al}) as a function of $z$ at
the times depicted in Fig.~\ref{fig:snapshots}. At each time, $\overline{U}$ is
close to zero below the critical layer, but then sharply increases to
$\overline{U}_c$ at the critical layer (i.e.\ the flow is ``spun-up''). Above
the critical layer, $\overline{U}$ varies slightly due to momentum transport
within the spun-up layer. This agrees with the expectation discussed in
Section~\ref{ss:crit_layer}.

Similarly, $F \lesssim F_{al}$ below the critical layer, and then decreases to
about zero above the critical layer. However, two notable deviations from the
discussion in Section~\ref{ss:crit_layer} can be observed: (i) the incident flux
on the critical layer fluctuates somewhat temporally, and (ii) there is a small
negative flux just above the critical layer at later times. These are addressed
in subsequent sections.
\begin{figure}
    \centering
    \includegraphics[width=0.9\columnwidth]{plots/nl_fluxes.png}
    \caption{The mean horizontal flow velocity $\overline{U}(z, t)$
    (Eq.~\eqref{eq:mean_flow}) and the dimensionless momentum flux $F(z, t) /
    F_{al}(z)$ (Eqs.~\eqref{eq:F_def} and~\eqref{eq:F_al}) are plotted at the
    same times as in Fig.~\ref{fig:snapshots} as a function of $z$ in our
    fiducial simulation. The two distinct zones of mean flow are separated by a
    critical layer. The propagation of this critical layer towards lower $z$ and
    the sharp deposition of $F$ at the critical layer are
    evident.}\label{fig:nl_fluxes}
\end{figure}

\subsection{Kelvin-Helmholtz Instability and Critical Layer Width}\label{ss:khi}

The formation of the critical layer is associated with a strong shear flow. What
is the width of this layer? Inspection of Fig.~\ref{fig:snapshots} suggests the
presents of the Kelvin-Helmholtz Instability (KHI) in the critical layer. In a
stratified medium, KHI occurs when the Richardson number (Eq.~\eqref{eq:ri_def})
satisfies $\mathrm{Ri} \lesssim 1/4$ \citep[e.g.][]{shu1991physics}. It is
natural to suspect that the shear flow cannot steepen further than the onset of
KHI\@. To test this, we compute the local $\mathrm{Ri}$ for the shear flow
around the critical layer.

To avoid noisiness, $\mathrm{Ri}$ is measured as follows: we first assign an
$\mathrm{Ri}_x(x, t)$ for every $x$ in the critical layer, then take the median
as $\mathrm{Ri}$ for the entire layer. $\mathrm{Ri}_x$ is computed using the
vertical distance over which the local $u_x$ increases from $0.3\bar{U}_c$ to
$\bar{U}_c$ (see Eq.~\eqref{eq:u_crit}). The value $0.3$ is necessary to exclude
the small mean flow generated in the weakly nonlinear regime far below the
critical layer. This procedure can be written:
\begin{align}
    z_{CL, \min}(x, t) &\equiv \argmin_z
        \z{z\mid u_x(x, z, t) > 0.3\overline{U}_c},\\
    z_{CL, \max}(x, t) &\equiv \argmax_z
        \z{z\mid u_x(x, z, t) < \overline{U}_c},\\
    \mathrm{Ri}_x(x, t) &\equiv
        \p{\frac{N^2 \p{z_{CL, \max} - z_{CL, \min}}^2}{(0.7
            \overline{U}_c)^2}},\\
    \mathrm{Ri}(t) &\equiv \med_x\mathrm{Ri}\p{x, t}.\label{eq:ri_med_def}
\end{align}
To understand the variation in $\mathrm{Ri}$ over $x$, we can also compute
$\min\limits_x \mathrm{Ri}_x(x, t)$ (the maximum is very noisy). Both of these
are shown in Fig.~\ref{fig:nl_f_ri}. The critical layer quickly narrows to its
minimum width, driven by the anti-diffusivity of IGWs.
\begin{figure}
    \centering
    \includegraphics[width=0.9\columnwidth]{plots/nl_f_ri.png}
    \caption{Local Richardson number (Eq.~\eqref{eq:ri_med_def}) of the flow at
    the critical layer over time (in units of $N^{-1}$) in our fiducial
    simulation. The red and solid green lines denote respectively the minimum
    and median of $\mathrm{Ri}_x(x, t)$. These numbers measure the mean and
    spread in width of the critical layer over $x$. Note that $\mathrm{Ri} \sim
    \frac{1}{4}$ corresponds to the KHI, so this plot suggests the shear at the
    critical layer does not steepen past the onset of the
    KHI.}\label{fig:nl_f_ri}
\end{figure}

\subsection{Flux Budget}\label{ss:flux_budget}

Knowing now the width of the critical layer, we next seek its location. The
downward propagation of the critical layer location $z_c(t)$ is driven by the
absorption of horizontal momentum flux at $z_c$, following
Eq.~\eqref{eq:zc_anal}. When analyzing the simulation, one must first compute
$z_c$ before examining the flux budget in its vicinity, but conceptually $z_c$
is a consequence of flux redistribution at the critical layer. Here, we proceed
in conceptual order and first describe how we analyze the flux transfer at
$z_c$.

In general, the flux budget at the critical layer can be decomposed as
\begin{equation}
    F_i(t) = F_a(t) + F_r(t) + F_s(t),\label{eq:f_budget}
\end{equation}
where $F_i$ is the incident flux, $F_a$ is the absorbed flux, $F_r$ is the
reflected flux, and $F_s$ is the ``residual'' flux above the critical layer,
likely responsible for momentum redistribution within the synchronized upper
layer. Careful accounting of $F_s$ turns out to be important to obtain the
correct $F_a$ and resulting critical layer propagation. A more specific physical
interpretation of $F_s$ is unclear; it is somewhat tempting but unfounded to
identify $F_s$ with the transmitted flux. In the simulation, we find $F_s < 0$.

After measuring $z_c$ (see Section~\ref{ss:cl_prop}) and $F(z)$
(Eq.~\eqref{eq:F_def}) at each time step, we can then determine each of $F_i$,
$F_a$, $F_r$, $F_s$ as follows:
\begin{align}
    F_i(t) &= F_{al}A_i^2(t),\\
    F_i(t) - F_r(t) &= \ev{F(z, t)}_{z \in [z_c - \Delta z - H, z_c - \Delta z]}
        ,\label{eq:fr_def}\\
    F_s(t) &= \ev{\z{F(z, t): F(z, t) < 0}}_{z \in [z_c, z_c + \Delta z]},
        \label{eq:fs_def}\\
    F_a(t) &= F_i - F_r - F_s.\label{eq:fa_def}
\end{align}
Here, $\ev{\dots}_{z \in [z_a, z_b]}$ denotes a vertical average over the
interval $[z_a, z_b]$. Below the critical layer, we average over an interval of
length $H$, also the vertical wavelength. The offset $\Delta z$ is necessary to
make the measurement of the incident flux unaffected by the turbulence within
the critical layer itself. The width of the critical layer is limited by
$\mathrm{Ri} \lesssim 1$ (see Section~\ref{ss:khi}), which bounds its vertical
extent $\sim \frac{1}{\abs{k_{z}}}$. We empirically found an offset of $\Delta z
= \frac{3}{\abs{k_z}}$ was necessary to be sufficiently far from strong
fluctuations near the critical layer.

Above the critical layer, we observe that the $F_s$ feature has varying width
(compare e.g.\ the $t = 1171.4/N$ and $t = 3437.8/N$ lines in the bottom panel
of Fig.~\ref{fig:nl_fluxes}) but contributes significantly to the total
flux budget. We average only where $F < 0$ so that $F_s$ is robust to such width
variations. We find that this is a sufficiently accurate way of measuring $F_s$
and determining $F_a$.

Fig.~\ref{fig:nl_f_amps2} depicts the four components of this flux decomposition.
\begin{figure}
    \centering
    \includegraphics[width=0.9\columnwidth]{plots/nl_f_amps2.png}
    \caption{Momentum flux decomposition calculated from the simulation. Plotted
    are the four components of the horizontal momentum flux budget over time
    (see Eq.~\eqref{eq:f_budget}), in units of the analytical estimate for the
    incident wave flux $F_{al}$ (Eq.~\eqref{eq:S_lin}): $F_i$, the flux incident
    on the critical layer (green); $F_a$, the flux absorbed by the critical
    layer (blue); $F_r$, the flux reflected at the critical layer (red); and
    $F_s$, the flux inside the synchronized layer
    (black).}\label{fig:nl_f_amps2}
\end{figure}


\subsection{Critical Layer Propagation}\label{ss:cl_prop}

With a careful determination of $F_a$, we can next make predictions for the
propagation of $z_c(t)$ and compare to the measured propagation in the
simulation.
In principle, $z_c$ is the location where the incident flux significantly
attenuates. In the simulation, shear turbulence causes $F$ to have significant
spatial and temporal fluctuations that translate to large temporal fluctuations
in $z_c(t)$. To minimize these spurious fluctuations, we measure the location of
the critical layer using a spatial average of where flux deposition occurs:
\begin{align}
    z_{c, \min}(t) &\equiv \argmin_z \z{z: F(z, t) > 0.3F_{al}},\\
    z_{c, \max}(t) &\equiv \argmax_z \z{z: F(z, t) < 0.3F_{al}},\\
    z_c(t) &\equiv \frac{z_{c, \min}(t) + z_{c, \max}(t)}{2}.\label{eq:zc_def}
\end{align}
Measuring $z_c$ in other ways does not significantly change the results of the
analysis.

Finally, in Fig.~\ref{fig:nl_front} we plot the numerically measured $z_c$
against two semi-analytic predictors: (i) integration of Eq.~\eqref{eq:zc_anal}
using the measured $F_a(t)$, and (ii) substituting the time-averaged
$\ev{F_a}_t$ (over the entire length of the simulation) into
Eq.~\eqref{eq:zc_sol}. Since $z_c(t)$ is less well-defined at early times (when
the critical layer is thick and transient behavior is strong), we solve
Eq.~\eqref{eq:zc_anal} by integrating backwards from the end of the simulation
($t = t_f$), using $z_c(t_f)$ as the initial condition. From
Fig.~\ref{fig:nl_front}, we see that the agreement between the measured $z_c(t)$
and its estimate via $F_a(t)$ and Eq.~\eqref{eq:zc_anal} is excellent.

By time-averaging the numerically measured $F_a$, we find $\ev{F_a}_t \approx
0.71F_{al}$, which is used for the second predictor in Fig.~\ref{fig:nl_front}.
Note that $F_a < F_{al}$, so momentum flux absorption at the critical layer is
incomplete.
% Moreover, careful comparison of the two predicted curves in
% Fig.~\ref{fig:nl_front} shows that there is . This suggests that
% momentum flux absorption changes significantly over time, in agreement with
% detailed analysis in Section~\ref{ss:reflectivity}.
\begin{figure}
    \centering
    \includegraphics[width=0.9\columnwidth]{plots/nl_front.png}
    \caption{Propagation of the critical layer over time. Shown are: (black)
    $z_c(t)$ from simulation data, (green) predictor of $z_c(t)$ using direct
    integration of Eq.~\eqref{eq:zc_anal} for $F_a(t)$ measured from
    simulation data (described in Eq.~\eqref{eq:fa_def}), and (blue) direct
    substitution of time-averaged $\ev{F_a(t)}_t$ into
    Eq.~\eqref{eq:zc_sol}. Predictors use the end of the simulation as initial
    conditions and integrate backwards, as $z_c$ is less well-defined at early
    times. The agreement of the directly-integrated predictor with the data
    shows Eq.~\eqref{eq:zc_anal} is a good description of the evolution of $z_c$.
    The the poorer but qualitatively correct agreement of the time-averaged
    predictor with the data shows both that $F_a(t) <
    F_{al}$.}\label{fig:nl_front}
\end{figure}

\subsection{Non-absorption at Critical Layer}\label{ss:reflectivity}

To further understand the behavior at the critical layer, we compare two
reflective behaviors observed in the simulation: (i) the presence of a reflected
wave with wave vector $\bm{k}_r = k_{x}\uv{x} - k_{z}\uv{z}$, and (ii) the
reflected flux $F_r$. The reflected wave amplitude and flux need not agree
exactly if some reflected flux is in higher-order modes, which is indeed the
case in our simulations. Both are of physical interest, however: the reflected
wave amplitude is essential for setting up standing modes in a realistic star,
while the flux is important for accurately tracking angular momentum transfer
during synchronization.

To measure the reflected wave amplitude $A_r(t)$, we use an approach similar to
the calculation of $A_i(t)$ (Eq.~\eqref{eq:ahat_def}):
\begin{equation}
    A_r(t) = \max_{\delta x}\frac{\int\limits_{z_b}^{z_t}\int\limits_0^{L_x}
        \overline{\rho}\p{\bm{u} \cdot \at{\bm{u}_{al,
        \bm{k}_r}}_{x = x + \delta x}}\;\mathrm{d}x\mathrm{d}z}
        {\int\limits_{z_b}^{z_t}\int\limits_0^{L_x}
        \overline{\rho}\abs{\bm{u}_{al}}^2\;\mathrm{d}x\mathrm{d}z},
        \label{eq:ar_def}
\end{equation}
where $z_b = z_0 + 3\sigma$ and $z_t = z_b + H$ as before. The primary
difference from Eq.~\eqref{eq:ahat_def} is the introduction of free parameter
$\delta x$, the horizontal phase offset of the reflected wave. Since $\delta x$
is unknown \emph{a priori}, we choose $\delta x \in [0, 2\pi]$ that maximizes
$A_r(t)$. In our simulation, the phase offset $\phi_r(t) \equiv k_x \delta x(t)
$ is consistent with reflection off a moving boundary at $z_c$, well
approximated by $\abs{\pdil{\phi_r}{t}} \simeq 2\abs{\pdil{(k_{z}z_c)}{t}}$.
Finally, note that $\bm{u}_{al}$ and $\bm{u}_{al, \bm{k}_r}$ are orthogonal
under the integral used for Eqs.~\eqref{eq:ahat_def} and~\eqref{eq:ar_def}, so
$A_i$ and $A_r$ are independent.

Fig.~\ref{fig:nl_f_amps} illustrates the behaviors of these two quantities. Both
of them vary significantly in time but their mean values appear to converge
towards the end of the simulation. A modest ($\sim20\%$), long-term fluctuation
in $A_i(t)$ is observed, likely of numerical origin.
\begin{figure}
    \centering
    \includegraphics[width=0.9\columnwidth]{plots/nl_f_amps.png}
    \caption{Wave amplitudes observed in the simulation. The
    top panel shows the incident wave amplitude $A_i(t)$ (green) and the
    reflected wave amplitude $A_r(t)$ (red) just above the forcing
    zone. The modest fluctuation in $A_i(t)$ is likely of numerical origin but
    has no bearing on our results.}\label{fig:nl_f_amps}
\end{figure}

Since $A_i(t), A_r(t)$ vary somewhat over time, we perform time averaging over
interval of approximately $8\pi/\omega$, denoted by angle brackets. We can then
define the amplitude reflectivity
\begin{equation}
    \mathcal{R}_A(t) \equiv \frac{\ev{A_r}(t)}{\ev{A_i}(t)}
        .\label{eq:Ra_def}
\end{equation}
% 7/N per timestep, 13 timestep smoothing kernel, 24/N period

This reflectivity can be compared to the flux redistribution at the critical
layer by calculating $\mathcal{R}_A^2(t)$ and the ratios of $F_r$ and $-F_s$ (as
$F_s < 0$) to $F_i$. A comparison between $\mathcal{R}_A^2$ and $\hat{F}_r$ is
appropriate as $F \propto A^2$ (Eq.~\eqref{eq:S_lin}). We define \begin{align}
\hat{F}_r &\equiv \frac{\ev{F_r}(t)}{\ev{F_i}(t)}, \label{eq:srefl_def1}\\
\hat{F}_s &\equiv -\frac{\ev{F_s}(t)}{\ev{F_i}(t)}. \label{eq:srefl_def}
\end{align} Fig.~\ref{fig:nl_f_refl} shows $\mathcal{R}_A^2$, $\hat{F}_r$,
$\hat{F}_s$ using the results of Fig.~\ref{fig:nl_f_amps2} and
Fig.~\ref{fig:nl_f_amps} as inputs. Visual inspection suggests the three
quantities have reached their asymptotic values (in the time-averaged sense) for
$t \gtrsim 2500/N$. The aforementioned fluctuations in $A_i$ do not affect our
reflectivity results thanks to the time averaging used in
Eqs.~\eqref{eq:Ra_def}--\eqref{eq:srefl_def}.% chktex 2
We see that in general $\hat{F}_r \gtrsim \mathcal{R}_A^2$, conforming with the
expectation that the reflected flux consists of the simple reflected mode and
higher order modes as well.
\begin{figure}
    \centering
    \includegraphics[width=0.9\columnwidth]{plots/nl_f_refl.png}
    \caption{Behavior of dimensionless reflectivities $\mathcal{R}_A^2$,
    $\hat{F}_r$, and residual flux $\hat{F}_s$
    (Eqs.~\eqref{eq:Ra_def}--\eqref{eq:srefl_def}) as a function of time (in
    units of $N^{-1}$) in the simulation. These quantities seem to become
    comparatively stable past about $t = 2500/N$, indicating that an asymptotic
    value may have been reached. That $\hat{F}_r \gtrsim \mathcal{R}_A^2$
    implies a substantial fraction of reflected flux is in higher-order modes
    than the reflected IGW.}\label{fig:nl_f_refl}
\end{figure}

\subsection{Resolution Study}\label{ss:convergence}

As the primary test of whether our simulations have reached sufficient
resolution, we consider the convergence of $\mathcal{R}_A^2$, $\hat{F}_r$,
$\hat{F}_s$. Furthermore, we compare our results against
Eqs.~\eqref{eq:crit_coeffs}, which suggest that wave reflection at the critical
layer depends only on the local Richardson number (see Section~\ref{ss:khi} for
details on our measurement of $\mathrm{Ri}$).

For each simulation in Tab.~\ref{tab:params}, we compute the median values of
$\mathcal{R}_A^2$, $\hat{F}_r$, $\hat{F}_s$, and $\mathrm{Ri}$
(Eqs.~\eqref{eq:Ra_def}--\eqref{eq:srefl_def} and~\eqref{eq:ri_def}
respectively) over the last $1/4$ of the simulation time from each simulation,
when these quantities have converged to their asymptotic values. These results
are shown in Fig.~\ref{fig:agg}.

\begin{figure}
    \centering
    \includegraphics[width=0.9\columnwidth]{plots/agg.png}
    \caption{Convergence of the median $\hat{F}_r$, $\mathcal{R}_A^2$,
    $\hat{F}_s$, and $\mathrm{Ri}$ (Eqs.~\eqref{eq:Ra_def}--\eqref{eq:srefl_def}
    and~\eqref{eq:ri_def} respectively) across runs with varying resolution and
    viscosity as given in Tab.~\ref{tab:params}. Vertical bars show the temporal
    variation of each measurement between the $16\%$ and $84\%$ range. Small
    horizontal displacements are made for data points at identical $\mathrm{Re}$
    for readability. Note that simulations with larger $\mathrm{Re}$ correspond
    to smaller viscosity and are more physically realistic.  At the smallest
    $\mathrm{Re}$ value, $\mathrm{Ri} \approx 50$ is too large to fit on the
    plot.}\label{fig:agg}
\end{figure}

It is apparent that the Richardson number rapidly converges to $\mathrm{Ri}
\approx 0.4$ but the reflectivity measure converges more slowly. This disagrees
with Eq.~\eqref{eq:crit_coeffs}, which is based on the linear theory. This
tension is natural: fluid motion within the critical layer is turbulent, so
reflecton at the critical layer cannot be captured by the linear theory.

\section{Summary and Discussion}\label{s:discussion}

\subsection{Key Results}

In this paper, we have performed numerical simulations of nonlinear breaking of
IGWs in a stratified isothermal atmosphere. Such a setup represents the
plane-parallel idealization of the outer envelope of massive stars, WDs, or
planets. Our simulations use the spectral code Dedalus \citep{dedalus,dedalus2},
and are carried out in 2D. We observe spontaneous formation of a critical layer
that separates a synchronized upper layer of fluid and a lower layer of fluid
with no mean horizontal flow. This critical layer then propagates downwards as a
continuous train of IGW are incident on it (see Fig.~\ref{fig:snapshots} for
snapshots from our fiducial simulation). Our primary conclusions regarding the
evolution of the critical layer are as
follows:
\begin{enumerate}
    \item The width of the turbulent critical layer is determined by requiring
        the local Richardson number (Eq.~\eqref{eq:ri_def}) $\mathrm{Ri} \gtrsim
        0.5$, as shown in Fig.~\ref{fig:nl_f_ri}.

    \item The location of the critical layer $z_c(t)$ can be predicted by
        careful measurement of the absorbed flux at the critical layer
        (see Eq.~\eqref{eq:zc_anal}) as shown in Fig.~\ref{fig:nl_front}.

    \item Flux absorption at the critical layer is incomplete. The
        critical layer only absorbs $\sim 70\%$ of the incident flux in most
        simulations (see Fig.~\ref{fig:agg} for convergence at high
        resolutions). The reflected flux is carried away from the critical layer
        as both a lowest-order reflected wave and waves with larger $z$
        wavenumbers.
\end{enumerate}
In the subsequent sections, we will discuss the validity of these results and
their application to astrophysical systems.

\subsection{Physical Sources of Dissipation in WDs}\label{ss:disp}

The most significant linear damping in WD g-modes comes from radiative damping
\citep{fullerI}. In \citep{wu} and \citep{fullerI}, the radiative damping rate
is given in terms of $\omega_i = \gamma \omega_r$, where $\omega_r$ is the
frequency of the g-mode. Typical values for $\gamma$ range from $10^{-4}$ to
$10^{-11}$ depending on $n$ of the g-mode.

We will assume this prescription directly transfers to propagating IGW, which
results in general agreement with \citep{bukart}'s estimate of radiative damping
rates. Then, making coarse identification $\omega_i \sim \nu k^2 \approx \nu
k_z^2$, we find that $\mathrm{Re} \sim \frac{1}{\gamma}$. Even at $\gamma =
10^{-4}$ however, the corresponding $\mathrm{Re}$ is far too weak to suppress
reflection/transmission at the critical layer (Fig.~\ref{fig:agg}).

Another source of dissipation considered in \citep{bukart} is turbulent
conbmtive damping. They find this damping rate to never exceed that of
radiative damping, and so it is also too weak to suppress critical layer
formation in our problem.

Finally, we consider the impact of magnetic winding. In \citep{bukart}, magnetic
winding is used to enforce solid body rotation on the grounds that $t_A \gg
t_{gw}$, where
\begin{equation}
    t_A = \int\limits_0^R \frac{\sqrt{4\pi \rho}}{B_0}\;\mathrm{d}r
        \sim 10^2\;\mathrm{yr}\p{\frac{10^3\;\mathrm{G}}{B}},
\end{equation}
the Alfv\'en wave crossing time (evaluated for a CO WD in \citet{fullerIV}),
measures the magnetic coupling time and $t_{gw}$ measures the gravitational wave
inspiral timescale. Before solid body rotation is attained, another relevant
timescale is the synchronization timescale $t_s$. For a tidal torque $\tau$ and
tidal forcing frequency $\sigma = m\p{\Omega - \Omega_{spin}}$, we note that
angular momentum transfer is $\pd{M_{sync}}{t} \sigma R^2 = \tau$, where
$M_{sync}$ denotes the mass of the WD that has synchronized. Thus, the
synchronization timescale is
\begin{align}
    t_{sync} &\sim \frac{M_{sync}\sigma R^2}{\tau},\\
        &\sim 20\;\mathrm{yr}
            \p{\frac{M_{sync}}{10^{-4}M_{\odot}}}
            \p{\frac{\sigma}{2\pi / (1\;\mathrm{hr})}}
            \p{\frac{R}{R_{\oplus}}}^2
            \p{\frac{10^{-14} GM_{\odot}^2/R_{\oplus}}{\tau}}.
\end{align}
A representative $\tau$ has been taken from \citep{bukart}. Only $M_{sync} \sim
10^{-4}M_{\odot}$ need be heated for the thermodynamically interesting effects
studied in \citep{fullerIV} and \citep{tidal_novae}, it seems that strong shear
flows may exist even in the presence of magnetic coupling.

\subsection{Applicability to Astrophysical Systems}\label{ss:other}

Below, we speculate on the applicability of our results to general astrophysical
problems of interest:
\begin{enumerate}
    \item In astrophysical systems, $k_{\perp} \ll k_r$, while in our study,
        $k_x \simeq k_z$. True turbulence is expected to be isotropic at small
        scales, which may introduce different behavior for different ratios of
        $k_z/k_x$. However, $k_x \ll k_z$ is a numerically difficult regime; we
        defer its exploration to future work.

    \item In~\cite{barker_ogilvie}, inwards-propagating IGW are excited that
        break via geometric focusing. They find no reflected wave despite their
        nonlinear timescales being $10\times$ shorter than their viscous
        timescale, in contrast with the reflective behavior observed in our
        simulations. It is plausible the discrepancy arises due to the differing
        causes of wave breaking (geometric focusing vs.\ density
        stratification), but their simulations are also at significantly higher
        artificial viscosities than ours, as we show below.

        Associating $t_{L} \sim \nu k^2 \approx \nu k_z^2$ with the visocus
        timescale and $t_{NL} \sim \bm{u} \cdot \bm{\nabla} \sim \omega$ for the
        nonlinear timescale, we find their $\lambda = \frac{t_{NL}}{t_L} \sim
        \mathrm{Re}$ our Reynolds number. Our simulations indicate $\mathrm{Re}
        \gtrsim 500$ are required to observe the correct asymptotic behavior in
        terms of horizontal momentum flux reflection/transmission, so it is
        possible their lack of reflection is resolution limited.

    \item Our results give some tentative understanding of where tidal heating
        is expected to occur. While we use a barotropic equation of state and do
        not track heat deposition, we can recall that the strongest small-scale
        variations in fluid flow are localized within the critical layer. Thus,
        we posit that a substantial fraction of the energy carried by the IGW is
        deposited in the very narrow critical layer. Then, as the critical layer
        propagates downwards, the entire star is heated. This tidal heating
        model differs flom that used in \citep{tidal_novae}, and its effects
        will be studied in future work.
\end{enumerate}

\section{Acknowledgements}\label{s:ack}

This work has been supported in part by NASA FINESST grant
19-ASTRO19-0041.%chktex 8

\bibliographystyle{mnras}
\bibliography{Su_IGW_break}

\clearpage
\onecolumn
\appendix

\section{Forcing Solution}\label{s:force_solved}

To solve for the linear excited IGW amplitude due to bulk forcing
(see Eq.~\eqref{eq:vol_drive}), we consider the
complexified linearized system of equations, with all dynamical variables having
dependence $u_z(x, z, t) = \tilde{u_z}(z) e^{ik_xx - i\omega t}$. Thus,
$\pdil{}{t} \to -i\omega t$, $\pd{}{x} \to ik_x$, and the dynamical fluid
equations become (see Eqs.~\eqref{se:nl_orig} and~\eqref{se:nl_upsilon}):
\begin{align*}
    \rd{u_{z}}{z} + ik_xu_x &= 0,\\
    -i\omega u_x + ik_x \varpi + gHik_x \Upsilon &= 0,\\
    -i\omega u_{z} + \rd{\varpi}{z} + gH\rd{\Upsilon}{z}
        - \frac{\varpi}{H} &= 0,\\
    -i\omega \Upsilon - \frac{u_{z}}{H} &=
        C\exp\s{-\frac{(z - z_0)^2}{2\sigma^2}} \equiv \mathcal{C}(z).
\end{align*}
These can be recast solely in terms of $u_{z}$ as
\begin{align*}
     \rtd{u_{z}}{z} - k_x^2u_{z} - \frac{1}{H}\rd{u_{z}}{z}
        + u_{z}\frac{N^2k_x^2}{\omega^2} &=
    -\frac{gk_x^2}{\omega^2}\mathcal{C}(z).
        .\label{eq:narrow_inhomo}
\end{align*}
The homogeneous solutions are of form $u_{z,\pm}(z) = \exp\s{\p{\frac{1}{2H} \pm
ik_z}\p{z - z_0}}$ where $k_z$ satisfies the dispersion relation
(Eq.~\eqref{eq:disp_rel}). We compute the solution to the inhomogeneous ODE by
the method of variation of parameters. The Wronskian is
\begin{equation}
    W \equiv \det \begin{vmatrix}
        u_{z,+} & u_{z,-} \\[4pt]
        \rdil{u_{z,+}}{z} & \rdil{u_{z,-}}{z}
    \end{vmatrix} = -2ik_ze^{z/H}.
\end{equation}
The general solution is then
\begin{equation}
    u_z = -u_{z,+}\int \frac{1}{W} u_{z,-} \p{-\frac{gk_x^2}{\omega^2}
            \mathcal{C}(z)}\;\mathrm{d}z
        + u_{z,-}\int \frac{1}{W} u_{z,+} \p{-\frac{gk_x^2}{\omega^2}
            \mathcal{C}(z)}\;\mathrm{d}z.
\end{equation}
Taking these integrals and applying the boundary conditions $u_z\p{z \to \infty}
= u_{z,+}$, $u_z\p{z \to -\infty} = u_{z,-}$ give the exact solution:
\begin{align}
    u(z) &= \frac{\sqrt{\pi}C\sigma}{2^{3/2}ik_z}\frac{gk_x^2}{\omega^2}
            \exp\s{\frac{\p{\frac{\sigma^2}{2H} \pm ik_z\sigma^2}^2}{2\sigma^2}}
                \s{-u_{z,+}\p{1 + \erf\p{\xi_+}}
                    + u_{z,-} \p{\erf\p{\xi_-} - 1}},\\
    \xi_{\pm} &\equiv \s{\frac{z - z_0}{\sqrt{2\sigma^2}} +
        \frac{\sigma}{2^{3/2}H} \pm \frac{ik_z\sigma}{\sqrt{2}}}
\end{align}
Here, the error function is defined $\erf(z) \equiv
\p{2/\sqrt{\pi}}\int\limits_0^z \exp\p{-t^2}\;\mathrm{d}t$. If we are concerned
with only $z$ scales significantly larger than $\sigma$, then we may take
$\erf(\xi_{\pm}) \approx \Theta(z - z_0)$ (the Heaviside step function). If we
further assume $\abs{k_zH} \gg 1$ and restore the $e^{ik_xx - i\omega t}$
factor, we recover Eq.~\eqref{eq:uz_lin} in the main text
\begin{equation}
    u_{z}(x, z, t) = -\frac{C}{2ik_z}\frac{gk_x^2}{\omega^2}
        e^{-k_z^2\sigma^2 / 2}
        \sqrt{2\pi \sigma^2} e^{ik_xx - i\omega t} \times
    \begin{cases}
        \exp\s{\p{\frac{1}{2H} + ik_z}\p{z - z_0} + i\frac{k_z\sigma^2}{2H}}
            & \text{for }z > z_0,\\[5pt]
        \exp\s{\p{\frac{1}{2H} - ik_z}\p{z - z_0} - i\frac{k_z\sigma^2}{2H}}
            & \text{for }z > z_0.
    \end{cases}
\end{equation}
Note that in the main text, this approximate form is used to compute
$\bm{u}_{al}$, as it is easier to work with and sufficiently accurate in the
regions of interest (many $\sigma$ away from $z_0$).

\section{Equation Implementations}\label{se:strat_impl}

The system of equations we wish to simulate consists of
Eqs.~\eqref{eq:nl_incomp},~\eqref{eq:nl_upsilon_u}, and~\eqref{eq:vol_drive}.
% can be explicitly written in
% component form and partial derivatives below:
% \begin{subequations}\label{se:nl_var}
%     \begin{align}
%         \bm{\nabla} \cdot \bm{u} &= 0,\\
%         \pd{\Upsilon}{t} + \p{\bm{u} \cdot \bm{\nabla}} \Upsilon
%             - \frac{u_z}{H} &= 0,\\
%         \pd{u_{x}}{t} + \p{\bm{u} \cdot \bm{\nabla}}u_{x}
%             + \pd{\varpi'}{x} + gH\pd{\Upsilon}{x}
%             + \varpi' \pd{\Upsilon}{x} &= 0,\\
%         \pd{u_z}{t} + \p{\bm{u} \cdot \bm{\nabla}}u_z
%             + \pd{\varpi'}{z} + gH\pd{\Upsilon}{z}
%             + \varpi' \pd{\Upsilon}{z} - \frac{\varpi'}{H} &= 0.
%     \end{align}
% \end{subequations}
The nonlinear terms in the these equations will transfer energy from lower
wavenumbers to higher wavenumbers. Since spectral codes have no numerical
dissipation, artificial dissipation must be added. To ensure the dissipitive
system conserves horizontal momentum exactly, we begin by adding dissipitive
terms to the flux-conservative form of the Euler fluid equations
(equivalent to Eqs.~\ref{se:nl_orig}):
\begin{subequations}
    \begin{align}
        \bm{\nabla} \cdot \bm{u} &= 0,\\
        \partial_t \rho + \bm{\nabla} \cdot (\rho \bm{u} - \nu
            \bm{\nabla}(\rho - \overline{\rho})) &= 0,\label{eq:visc_cons_mom}\\
        \partial_t (\rho \bm{u}) + \bm{\nabla} \cdot (\rho \bm{u} \bm{u}
            + \mathrm{diag}(\rho \varpi)
            - \nu \rho \bm{\nabla}\bm{u})
            + \rho g \uv{z} &= 0.
    \end{align}
\end{subequations}
The same viscosity $\nu$ is used for both the diffusive and viscous terms,
although this is not required. Since the dissipation is not physical and is
purely used for numerical stability, we choose it such that hydrostatic
equilibrium is not modified (hence $\nu$ acts only on $\rho - \overline{\rho}$).

It is necessary to mask out nonlinear terms in the forcing zone using a form
similar to Eq.~\eqref{eq:Gamma}. In the absence of this mask, a nonphysical mean
flow localized to the forcing zone develops. We use mask
\begin{equation}
    \Gamma_{NL}(z) = \frac{1}{2}\s{2
        + \tanh \frac{z - (z_0 + 8\sigma)}{\sigma}
        - \tanh \frac{z - z_B}{\sigma}}.
\end{equation}

Including the damping zones and forcing terms as described in
Section~\ref{s:numerics}, and again making change of variables to $\Upsilon,
\varpi$, we finally obtain the full system of equations as simulated in Dedalus:
\begin{subequations}\label{se:dedalus_eqs}
    \begin{align}
        \bm{\nabla} \cdot \bm{u} ={}& 0,\\
        \partial_t \Upsilon - \frac{u_z}{H}
            ={}& -\Gamma(z) \Upsilon
                + \frac{F}{\overline{\rho}(z)}e^{-\frac{(z - z_0)^2}{2\sigma^2}}
                    \cos \p{k_xx - \omega t}\nonumber\\
            & + \Gamma_{NL} \bigg[-\p{\bm{u} \cdot \bm{\nabla}}\Upsilon
                + \nu\p{\nabla^2 \Upsilon + \p{
                    \bm{\nabla} \Upsilon} \cdot \p{\bm{\nabla}\Upsilon}
                    - \frac{2}{H}\partial_z \Upsilon
                    + \frac{1 - e^{-\Upsilon}}{H^2}}\bigg],\\
        \pd{u_x}{t} + \pd{\varpi'}{x} + gH\pd{\Upsilon}{x} ={}&
            -\Gamma(z) u_x
            + \Gamma_{NL}\bigg[\nu \nabla^2 u_x
            - u_x \nu\p{\nabla^2 \Upsilon + \p{\bm{\nabla} \Upsilon} \cdot
                \p{\bm{\nabla}\Upsilon} - \frac{2}{H}\partial_z \Upsilon
                + \frac{1 - e^{-\Upsilon}}{H^2}}\nonumber\\
            &+ 2\nu \p{\p{\p{\bm{\nabla}\Upsilon} \cdot \bm{\nabla}}u_x
                - \frac{1}{H}\partial_z u_x}
                - \p{\bm{u} \cdot \bm{\nabla}}u_x
                - \varpi' \pd{\Upsilon}{x}\bigg],\\
        \pd{u_z}{t} + \pd{\varpi'}{z} + gH\pd{\Upsilon}{z} - \frac{\varpi'}{H}
            ={}& -\Gamma(z) u_z
            +\Gamma_{NL}\bigg[\nu \nabla^2 u_z
            - u_z \nu\p{\nabla^2 \Upsilon + \p{\bm{\nabla} \Upsilon} \cdot
                \p{\bm{\nabla}\Upsilon} - \frac{2}{H}\partial_z \Upsilon
                + \frac{1 - e^{-\Upsilon}}{H^2}}\nonumber\\
            &+ 2\nu \p{\p{\p{\bm{\nabla}\Upsilon} \cdot \bm{\nabla}}u_z -
                \frac{1}{H}\partial_z u_{z}}
            - \p{\bm{u} \cdot \bm{\nabla}}u_z
            - \varpi' \pd{\Upsilon}{z}\bigg].
    \end{align}
\end{subequations}
\label{lastpage} % chktex 24
\end{document}
