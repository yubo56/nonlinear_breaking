    \documentclass[dvipsnames, 10pt]{beamer}
    \usetheme{Madrid}
    \usefonttheme{professionalfonts}
    \usepackage{
        amsmath,
        amssymb,
        fouriernc, % fourier font w/ new century book
        fancyhdr, % page styling
        lastpage, % footer fanciness
        hyperref, % various links
        setspace, % line spacing
        amsthm, % newtheorem and proof environment
        mathtools, % \Aboxed for boxing inside aligns, among others
        float, % Allow [H] figure env alignment
        enumerate, % Allow custom enumerate numbering
        graphicx, % allow includegraphics with more filetypes
        wasysym, % \smiley!
        upgreek, % \upmu for \mum macro
        listings, % writing TrueType fonts and including code prettily
        tikz, % drawing things
        booktabs, % \bottomrule instead of hline apparently
        cancel % can cancel things out!
    }
    \usepackage[
        labelfont=bf, % caption names are labeled in bold
        font=scriptsize % smaller font for captions
    ]{caption}
    \usepackage[font=scriptsize]{subcaption} % subfigures

    \newcommand*{\scinot}[2]{#1\times10^{#2}}
    \newcommand*{\dotp}[2]{\left<#1\,\middle|\,#2\right>}
    \newcommand*{\rd}[2]{\frac{\mathrm{d}#1}{\mathrm{d}#2}}
    \newcommand*{\pd}[2]{\frac{\partial#1}{\partial#2}}
    \newcommand*{\rtd}[2]{\frac{\mathrm{d}^2#1}{\mathrm{d}#2^2}}
    \newcommand*{\ptd}[2]{\frac{\partial^2 #1}{\partial#2^2}}
    \newcommand*{\md}[2]{\frac{\mathrm{D}#1}{\mathrm{D}#2}}
    \newcommand*{\pvec}[1]{\vec{#1}^{\,\prime}}
    \newcommand*{\svec}[1]{\vec{#1}\;\!}
    \newcommand*{\bm}[1]{\boldsymbol{\mathbf{#1}}}
    \newcommand*{\ang}[0]{\;\text{\AA}}
    \newcommand*{\mum}[0]{\\;upmu \mathrm{m}}
    \newcommand*{\at}[1]{\left.#1\right|}

    \let\Re\undefined
    \let\Im\undefined
    \DeclareMathOperator{\Res}{Res}
    \DeclareMathOperator{\Re}{Re}
    \DeclareMathOperator{\Im}{Im}
    \DeclareMathOperator{\Log}{Log}
    \DeclareMathOperator{\Arg}{Arg}
    \DeclareMathOperator{\Tr}{Tr}
    \DeclareMathOperator{\E}{E}
    \DeclareMathOperator{\Var}{Var}
    \DeclareMathOperator*{\argmin}{argmin}
    \DeclareMathOperator*{\argmax}{argmax}
    \DeclareMathOperator{\sgn}{sgn}
    \DeclareMathOperator{\diag}{diag\;}

    \DeclarePairedDelimiter\bra{\langle}{\rvert}
    \DeclarePairedDelimiter\ket{\lvert}{\rangle}
    \DeclarePairedDelimiter\abs{\lvert}{\rvert}
    \DeclarePairedDelimiter\ev{\langle}{\rangle}
    \DeclarePairedDelimiter\p{\lparen}{\rparen}
    \DeclarePairedDelimiter\s{\lbrack}{\rbrack}
    \DeclarePairedDelimiter\z{\lbrace}{\rbrace}

    % \everymath{\displaystyle} % biggify limits of inline sums and integrals
    \tikzstyle{circ} % usage: \node[circ, placement] (label) {text};
        = [draw, circle, fill=white, node distance=3cm, minimum height=2em]
    \definecolor{commentgreen}{rgb}{0,0.6,0}
    \lstset{
        basicstyle=\ttfamily\footnotesize,
        frame=single,
        numbers=left,
        showstringspaces=false,
        keywordstyle=\color{blue},
        stringstyle=\color{purple},
        commentstyle=\color{commentgreen},
        morecomment=[l][\color{magenta}]{\#}
    }

\begin{document}

\title[IGWs \& Mean Flow Steepening]{
Mean Flow Steepening in Internal Gravity Wave Breaking}
\subtitle{Group Meeting Presentation}
\author{Yubo Su}
\date{Nov 30, 2018}

\maketitle

\begin{frame}
    \frametitle{Background}
    \framesubtitle{Goldreich \& Nicholson 1989}

    \begin{itemize}
        \item \emph{Tidal Friction in Early-Type Stars}.

        \item Tidal torques excite outgoing internal gravity waves (IGW) at
            boundary between convective core and radiative envelope.

        \item IGW amplify as they propagate due to density rarefaction, break
            and deposit angular momentum.

        \item Transfers energy from the orbit to the star and synchronizing the
            spin to the orbit outside-in.
    \end{itemize}
    \begin{figure}[t]
        \centering
        \includegraphics[width=0.3\textwidth]{../Su_GM_092118/conv_core.jpg}
    \end{figure}
\end{frame}

\begin{frame}
    \frametitle{Background}
    \framesubtitle{Problem Setup}
    \begin{columns}
        \begin{column}{0.5\textwidth}
            {\small
            \begin{itemize}
                \item 2D incompressible, isothermal, stratified  plane parallel
                    atmosphere.

                \item For a barotropic $P(\rho, S) = P(\rho)$, incompressibility
                    $c_s^2 \to \infty$ produces
                    \begin{subequations}\label{se:fc_orig}
                        \begin{align}
                            \vec{\nabla} \cdot \vec{u} &= 0,\\
                            \pd{\rho}{t} + \vec{u} \cdot \vec{\nabla}\rho &= 0,\\
                            \pd{\vec{u}}{t} + \p*{\vec{u} \cdot \vec{\nabla}}\vec{u}
                                + \frac{\vec{\nabla}P}{\rho}
                                + g\hat{z} &= 0.
                        \end{align}
                    \end{subequations}
            \end{itemize}
            }
        \end{column}
        \begin{column}{0.5\textwidth}
            {\small
            \begin{itemize}
                \item Numerics: instead of $\frac{P}{\rho}$, $\rho T = P$ and
                    $\rho = \rho_0 e^\Upsilon$ for stratification $\rho(z)
                    \propto e^{-z/H}$.
                \begin{subequations}
                    \begin{align}
                        \vec{\nabla} \cdot \vec{u} &= 0,\\
                        \pd{\Upsilon}{t} - \frac{u_z}{H} &= 0,\\
                        \pd{\vec{u}}{t} + \p*{\vec{u} \cdot \vec{\nabla}}\vec{u}
                            + \vec{\nabla}T + gH \vec{\nabla}\Upsilon &\nonumber\\
                            + T_1\vec{\nabla}\Upsilon - \frac{T_1\hat{z}}{H} &= 0.
                    \end{align}
                \end{subequations}
            \end{itemize}
            }
        \end{column}
    \end{columns}
\end{frame}

\begin{frame}
    \frametitle{Background}
    \framesubtitle{Theory}

    \begin{itemize}
        \item In linear regime $\p*{\vec{u} \cdot \vec{\nabla}} \ll \partial_t$,
            $\vec{u} \propto e^{z/2H}$.

        \item Perturbation $\p*{\vec{u}, \rho_1, P_1}$ carries average
            horizontal momentum flux $\ev*{F_{p, x}}_x = \ev*{\rho u_x u_z}_x$.

        \item Induces mean flow
            \begin{equation}
                 \ev*{u_x}_x \equiv \bar{U}_x(z) \neq 0
                    = \frac{\ev*{u_xu_z}_x}{c_{g, z}}.
            \end{equation}

        \item Critical layer (equivalent to corotation resonance in other
            systems): where Doppler-shifted frequency (in fluid rest frame)
            $\omega \Rightarrow \omega - k_x\bar{U}_x = 0$.

        \item Note $\bar{U}_x \propto e^{z/H}$ so there is eventually $z_c:
            \omega - k_x\bar{U}_x(z_c) = 0$.
    \end{itemize}
\end{frame}

\begin{frame}
    \frametitle{Background}
    \framesubtitle{Hypothesis}


    \begin{itemize}
        \item Since critical layers almost always induce full absorption
            (recall, $\propto \exp\s*{-\pi \sqrt{\mathrm{Ri}^2 -
            \frac{1}{4}}}, \mathrm{Ri} = \frac{N}{\bar{U}_x'}$),
            hypothesis:

            \begin{itemize}
                \item $\vec{u}_1$ is excited, induces $\bar{U}_x$ mean flow.

                \item Where $\bar{U}_x$ satisfies critical layer criterion,
                    $F_{p, x}$ is fully absorbed.

                \item Horizontal momentum goes into spinning up more fluid up to
                    $\bar{U}_{x, crit} = \frac{\omega}{k_x}$.

                \item Thus, critical layer should propagate down.
            \end{itemize}

        \item Exactly the \emph{spin up outside in} tidal synchronization
            picture..
    \end{itemize}
\end{frame}

\begin{frame}
    \frametitle{Simulations}
    \framesubtitle{Large Domain, Low-Amplitude}

    \begin{itemize}
        \item Permit $z \in [0H, 10H]$, full $\rho_0 \propto e^{-z/H}$, allow
            $\vec{u} \propto e^{z/2H}$ to source growing mean flow.

        \item Low-Amplitude ($k_z\xi_z \ll 1$ everywhere). Orange = analytical
            solution.
    \end{itemize}
    \begin{figure}[t]
        \centering
        % linear0/p_041.png
        \includegraphics[width=0.55\textwidth]{lin_nonu.png}
        \caption{Low-Amplitude, nearly zero viscosity.}
    \end{figure}
\end{frame}

\begin{frame}
    \frametitle{Simulations}
    \framesubtitle{Large Domain, Low-Amplitude}
    \begin{figure}[t]
        \centering
        \hspace*{-19mm}%
        \begin{subfigure}{0.55\textwidth}
            \centering
            % linear_1/t_018.png
            \includegraphics[width=\textwidth]{lin_early.png}
            \caption{Early Low-A}
        \end{subfigure}
        \begin{subfigure}{0.55\textwidth}
            \centering
            % linear_1/t_049.png
            \includegraphics[width=\textwidth]{lin_late.png}
            \caption{Later Low-A}
        \end{subfigure}
        \hspace*{-19mm}%
    \end{figure}
\end{frame}

\begin{frame}
    \frametitle{Simulations}
    \framesubtitle{Large Domain, High-Amplitude}

    \begin{itemize}
        \item Note steepening region ($N = 1$, so $\pd{\bar{U}_x}{z} =
            \frac{N}{\mathrm{Ri}} = \mathrm{Ri}^{-1}$).
    \end{itemize}

    \begin{figure}[t]
        \centering
        \hspace*{-19mm}%
        \begin{subfigure}{0.53\textwidth}
            \centering
            % nl1_035/t_008.png
            \includegraphics[width=\textwidth]{nl_low_1.png}
            \caption{Lower-res.}
        \end{subfigure}
        \begin{subfigure}{0.53\textwidth}
            \centering
            % nl_full/t_029.png
            \includegraphics[width=\textwidth]{nl_full_1.png}
            \caption{Double $N_z$.}
        \end{subfigure}
        \hspace*{-19mm}%
    \end{figure}
\end{frame}

\begin{frame}
    \frametitle{Simulations}
    \framesubtitle{Large Domain, High-Amplitude}

    \begin{itemize}
        \item Later
    \end{itemize}

    \begin{figure}[t]
        \centering
        \hspace*{-19mm}%
        \begin{subfigure}{0.53\textwidth}
            \centering
            % nl1/t_049.png
            \includegraphics[width=\textwidth]{nl_low_2.png}
            \caption{Lower-res.}
        \end{subfigure}
        \begin{subfigure}{0.53\textwidth}
            \centering
            % nl_full/t_049.png
            \includegraphics[width=\textwidth]{nl_full_2.png}
            \caption{Double $N_z$.}
        \end{subfigure}
        \hspace*{-19mm}%
    \end{figure}
\end{frame}

\begin{frame}
    \frametitle{Simulations}
    \framesubtitle{Large Domain, Comments}

    \begin{itemize}
        \item While damping layers pump up $\bar{U}_x$, ran on higher amplitude,
            damping layer is not chief cause of $\bar{U}_x$ critical layer
            behavior. % nl2.png, nl2/t_018.png

        \item Would expect reflections due to $\mathrm{Ri} \gtrsim 1/2$,
            different from hypothesis!
            \begin{itemize}
                \item No reflections seen, could be \emph{viscously limited}?

                \item Indeed, $\mathrm{Ri}^{-1} \sim \abs*{\vec{k}}d$ where $d$
                    is the width of the spinup layer, then $\nu \rho_0
                    \frac{\bar{U}_{x, c}}{d^2} \sim \frac{F_{p, x}}{d}$.
            \end{itemize}

        \item Separately, matching $F_{p, x} = \ev*{\rho_0 u_x u_z}$ with
            spinning up mass $F_{p, x} = u_{front} \rho_0 \bar{U}_{x, crit}$
            lets us predict $u_{front}$, next page.
    \end{itemize}
\end{frame}

\begin{frame}
    \frametitle{Simulations}
    \framesubtitle{Large Domain, Predicting $u_{front}$}

    \begin{itemize}
        \item Front position is $z_f = \argmax_z \rd{\bar{U}_x}{z}$, while
            $F_{p, x} = 2F_{p, x}(z_f)$ ($\sim$ halfway in front).

        \item Predicts $u_{front} = \frac{F_{p, x}}{\rho_0(z) \bar{U}_{x, c}}
            \approx \scinot{2.2}{-3}NH$ (using $z \approx 5.5H$), or $2H/3$ in
            $300N$. Pretty close, seems to imply perfect absorption.
    \end{itemize}
    \begin{figure}[t]
        \centering
        \hspace*{-19mm}%
        \begin{subfigure}{0.53\textwidth}
            \centering
            % nl1_35/front.png
            \includegraphics[width=\textwidth]{front_nl.png}
            \caption{Front Position. Slight exponential.}
        \end{subfigure}
        \begin{subfigure}{0.53\textwidth}
            \centering
            % nl1_35/fluxes.png
            \includegraphics[width=\textwidth]{fluxes_nl.png}
            \caption{$F_{p, x} = \ev*{\rho u_x u_z}_x$.}
        \end{subfigure}
        \hspace*{-19mm}%
    \end{figure}
\end{frame}

\begin{frame}
    \frametitle{Simulations}
    \framesubtitle{Local Boussinesq}

    \begin{itemize}
        \item Go to Boussinesq, ``zoom in.'' Use $\mathfrak{D} = \nabla^6$
            regularization. Set up initial mean flow $\bar{U}_x(z)$ such that
            $\max_z \bar{U}_x(z) = \bar{U}_{x, c}$.

        \item Reflection indeed develops!
    \end{itemize}

    \begin{figure}[t]
        \centering
        \hspace*{-19mm}%
        \begin{subfigure}{0.55\textwidth}
            \centering
            % vstrat/t_000.png
            \includegraphics[width=\textwidth]{vstrat_0.png}
        \end{subfigure}
        \begin{subfigure}{0.55\textwidth}
            \centering
            % vstrat/t_010.png
            \includegraphics[width=\textwidth]{vstrat_1.png}
        \end{subfigure}
        \hspace*{-19mm}%
    \end{figure}
\end{frame}

\begin{frame}
    \frametitle{Simulations}
    \framesubtitle{Local Boussinesq}
    \begin{itemize}
        \item Expect Kelvin-Helmholtz instability to develop; resolution-limit?

        \item Not viscosity-limited: $\nu^{(6)} \rho_0
            \frac{\bar{U}_{x, c}}{d^6} \ll \frac{F_{p, x}}{d}$!
    \end{itemize}

    \begin{figure}[t]
        \centering
        \hspace*{-19mm}%
        \begin{subfigure}{0.55\textwidth}
            \centering
            % vstrat/t_060.png
            \includegraphics[width=\textwidth]{vstrat_2.png}
        \end{subfigure}
        \begin{subfigure}{0.55\textwidth}
            \centering
            % vstrat/t_105.png
            \includegraphics[width=\textwidth]{vstrat_3.png}
        \end{subfigure}
        \hspace*{-19mm}%
    \end{figure}
\end{frame}

\begin{frame}
    \frametitle{Simulations}
    \framesubtitle{Comparing $u_{front}$}

    \begin{itemize}
        \item Same as for anelastic, but predicts
            $u_{front} \approx \frac{\scinot{4.5}{-7}}{\rho_0 \bar{U}_{x, c}}
            \approx \scinot{1.4}{-5}HN$ or $0.014H$ in $1000N$. $t \in [1000,
            2000]$ accurate, $t \in [2000, 3000]$ less so.

        \item No front slowdown is expected since $\rho_0$ is constant in space,
            but is observed; could be explained by increasing reflectivity.
    \end{itemize}
    \begin{figure}[t]
        \centering
        \hspace*{-19mm}%
        \begin{subfigure}{0.53\textwidth}
            \centering
            \includegraphics[width=\textwidth]{front_vstrat.png}
            \caption{Front Position. Slowing down.}
        \end{subfigure}
        \begin{subfigure}{0.53\textwidth}
            \centering
            \includegraphics[width=\textwidth]{fluxes_vstrat.png}
            \caption{$F_{p, x}$.}
        \end{subfigure}
        \hspace*{-19mm}%
    \end{figure}
\end{frame}

\end{document}

