    \documentclass[dvipsnames]{beamer}
    \usetheme{Madrid}
    \usefonttheme{professionalfonts}
    \usepackage{
        amsmath,
        amssymb,
        fouriernc, % fourier font w/ new century book
        fancyhdr, % page styling
        lastpage, % footer fanciness
        hyperref, % various links
        setspace, % line spacing
        amsthm, % newtheorem and proof environment
        mathtools, % \Aboxed for boxing inside aligns, among others
        float, % Allow [H] figure env alignment
        enumerate, % Allow custom enumerate numbering
        graphicx, % allow includegraphics with more filetypes
        wasysym, % \smiley!
        upgreek, % \upmu for \mum macro
        listings, % writing TrueType fonts and including code prettily
        tikz, % drawing things
        booktabs, % \bottomrule instead of hline apparently
        cancel % can cancel things out!
    }
    \usepackage[
        labelfont=bf, % caption names are labeled in bold
        font=scriptsize % smaller font for captions
    ]{caption}
    \usepackage[font=scriptsize]{subcaption} % subfigures

    \newcommand*{\scinot}[2]{#1\times10^{#2}}
    \newcommand*{\dotp}[2]{\left<#1\,\middle|\,#2\right>}
    \newcommand*{\rd}[2]{\frac{\mathrm{d}#1}{\mathrm{d}#2}}
    \newcommand*{\pd}[2]{\frac{\partial#1}{\partial#2}}
    \newcommand*{\rtd}[2]{\frac{\mathrm{d}^2#1}{\mathrm{d}#2^2}}
    \newcommand*{\ptd}[2]{\frac{\partial^2 #1}{\partial#2^2}}
    \newcommand*{\md}[2]{\frac{\mathrm{D}#1}{\mathrm{D}#2}}
    \newcommand*{\pvec}[1]{\vec{#1}^{\,\prime}}
    \newcommand*{\svec}[1]{\vec{#1}\;\!}
    \newcommand*{\bm}[1]{\boldsymbol{\mathbf{#1}}}
    \newcommand*{\ang}[0]{\;\text{\AA}}
    \newcommand*{\mum}[0]{\\;upmu \mathrm{m}}
    \newcommand*{\at}[1]{\left.#1\right|}

    \let\Re\undefined
    \let\Im\undefined
    \DeclareMathOperator{\Res}{Res}
    \DeclareMathOperator{\Re}{Re}
    \DeclareMathOperator{\Im}{Im}
    \DeclareMathOperator{\Log}{Log}
    \DeclareMathOperator{\Arg}{Arg}
    \DeclareMathOperator{\Tr}{Tr}
    \DeclareMathOperator{\E}{E}
    \DeclareMathOperator{\Var}{Var}
    \DeclareMathOperator*{\argmin}{argmin}
    \DeclareMathOperator*{\argmax}{argmax}
    \DeclareMathOperator{\sgn}{sgn}
    \DeclareMathOperator{\diag}{diag\;}

    \DeclarePairedDelimiter\bra{\langle}{\rvert}
    \DeclarePairedDelimiter\ket{\lvert}{\rangle}
    \DeclarePairedDelimiter\abs{\lvert}{\rvert}
    \DeclarePairedDelimiter\ev{\langle}{\rangle}
    \DeclarePairedDelimiter\p{\lparen}{\rparen}
    \DeclarePairedDelimiter\s{\lbrack}{\rbrack}
    \DeclarePairedDelimiter\z{\lbrace}{\rbrace}

    % \everymath{\displaystyle} % biggify limits of inline sums and integrals
    \tikzstyle{circ} % usage: \node[circ, placement] (label) {text};
        = [draw, circle, fill=white, node distance=3cm, minimum height=2em]
    \definecolor{commentgreen}{rgb}{0,0.6,0}
    \lstset{
        basicstyle=\ttfamily\footnotesize,
        frame=single,
        numbers=left,
        showstringspaces=false,
        keywordstyle=\color{blue},
        stringstyle=\color{purple},
        commentstyle=\color{commentgreen},
        morecomment=[l][\color{magenta}]{\#}
    }

\begin{document}

\title{Quick Research Review}
\author{Yubo Su}
\date{Nov. 06, 2018}

\maketitle

\begin{frame}
    \frametitle{Objectives}

    \begin{itemize}
        \item Continuous train of IGW excited in stratified WD atmosphere by
            binary companion, grows due to stratification, reaches nonlinear
            amplitudes $\xi_z k_z \gtrsim 1$ before the peak of the envelope.

        \item Seek long-term behavior of nonlinear dissipation, whether
            steady-state or periodic.
    \end{itemize}
\end{frame}

\begin{frame}
    \frametitle{Numerical Simulations, No Shear}
    \framesubtitle{Setup}

    \begin{itemize}
        \item 2D, Fourier $\times$ Chebyshev, damping zones, ran at both $128
            \times 512$ and $256 \times 1024$ over $3H \times 12H$ grid, where
            $H$ is scale height.

        \item Target waves $k_x = \frac{2\pi}{3H}, k_z = \frac{2\pi}{H}, \omega
            \approx 0.32N$, $N$ B-V frequency.

        \item Wave is excited by volumetric forcing
            $e^{-z^2/2\sigma^2}\cos\p*{k_xx - \omega t}$ with $\sigma =
            \frac{3H}{32} \lesssim \frac{1}{k_z}$.

        \item Regularized by numerical viscosity $\nu \gtrsim
            \frac{\omega}{k_{\max}k_z}$.
    \end{itemize}
\end{frame}

\begin{frame}
    \frametitle{Numerical Simulations, No Shear}
    \framesubtitle{Results}

    \begin{itemize}
        \item High resolution ($\nu = 4\frac{\omega}{k_{\max}k_z}$):
            \lstinline{nl3.mp4}. Note that mean flow starts to build up at the
            outgoing damping zone, and once it hits criticality it starts to
            move down while continuing to steepen. Eventually, it stops moving
            and produces reflection below, while leaving behind a sinusoidal
            mean flow above.

        \item Low resolution ($\nu = 3\frac{\omega}{k_{\max}k_z}$):
            \lstinline{nl3_lowres.mp4}. Qualitatively similar, but the mean flow
            forms later.

        \item Tried masking nonlinear terms near damping zones,
            \lstinline{nl3_lowres_test.mp4}. Still mean flow interaction!
    \end{itemize}
\end{frame}

\begin{frame}
    \frametitle{Numerical Simulations, No Shear}
    \framesubtitle{Hypotheses}
    \begin{itemize}
        \item Damping layer is causing mean flow to begin building up
            like Daniel said; is removing nonlinear terms near damping layer a
            good idea?

        \item Further growth/propagation is critical layer absorption?
            Eventual reflection might build up when shear flow is too
            steep $\mathrm{Ri} \gtrsim 1/2$ or WKB breaks down
            $\pd{k_z}{z} \sim \frac{k_z}{\lambda_z}$.

            The sign of the flow is
            wrong though, $\sim -\frac{1}{c_{gz}}\ev*{u_xu_z}_x$!
    \end{itemize}
\end{frame}

\begin{frame}
    \frametitle{Numerical Simulations, Shear Flow}
    \framesubtitle{Setup}

    \begin{itemize}
        \item 2D, Fourier $\times$ Fourier, damping zones, running at $64 \times
            256$. No height stratification, domain is $ H \times H$ (though $H$
            is now physically meaningless).

        \item Target waves $k_x = \frac{2\pi}{H}, k_z = \frac{20\pi}{H}, \omega
            \approx 0.1N$, $N$ B-V frequency. Volumetric forcing.

        \item Regularized by $\nabla^6$ hyperviscosity.
    \end{itemize}
\end{frame}

\begin{frame}
    \frametitle{Numerical Simulations, Shear Flow}
    \framesubtitle{Results}

    \begin{itemize}
        \item Initialize w/ shear flow and let evolve. Try both narrow and wide
            profiles (\lstinline{vstrat*.mp4}).

        \item Conclusion: attenuated transmission when thin, barely-critical
            layer, smooth WKB-like wavelength shortening when broad, critical
            flow, but steep $\pd{U_0}{z}$ gives reflection!

        \item Recall Booker \& Bretherton result
            \begin{align}
                T &\sim \exp\s*{-\pi \sqrt{\frac{N^2}{(U_0')^2} - \frac{1}{4}}}
                & R &\sim
                    \exp\s*{-2\pi \sqrt{\frac{N^2}{(U_0')^2} - \frac{1}{4}}}.
            \end{align}
            WKB criterion $\Rightarrow \frac{k_z^2}{N}U_0' \frac{\lambda^2}{2}
            \ll 1$.

        \item Winters d'Asaro had a local $\mathrm{Ri} \sim 0.5$.
    \end{itemize}
\end{frame}

\end{document}

