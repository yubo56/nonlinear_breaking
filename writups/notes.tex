    \documentclass[11pt,
        usenames, % allows access to some tikz colors
        dvipsnames % more colors: https://en.wikibooks.org/wiki/LaTeX/Colors
    ]{report}
    \usepackage{
        amsmath,
        amssymb,
        fouriernc, % fourier font w/ new century book
        fancyhdr, % page styling
        lastpage, % footer fanciness
        hyperref, % various links
        setspace, % line spacing
        amsthm, % newtheorem and proof environment
        mathtools, % \Aboxed for boxing inside aligns, among others
        float, % Allow [H] figure env alignment
        enumerate, % Allow custom enumerate numbering
        graphicx, % allow includegraphics with more filetypes
        wasysym, % \smiley!
        upgreek, % \upmu for \mum macro
        listings, % writing TrueType fonts and including code prettily
        tikz, % drawing things
        booktabs, % \bottomrule instead of hline apparently
        cancel % can cancel things out!
    }
    \usepackage[margin=1in]{geometry} % page geometry
    \usepackage[
        labelfont=bf, % caption names are labeled in bold
        font=scriptsize % smaller font for captions
    ]{caption}
    \usepackage[font=scriptsize]{subcaption} % subfigures

    \newcommand*{\scinot}[2]{#1\times10^{#2}}
    \newcommand*{\bra}[1]{\left<#1\right|}
    \newcommand*{\ket}[1]{\left|#1\right>}
    \newcommand*{\dotp}[2]{\left<#1\,\middle|\,#2\right>}
    \newcommand*{\rd}[2]{\frac{\mathrm{d}#1}{\mathrm{d}#2}}
    \newcommand*{\pd}[2]{\frac{\partial#1}{\partial#2}}
    \newcommand*{\rtd}[2]{\frac{\mathrm{d}^2#1}{\mathrm{d}#2^2}}
    \newcommand*{\ptd}[2]{\frac{\partial^2 #1}{\partial#2^2}}
    \newcommand*{\md}[2]{\frac{\mathrm{D}#1}{\mathrm{D}#2}}
    \newcommand*{\norm}[1]{\left|\left|#1\right|\right|}
    \newcommand*{\abs}[1]{\left|#1\right|}
    \newcommand*{\pvec}[1]{\vec{#1}^{\,\prime}}
    \newcommand*{\svec}[1]{\vec{#1}\;\!}
    \newcommand*{\bm}[1]{\boldsymbol{\mathbf{#1}}}
    \newcommand*{\expvalue}[1]{\left<#1\right>}
    \newcommand*{\ang}[0]{\text{\AA}}
    \newcommand*{\mum}[0]{\upmu \mathrm{m}}
    \newcommand*{\at}[1]{\left.#1\right|}

    \newtheorem{theorem}{Theorem}[section]

    \let\Re\undefined
    \let\Im\undefined
    \DeclareMathOperator{\Res}{Res}
    \DeclareMathOperator{\Re}{Re}
    \DeclareMathOperator{\Im}{Im}
    \DeclareMathOperator{\Log}{Log}
    \DeclareMathOperator{\Arg}{Arg}
    \DeclareMathOperator{\Tr}{Tr}
    \DeclareMathOperator{\E}{E}
    \DeclareMathOperator{\Var}{Var}
    \DeclareMathOperator*{\argmin}{argmin}
    \DeclareMathOperator*{\argmax}{argmax}
    \DeclareMathOperator{\sgn}{sgn}
    \DeclareMathOperator{\diag}{diag\;}

    \DeclarePairedDelimiter\p{\lparen}{\rparen}
    \DeclarePairedDelimiter\s{\lbrack}{\rbrack}
    \DeclarePairedDelimiter\z{\lbrace}{\rbrace}

    % \everymath{\displaystyle} % biggify limits of inline sums and integrals
    \tikzstyle{circ} % usage: \node[circ, placement] (label) {text};
        = [draw, circle, fill=white, node distance=3cm, minimum height=2em]
    \definecolor{commentgreen}{rgb}{0,0.6,0}
    \lstset{
        basicstyle=\ttfamily\footnotesize,
        frame=single,
        numbers=left,
        showstringspaces=false,
        keywordstyle=\color{blue},
        stringstyle=\color{purple},
        commentstyle=\color{commentgreen},
        morecomment=[l][\color{magenta}]{\#}
    }

\begin{document}

\def\Snospace~{\S{}} % hack to remove the space left after autorefs
\renewcommand*{\sectionautorefname}{\Snospace}
\renewcommand*{\appendixautorefname}{\Snospace}
\renewcommand*{\figureautorefname}{Fig.}
\renewcommand*{\equationautorefname}{Eq.}
\renewcommand*{\tableautorefname}{Tab.}

\onehalfspacing

\pagestyle{fancy}
\rfoot{Yubo Su}
\cfoot{\thepage/\pageref{LastPage}}

\title{Research Notes}
\author{Yubo Su}

\maketitle

\tableofcontents

\newpage

\chapter{2D Wave Breaking in Atmospheres}

The goal of this will be to lay out a formalism that can reproduce Sutherland et
al.~2011\footnote{DOI:10.1175/JAS-D-11--097.1} and also investigate driven
oscillations versus the breaking of a single wave packet. This represents wave
breaking in the atmosphere.

\section{Dynamical Setup}

We adopt notation where $q_0$ is the background quantity and $q_1$ is the
perturbed quantity from the propagating wave.

\subsection{Nonturbulent Equations of Motion}

The equations of motion are the usual fluid equations
\begin{subequations}\label{se:fl_eq}
    \begin{align}
        \pd{\rho}{t} + \vec{\nabla} \cdot \rho \vec{u} &= 0,
            \label{eq:fl_eq.cont}\\
        \rd{\vec{u}}{t} &= -\frac{\vec{\nabla}P}{\rho} - g\hat{z},
            \label{eq:fl_eq.mom}
    \end{align}
\end{subequations}
where $\rd{}{t} = \pd{}{t} + \vec{u} \cdot \vec{\nabla}$ denotes the Lagrangian
derivative. We take $\vec{u}_0 = 0$ no background flow, $P_0(x, z) = P_0(z)
\propto e^{-z/H}$ where $H$ is the scale height of the atmosphere. We take the
fluid to be incompressible $\vec{\nabla} \cdot \vec{u} = 0$, which turns the
continuity equation \autoref{eq:fl_eq.cont} into $\rd{\rho}{t} = 0$. Finally, we
take the atmosphere to be isothermal, implying $\rho_0 \propto P_0 \propto
e^{-z/H}$, and more importantly $\frac{P_0}{\rho_0} = \frac{P_1}{\rho_1} =
\frac{k_BT}{\mu m_p}$. We denote $P_0 = P_0(z=0), \rho_0 = \rho_0(z=0)$.

We write out the new fluid equations under these assumptions; note that we do
not assume perturbed quantities are small and must keep all terms
\begin{subequations}\label{se:lin_eq}
    \begin{align}
        \pd{\rho_1}{t} &= \frac{u_{1z}}{H}\rho_0e^{-z/H} - \vec{u}_1 \cdot
            \vec{\nabla} \rho_1, \label{eq:lin_eq.cont}\\
        \pd{\vec{u}_1}{t} &=
            -\p*{\vec{u}_1 \cdot \vec{\nabla}}\vec{u}_1
            -\frac{-\frac{\hat{z}}{H}P_0e^{-z/H} + \frac{k_BT}{\mu m_p}
                \vec{\nabla} \rho_1}{\rho_0e^{-z/H} + \rho_1}
                - g\hat{z}, \label{eq:lin_eq.mom}.
    \end{align}
\end{subequations}
This consists of four equations in four unknowns, $\rho_1$ and the three
components of $\vec{u}_1$.

\subsection{Turbulent Dissipation}

In order to mimic the effect of turbulent dissipation, we introduce a viscosity
term into the equations of motion. Per the incompressible Navier-Stokes
equation, we quantify viscosity via $\nu \nabla^2 \vec{u}_1$ where $\nu$ is the
turbulent viscosity. For this problem, we simply use $\nu \neq
0.1\;\mathrm{cm^2/s}$ for air.

We identify the kinetic energy density carried by the perturbation to be
$\mathcal{U}_T = \frac{1}{2}\rho \p*{\vec{u}_1 \cdot \vec{u}_1}$. We require
$\rd{\mathcal{U}}{t} = 0$ under the dissipation-free fluid equations
\autoref{se:fl_eq} where we denote $\mathcal{U} \equiv \mathcal{U}_T +
\mathcal{U}_V$ the sum of the kinetic and potential energy densities; we assume
incompressibility as per our present problem:
\begin{align}
    \rd{\p*{\mathcal{U}_T + \mathcal{U}_V}}{t} = 0
        &= \frac{1}{2}u_1^2 \cancelto{0}{\rd{\rho}{t}}
        + \rho \vec{u}_1 \cdot \rd{\vec{u}_1}{t} + \rd{\mathcal{U}_V}{t}
        ,\nonumber\\
        &= \rho \p*{
                -\frac{\vec{u}_1 \cdot \vec{\nabla}P}{\rho} - u_{1z}g
            } + \rd{\mathcal{U}_V}{t},\nonumber\\
        &= -\cancelto{0}{\rd{P}{t}} + \cancelto{0}{\pd{P_0}{t}}
            + \pd{P_1}{t} - \rho u_{1z}g + \rd{\mathcal{U}_V}{t}.
\end{align}

%TODO figure this out from L&L, figure out the energy dissipation upon adding
%Navier-Stokes term.

\subsection{Boundary Conditions}

We must bound this to a finite domain. We choose periodic boundary conditions in
$x$ with total length $L_x$, $x \in \s*{-\frac{L}{2}, \frac{L}{2}}$. We choose
$z$ dimension to have total length $L_z$, $z \in [0, L_z]$.

To set up the boundary condition at $z = 0$, we recall that gravity waves in an
atmosphere with $e^{-z/H}$ profile have form
\begin{equation}
    u_{1z} \propto e^{\frac{z}{2H}}e^{i\p*{k_xx + k_zz - \omega t}},
        \label{eq:bc_below}
\end{equation}
where ($N^2 \equiv \frac{g}{H}$ is the Brunt-V\"ais\"al\"a frequency in an
incompressible, stratified atmosphere)
\begin{equation}
    \omega^2 = \frac{N^2 k_x^2}{k_x^2 + k_z^2 + \frac{1}{4H^2}}.
        \label{eq:omega}
\end{equation}
Thus, at constant $z = 0$ we must have
\begin{equation}
    \vec{u}_{1z}(z = 0) = e^{i(k_xx - \omega t)}\label{eq:bc_u1z}.
\end{equation}

The boundary condition at $z = L_z$ is much harder to determine. It is clear
that $\vec{u}_1(z = \infty) = \rho_1(z = \infty) = 0$, and for $L_z \gg H$ this
would be a reasonable approximation, if simply because we expect the majority of
the wave to dissipate via turbulent dissipation as $z$ reaches many $H$. We will
simply choose the BC to be many multiples of $H$. We can solve with both a
Dirichlet and Neumann BC and compare the two solutions; if the solutions
differ significantly then we must choose a larger $L_z$. These are the only two
solutions that can be implemented where we do not need the phase of the linear
wave, which we lose during the nonlinear breaking region, to relate the function
and derivative at the boundary.

For the boundary conditions on $\rho$, we note that in the linear regime it
should just have a phase offset from $u_{1z}$, thus we choose $\rho(z = 0)
\propto e^{\frac{z}{2H}}ie^{i\p*{k_xx + k_zz - \omega t}}$ and a similar
treatment at $z = L_z$, taking both a Dirichlet and Neumann BC\@. This should be
sufficient information to constrain our problem, as satisfying incompressibility
along the boundaries constrains $u_{1x}$.

\subsubsection{Old Stuff---$u_{1z}(z = L_z)$}

\textbf{Note:} The reason the below is difficult to make work is that it tries
to restore a linear treatment to the wave. However, since we must capture
nonlinear dynamics, we lose the concept of the ``phase'' of the wave in the wave
breaking region, and it proves difficult to write down a BC that is
phase-independent.

However, at any finite boundary there is nonzero reflection thanks to the
continually varying impedance. \emph{The correct BC to use when performing a
finite truncation must match propagation characteristics with an untruncated
medium.}

We assume the perturbations have once again become small at $L_z$, i.e.\
dissipation between $[0, L_z]$ is sufficient that only a small amplitude
perturbation persists to $z = L_z$. This is a necessary criterion for choosing
$L_z$ else we do not have a representative portrait of the energy dissipation
profile of the propagating wave with our simulation. Then we recall that in the
linear regime, the solution given by \autoref{eq:bc_below} is the propagating
solution. We assume that $k_x, \omega$ are the same as the values at the $z = 0$
boundary, then
\begin{equation}
    \frac{\rho_{,z}}{\rho} = \frac{u_{1z, z}}{u_{1z}} = \frac{1}{2H} + ik_z.
        \label{eq:bc_above}
\end{equation}
Recall that since our equations of motion are nonlinear, we cannot use complex
notation except in very limited, time-averaged cases. Thus, how to implement
this BC is tricky\dots

\subsection{Simulation}

We begin our simulation with $\rho_1 = 0, \vec{u}_1 = 0$ strictly within the
domain of simulation. We will borrow some values from Sutherland's paper and use
$k_z = 2\;\mathrm{km^{-1}}$ then define $k_x = -0.4k_z, H = 10 / k_z, A = 0.05 /
k_z, L_z = 300 / k_z, L_x = 20 / k_z$, We also use $\mu \approx 29, T =
273\;\mathrm{K}, \rho_0 = 1\;\mathrm{kg/m^3}, P_0 = \frac{\rho_0 k_BT}{\mu m_p},
g = 10\;\mathrm{m/s^2}$.

\end{document}

