    \documentclass[11pt,
        usenames, % allows access to some tikz colors
        dvipsnames % more colors: https://en.wikibooks.org/wiki/LaTeX/Colors
    ]{report}
    \usepackage{
        amsmath,
        amssymb,
        fouriernc, % fourier font w/ new century book
        fancyhdr, % page styling
        lastpage, % footer fanciness
        hyperref, % various links
        setspace, % line spacing
        amsthm, % newtheorem and proof environment
        mathtools, % \Aboxed for boxing inside aligns, among others
        float, % Allow [H] figure env alignment
        enumerate, % Allow custom enumerate numbering
        graphicx, % allow includegraphics with more filetypes
        wasysym, % \smiley!
        upgreek, % \upmu for \mum macro
        listings, % writing TrueType fonts and including code prettily
        tikz, % drawing things
        booktabs, % \bottomrule instead of hline apparently
        cancel % can cancel things out!
    }
    \usepackage[margin=1in]{geometry} % page geometry
    \usepackage[
        labelfont=bf, % caption names are labeled in bold
        font=scriptsize % smaller font for captions
    ]{caption}
    \usepackage[font=scriptsize]{subcaption} % subfigures

    \newcommand*{\scinot}[2]{#1\times10^{#2}}
    \newcommand*{\bra}[1]{\left<#1\right|}
    \newcommand*{\ket}[1]{\left|#1\right>}
    \newcommand*{\dotp}[2]{\left<#1\,\middle|\,#2\right>}
    \newcommand*{\rd}[2]{\frac{\mathrm{d}#1}{\mathrm{d}#2}}
    \newcommand*{\pd}[2]{\frac{\partial#1}{\partial#2}}
    \newcommand*{\rtd}[2]{\frac{\mathrm{d}^2#1}{\mathrm{d}#2^2}}
    \newcommand*{\ptd}[2]{\frac{\partial^2 #1}{\partial#2^2}}
    \newcommand*{\md}[2]{\frac{\mathrm{D}#1}{\mathrm{D}#2}}
    \newcommand*{\norm}[1]{\left|\left|#1\right|\right|}
    \newcommand*{\abs}[1]{\left|#1\right|}
    \newcommand*{\pvec}[1]{\vec{#1}^{\,\prime}}
    \newcommand*{\svec}[1]{\vec{#1}\;\!}
    \newcommand*{\bm}[1]{\boldsymbol{\mathbf{#1}}}
    \newcommand*{\expvalue}[1]{\left<#1\right>}
    \newcommand*{\ang}[0]{\text{\AA}}
    \newcommand*{\mum}[0]{\upmu \mathrm{m}}
    \newcommand*{\at}[1]{\left.#1\right|}

    \newtheorem{theorem}{Theorem}[section]

    \let\Re\undefined
    \let\Im\undefined
    \DeclareMathOperator{\Res}{Res}
    \DeclareMathOperator{\Re}{Re}
    \DeclareMathOperator{\Im}{Im}
    \DeclareMathOperator{\Log}{Log}
    \DeclareMathOperator{\Arg}{Arg}
    \DeclareMathOperator{\Tr}{Tr}
    \DeclareMathOperator{\E}{E}
    \DeclareMathOperator{\Var}{Var}
    \DeclareMathOperator*{\argmin}{argmin}
    \DeclareMathOperator*{\argmax}{argmax}
    \DeclareMathOperator{\sgn}{sgn}
    \DeclareMathOperator{\diag}{diag\;}

    \DeclarePairedDelimiter\p{\lparen}{\rparen}
    \DeclarePairedDelimiter\s{\lbrack}{\rbrack}
    \DeclarePairedDelimiter\z{\lbrace}{\rbrace}

    % \everymath{\displaystyle} % biggify limits of inline sums and integrals
    \tikzstyle{circ} % usage: \node[circ, placement] (label) {text};
        = [draw, circle, fill=white, node distance=3cm, minimum height=2em]
    \definecolor{commentgreen}{rgb}{0,0.6,0}
    \lstset{
        basicstyle=\ttfamily\footnotesize,
        frame=single,
        numbers=left,
        showstringspaces=false,
        keywordstyle=\color{blue},
        stringstyle=\color{purple},
        commentstyle=\color{commentgreen},
        morecomment=[l][\color{magenta}]{\#}
    }

\begin{document}

\def\Snospace~{\S{}} % hack to remove the space left after autorefs
\renewcommand*{\sectionautorefname}{\Snospace}
\renewcommand*{\appendixautorefname}{\Snospace}
\renewcommand*{\figureautorefname}{Fig.}
\renewcommand*{\equationautorefname}{Eq.}
\renewcommand*{\tableautorefname}{Tab.}

\onehalfspacing

\pagestyle{fancy}
\rfoot{Yubo Su}
\cfoot{\thepage/\pageref{LastPage}}

\title{Research Notes}
\author{Yubo Su}
\rhead{}

\maketitle

\tableofcontents

\newpage

\chapter{Preliminary Problems}

To get an intuition for how Dedalus and fluid mechanics works, we will solve
some toy problems. Recall fluid equations in the presence of a uniform
gravitational field $\vec{g} = -g\hat{z}$:
\begin{equation}
    \begin{split}
        \rd{\rho}{t} + \vec{\nabla} \cdot \vec{u} &= 0,\\
        \rd{\vec{u}}{t} + \frac{\vec{\nabla}P}{\rho} - \vec{g} &= 0.
    \end{split}\label{eq:fluid_eq}
\end{equation}

In the incompressible limit, $\rd{\rho}{t} = 0$, which implies $\vec{\nabla}
\cdot \vec{u} = 0$. We use subscripts to indicate perturbed quantities, $Q_0$ is
background and $Q_1$ is perturbed. We will generally use $\vec{u}_0 = 0$ unless
otherwise noted. We will also generally assume symmetry along all axes except
$z$ the vertical axis.

In the incompressible limit, the fluid equations become
\begin{equation}
    \begin{split}
        \vec{\nabla} \cdot \vec{u}_1 &= 0,\\
        \pd{\rho_1}{t} + u_{1z}\pd{\rho_0}{z} &= 0,\\
        \pd{\vec{u}_1}{t} + \frac{1}{\rho_0}\vec{\nabla}P_1
            + \frac{\rho_1 g \hat{z}}{\rho_0} &= 0.
    \end{split}\label{eq:lin.incomp}
\end{equation}
We have used $\vec{\nabla}P_0 = -\rho_0 g \hat{z}$ in the absence of
perturbations.

\section{Incompressible, No Gravity}

We note that in the no gravity limit that $\rho_1$ does not have an effect on
other dynamical variables, so the equations of motion we must solve are
\begin{equation}
    \begin{split}
        \vec{\nabla} \cdot \vec{u}_1 &= 0,\\
        \pd{\vec{u}_1}{t} + \frac{\vec{\nabla}P_1}{\rho_0} &= 0.
    \end{split}\label{eq:lin.no_g_eom}
\end{equation}
We can take the divergence of the momentum equation and substitute the
continuity equation to get $\nabla^2 P = 0$.

\subsection{Dirichlet BCs}

This is a Laplace equation, which we've solved countless times. Imposing
periodic boundary conditions in the $x$ direction and $P_1(z = L) = 0, P_1(z =
0) = \mathcal{P}(x, t)$, we obtain eigenfunctions
\begin{equation}
    \begin{split}
        P_{1, n}(x, z, t) &=
            \frac{\mathcal{P}_n(t)}{\sinh(k_nL)}e^{ik_nx}\sinh\p*{k_n(L - z)},\\
        u_{1x, n}(x, z, t) &=
            \int\limits^t -\frac{1}{\rho_0}\pd{P_{1, n}}{x}\;\mathrm{d}t,\\
        u_{1z, n}(x, z, t) &=
            \int\limits^t -\frac{1}{\rho_0}\pd{P_{1, n}}{z}\;\mathrm{d}t.
    \end{split}\label{eq:incomp_nog}
\end{equation}
We define $k_n = \frac{2\pi n}{L}, n \geq 0$ and $\mathcal{P}(x, t) =
\sum\limits_n \mathcal{P}_n(t)e^{ik_nx}$.

Thus, if we impose BCs $\mathcal{P}(x, t) = \sin \frac{2\pi x}{L}$ and start
with initial conditions such that all quantities are zero, we would expect after
transients die out that
\begin{equation}
    \begin{split}
        P(x, z, t) &= \frac{\sin \frac{2\pi x}{L}}{\sinh 2\pi}
            \sinh \p*{2\pi \frac{L - z}{L}},\\
        u_{1x}(x, z, t) &= -\frac{2\pi t}{L\rho_0}
            \frac{\cos \frac{2\pi x}{L}}{\sinh 2\pi}\sinh \p*{2\pi \frac{L -
                z}{L}},\\
        u_{1z}(x, z, t) &= +\frac{2\pi t}{L\rho_0}
            \frac{\sin \frac{2\pi z}{L}}{\sinh 2\pi}\cosh \p*{2\pi \frac{L -
                z}{L}}.
    \end{split}\label{eq:incomp_nog_sol}
\end{equation}
This is in good agreement with the results, presented in \autoref{fig:no_g}.
Note that $P$ is constant while $\vec{u}$ increases linearly in time, and we
observe the expected $\sim \sin x \sinh \frac{L - z}{z}$ dependence. In fact,
$u_{1x}, u_{1z}$ are exactly $\frac{2\pi}{10}$ at $t = 1$.

\begin{figure}[!h]
    \centering
    \begin{subfigure}{0.3\textwidth}
        \centering
        \includegraphics[width=\textwidth]{../sims/prelim/no_g/no_g_P_t50.png}
    \end{subfigure}
    \begin{subfigure}{0.3\textwidth}
        \centering
        \includegraphics[width=\textwidth]{../sims/prelim/no_g/no_g_ux_t50.png}
    \end{subfigure}
    \begin{subfigure}{0.3\textwidth}
        \centering
        \includegraphics[width=\textwidth]{../sims/prelim/no_g/no_g_uz_t50.png}
    \end{subfigure}

    \begin{subfigure}{0.3\textwidth}
        \centering
        \includegraphics[width=\textwidth]{../sims/prelim/no_g/no_g_P_t100.png}
    \end{subfigure}
    \begin{subfigure}{0.3\textwidth}
        \centering
        \includegraphics[width=\textwidth]{../sims/prelim/no_g/no_g_ux_t100.png}
    \end{subfigure}
    \begin{subfigure}{0.3\textwidth}
        \centering
        \includegraphics[width=\textwidth]{../sims/prelim/no_g/no_g_uz_t100.png}
    \end{subfigure}
    \caption{$P, u_x, u_z$ at $t = 0.5$ and $t = 1$ for $\rho_0 = 1$. We choose
    a $L = 10$ square domain.}\label{fig:no_g}
\end{figure}

It is worth noting that, since our \autoref{eq:lin.no_g_eom} reduced to a Laplace
equation, we needed two $z$ BCs and two $x$ BCs (periodic BCs amount to equating
the value and derivative of the function). This is in agreement with the
observation that the original \autoref{eq:lin.no_g_eom} had two derivatives in
$x, z$ apiece, so we needed two BCs each.

\subsection{Sommerfield/Radiative + Driving BCs}

This is the more interesting case. Let's go back to the Laplace equation
$\nabla^2 P = 0$ but instead implement a driving term at $z = 0$ and radiative
BCs at $z = L$. This

\section{Incompressible, Stratified w/ Gravity}

\subsection{Eigenfunctions}

Let's restore the $\rho_1 g$ term now. For funsies, we begin by solving for
arbitrary stratification $\rho_0(z)$ first. The fluid equations to first order
reduce to
\begin{equation}
    \begin{split}
        \pd{\rho_1}{t} + \p*{\vec{u}_1 \cdot \vec{\nabla}}\rho_0 &= 0,\\
        \vec{\nabla} \cdot \vec{u}_1 &= 0,\\
        \pd{\vec{u}_1}{t} &= -\frac{\vec{\nabla}P_1}{\rho_0}
            - \frac{\rho_1 g}{\rho_0}
    \end{split}\label{eq:lin_incomp}
\end{equation}
We expect there to be some $z$ dependence in the amplitude, so we substitute
variables of form $e^{i(kx - \omega t)}$ and do not specify the $z$ dependence.
This gives us
\begin{equation}
    \begin{split}
        -i\omega \rho_1 - u_{1z}\pd{\rho_0}{z} &= 0,\\
        iku_{1x} + \pd{u_{1z}}{z} &= 0,\\
        -iw u_{1x} + \frac{ik_x P_1}{\rho_0} &= 0,\\
        -iw u_{1z} + \frac{1}{\rho_0}\pd{P_1}{z} +
            \frac{\rho_1 g}{\rho_g} &= 0.
    \end{split}
\end{equation}
We substitute $N^2 = -\frac{g}{\rho_0}\pd{\rho_0}{z}$ to obtain
\begin{subequations}
    \begin{align}
        -i\omega \rho_1 - u_{1z}\frac{\rho_0N^2}{g} &= 0,\label{eq:lin.cont}\\
        iku_{1x} + \pd{u_{1z}}{z} &= 0,\label{eq:lin.div}\\
        -iw u_{1x} + \frac{ik_x P_1}{\rho_0} &= 0,\label{eq:lin.mom_x}\\
        -iw u_{1z} + \frac{1}{\rho_0}\pd{P_1}{z} +
            \frac{\rho_1g}{\rho_0} &= 0.\label{eq:lin.mom_z}
    \end{align}
\end{subequations}
Eliminating $u_{1x}$ by substituting~\eqref{eq:lin.div}
into~\eqref{eq:lin.mom_x} and $\rho_1$ by substituting~\eqref{eq:lin.cont}
into~\eqref{eq:lin.mom_z} give
\begin{subequations}
    \begin{align}
        i\omega \pd{u_{1z}}{z} + \frac{k_x^2 P_1}{\rho_0} &= 0,
            \label{eq:lin.eq1}\\
        \p*{\omega^2 - N^2 }u_{1z} + \frac{i\omega}{\rho_0}\pd{P_1}{z} &= 0.
            \label{eq:lin.eq2}
    \end{align}
\end{subequations}
Finally, we multiply~\eqref{eq:lin.eq1} with $\rho_0$ and differentiate
$\mathrm{d}z$ and combine with~\eqref{eq:lin.eq2} to give
\begin{equation}
    \rtd{u_{1z}}{z} + \frac{1}{\rho_0}\pd{\rho_0}{z}\pd{u_{1z}}{z}
        + k_x^2\p*{\frac{N^2}{\omega^2} - 1}u_{1z} = 0.\label{eq:lin.incomp.gen}
\end{equation}

Let's now pick stratification $\rho \propto e^{-z/H}$
\autoref{eq:lin.incomp.gen} clearly has exponential solutions $e^{\kappa z}$ for
\begin{equation}
    \kappa^2 - \frac{\kappa}{H} + k_x^2\p*{\frac{N^2}{\omega^2} - 1} = 0.
\end{equation}
We permit complex $\kappa = \frac{1}{2H} + ik_z$, and from the above clearly
\begin{align}
    k_z^2 &= -\frac{1}{4H^2} + k_x^2\p*{\frac{N^2}{\omega^2} - 1},\nonumber\\
    \omega^2 &= \frac{N^2k_x^2}{k_x^2 + k_z^2 + \frac{1}{4H^2}}.
\end{align}
Thus the eigenfunctions are
\begin{equation}
    \begin{split}
        u_{1z} &= e^{z/2H} e^{i(k_zz + k_xx - \omega t)},\\
        u_{1x} &= -\frac{k_z + i/2H}{k_x} u_{1z},\\
        \rho_1 &= \frac{i \rho_0}{H\omega} u_{1z},\\
        P_1 &= -\frac{\rho_0 \omega}{k_x^2}\p*{k_z + i/2H}u_{1z}.
    \end{split}\label{eq:strat_eigens}
\end{equation}

\subsection{Solving an IVP}

\section{(Algebra) The Anelastic/Boussinesq Approximations}

\subsection{Developing the Anelastic/Boussinesq Approximations}

Let's relax the incompressibility constraint (we will expand the continuity
equation to first order, but the momentum equation will merit a separate
treatment):
\begin{equation}
    \begin{split}
        \pd{\rho_1}{t} + \vec{\nabla} \cdot \p*{\rho_0 \vec{u}_1} &= 0,\\
        \pd{\vec{u}_1}{t} &= -\frac{\vec{\nabla}P}{\rho} - \vec{g}.
    \end{split}
\end{equation}
Suppose we are interested in phenomena with characteristic length scale $L$ and
time scale $\tau$. Let's first examine the relative magnitudes of the terms in
the continuity equation
\begin{equation*}
    \frac{\rho_1}{\tau} + \frac{\rho_0 \abs{u_1}}{L} = 0.
\end{equation*}
Thus, if we are interested in time scales $\tau \gg
\frac{\rho_1}{\rho_0}\frac{L}{\abs{u_1}}$ then we neglect the first term, the
time derivative. This corresponds to making the perturbation incompressible;
note that $\pd{\rho_1}{t} \approx \rd{\rho_1}{t}$ to first order, so we drop the
high frequency restoring forces in the perturbation.

For the momentum equation, we instead first manipulate to first order
\begin{align}
    -\frac{\vec{\nabla}P}{\rho} - \vec{g}
        &= -\frac{\vec{\nabla}P_0}{\rho} - \frac{\vec{\nabla}P_1}{\rho_0}
            - \vec{g},\nonumber\\
        &= - \frac{\vec{\nabla}P_1}{\rho_0} +
            \p*{\frac{\rho_0}{\rho} - 1}\vec{g},\nonumber\\
        &= -\vec{\nabla}\p*{\frac{P_1}{\rho_0}}
            - \frac{P_1}{\rho_0^2}\vec{\nabla}\rho_0
            - \frac{\rho_1}{\rho_0}\vec{g}.
\end{align}

We now have three equations for four variables, $\vec{u}_1, \rho_1, P_1$. We
must introduce a fourth equation, a thermodynamic equation. For an adiabatic
process $P\rho^{-\gamma} \propto P^{1-\gamma}T^\gamma$ is constant. We thus
introduce the concept of the \emph{potential temperature}
\begin{equation}
    \theta = T\p*{\frac{P_0}{P}}^\kappa.
        \label{eq:pot_temp}
\end{equation}
For an adiabatic process, $\rd{\theta}{t} = 0$. Motivated by this, we use
\begin{equation}
    \begin{split}
        \pd{1}{\rho_0}\pd{\rho_0}{z} &= \frac{1}{\gamma P_0}\pd{P_0}{z}
            - \frac{1}{\theta_0}\pd{\theta_0}{z},\\
        \frac{\rho_1}{\rho_0} &= \frac{1}{\gamma} \frac{P_1}{P_0}
            - \frac{\theta_1}{\theta_0},
    \end{split}
\end{equation}
to give the momentum equation form
\begin{equation}
    \rd{\vec{u}_1}{t} = -\vec{\nabla}\p*{\frac{P_1}{\rho_0}}
        + \frac{P_1}{\rho_0}\p*{\frac{1}{\theta_0}\vec{\nabla}\theta_0}
        + \vec{g} \frac{\theta_1}{\theta_0}.\label{eq:mom_unsimp}
\end{equation}

We also recognize $N^2 = \frac{g}{\theta_0}\pd{\theta_0}{z}$. We now do the same
trick where we consider dynamics on length scale $D$ and compare the first and
second terms in \autoref{eq:mom_unsimp}. Their ratio is $\frac{N^2 D}{g}$, and
so as $N^2 \ll \frac{g}{D}$ the freefall time we neglect the second term.

The anelastic fluid equations thus read
\begin{equation}
    \begin{split}
        \vec{\nabla} \cdot \p*{\rho_0\vec{u}} &= 0,\\
        \pd{\vec{u}_1}{t} + \vec{\nabla}\p*{\frac{P_1}{\rho_0}}
            - \vec{g}\frac{\theta_1}{\theta_0} &= 0,\\
        \pd{\theta_1}{t} + \p*{\vec{u} \cdot \vec{\nabla}}\theta_0 &= 0.
    \end{split}\label{eq:lin_anelastic}
\end{equation}

The Boussinesq equations are obtained from these in the limit where $H \gg D$
the relevant length scale, thus we allow $\rho_0$ to be approximately constant.

\subsection{Anelastic Solution to Stratified Atmosphere}

We simply substitute $e^{i(\vec{k} \cdot \vec{r} - \omega t)}$ into
\autoref{eq:lin_anelastic} with $\rho_0 \propto e^{-z/H}$ and obtain
\begin{equation}
    \begin{bmatrix}
        0 & 0 & ik_x\rho_0 & ik_z \rho_0 - \frac{\rho_0}{H}\\
        0 & -i\omega & 0 & \frac{N^2\theta_0}{g}\\
        \frac{ik_x}{\rho_0} & 0 & -i\omega & 0\\
        \frac{ik_z}{\rho_0} + \frac{1}{\rho_0 H} & -\frac{g}{\theta_0}
            & 0 & -i\omega
    \end{bmatrix} \begin{bmatrix}
        P_1 \\ \theta_1 \\ u_{1x} \\ u_{1z}
    \end{bmatrix} = 0.
\end{equation}
Taking the determinant of this matrix produces
\begin{align}
    -k_x^2\p*{-N^2 + \omega^2} + \p*{ik_z - \frac{1}{H}}
        \p*{ik_z + \frac{1}{H}}\omega^2 &= 0,\nonumber\\
    \frac{N^2k_x^2}{k_x^2 + k_z^2 + \frac{1}{4H^2}} &= \omega^2.
\end{align}

\chapter{2D Wave Breaking in Atmospheres}

\section{Dynamical Setup}

TODO (fluid equations, driven on bottom, parameters)

\section{Boundary Conditions}

TODO (periodic in x, show that right number of BCs in $z$, discuss whether gauge
choice).

\section{Simulation}

TODO (CFL conditions etc.)

\end{document}

