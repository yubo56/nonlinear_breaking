    \documentclass[11pt,
        usenames, % allows access to some tikz colors
        dvipsnames % more colors: https://en.wikibooks.org/wiki/LaTeX/Colors
    ]{report}
    \usepackage{
        amsmath,
        amssymb,
        fouriernc, % fourier font w/ new century book
        fancyhdr, % page styling
        lastpage, % footer fanciness
        hyperref, % various links
        setspace, % line spacing
        amsthm, % newtheorem and proof environment
        mathtools, % \Aboxed for boxing inside aligns, among others
        float, % Allow [H] figure env alignment
        enumerate, % Allow custom enumerate numbering
        graphicx, % allow includegraphics with more filetypes
        wasysym, % \smiley!
        upgreek, % \upmu for \mum macro
        listings, % writing TrueType fonts and including code prettily
        tikz, % drawing things
        booktabs, % \bottomrule instead of hline apparently
        cancel % can cancel things out!
    }
    \usepackage[margin=1in]{geometry} % page geometry
    \usepackage[
        labelfont=bf, % caption names are labeled in bold
        font=scriptsize % smaller font for captions
    ]{caption}
    \usepackage[font=scriptsize]{subcaption} % subfigures

    \newcommand*{\scinot}[2]{#1\times10^{#2}}
    \newcommand*{\bra}[1]{\left<#1\right|}
    \newcommand*{\ket}[1]{\left|#1\right>}
    \newcommand*{\dotp}[2]{\left<#1\,\middle|\,#2\right>}
    \newcommand*{\rd}[2]{\frac{\mathrm{d}#1}{\mathrm{d}#2}}
    \newcommand*{\pd}[2]{\frac{\partial#1}{\partial#2}}
    \newcommand*{\rtd}[2]{\frac{\mathrm{d}^2#1}{\mathrm{d}#2^2}}
    \newcommand*{\ptd}[2]{\frac{\partial^2 #1}{\partial#2^2}}
    \newcommand*{\md}[2]{\frac{\mathrm{D}#1}{\mathrm{D}#2}}
    \newcommand*{\norm}[1]{\left|\left|#1\right|\right|}
    \newcommand*{\abs}[1]{\left|#1\right|}
    \newcommand*{\pvec}[1]{\vec{#1}^{\,\prime}}
    \newcommand*{\svec}[1]{\vec{#1}\;\!}
    \newcommand*{\bm}[1]{\boldsymbol{\mathbf{#1}}}
    \newcommand*{\expvalue}[1]{\left<#1\right>}
    \newcommand*{\ang}[0]{\text{\AA}}
    \newcommand*{\mum}[0]{\upmu \mathrm{m}}
    \newcommand*{\at}[1]{\left.#1\right|}

    \newtheorem{theorem}{Theorem}[section]

    \let\Re\undefined
    \let\Im\undefined
    \DeclareMathOperator{\Res}{Res}
    \DeclareMathOperator{\Re}{Re}
    \DeclareMathOperator{\Im}{Im}
    \DeclareMathOperator{\Log}{Log}
    \DeclareMathOperator{\Arg}{Arg}
    \DeclareMathOperator{\Tr}{Tr}
    \DeclareMathOperator{\E}{E}
    \DeclareMathOperator{\Var}{Var}
    \DeclareMathOperator*{\argmin}{argmin}
    \DeclareMathOperator*{\argmax}{argmax}
    \DeclareMathOperator{\sgn}{sgn}
    \DeclareMathOperator{\diag}{diag\;}

    \DeclarePairedDelimiter\p{\lparen}{\rparen}
    \DeclarePairedDelimiter\s{\lbrack}{\rbrack}
    \DeclarePairedDelimiter\z{\lbrace}{\rbrace}

    % \everymath{\displaystyle} % biggify limits of inline sums and integrals
    \tikzstyle{circ} % usage: \node[circ, placement] (label) {text};
        = [draw, circle, fill=white, node distance=3cm, minimum height=2em]
    \definecolor{commentgreen}{rgb}{0,0.6,0}
    \lstset{
        basicstyle=\ttfamily\footnotesize,
        frame=single,
        numbers=left,
        showstringspaces=false,
        keywordstyle=\color{blue},
        stringstyle=\color{purple},
        commentstyle=\color{commentgreen},
        morecomment=[l][\color{magenta}]{\#}
    }

\begin{document}

\def\Snospace~{\S{}} % hack to remove the space left after autorefs
\renewcommand*{\sectionautorefname}{\Snospace}
\renewcommand*{\appendixautorefname}{\Snospace}
\renewcommand*{\figureautorefname}{Fig.}
\renewcommand*{\equationautorefname}{Eq.}
\renewcommand*{\tableautorefname}{Tab.}

\onehalfspacing

\pagestyle{fancy}
\rfoot{Yubo Su}
\cfoot{\thepage/\pageref{LastPage}}

\title{Research Notes}
\author{Yubo Su}

\maketitle

\tableofcontents

\newpage

\chapter{2D Wave Breaking in Atmospheres}

The goal of this will be to lay out a formalism that can reproduce Sutherland et
al.~2011\footnote{DOI:10.1175/JAS-D-11--097.1} and also investigate driven
oscillations versus the breaking of a single wave packet. This represents wave
breaking in the atmosphere.

\section{Dynamical Setup}

We adopt notation where $q_0$ is the background quantity and $q_1$ is the
perturbed quantity from the propagating wave.

The fluid equations are
\begin{subequations}\label{se:fl_eq}
    \begin{align}
        \pd{\rho}{t} + \vec{\nabla} \cdot \rho \vec{u} &= 0,
            \label{eq:fl_eq.cont}\\
        \rd{\vec{u}}{t} &= -\vec{\nabla}\frac{P}{\rho} - g\hat{z},
            \label{eq:fl_eq.mom}
    \end{align}
\end{subequations}
where we will take gravity to be uniform throughout the domain of interest. We
will study for no background flow $\vec{u}_0 = 0$ and in the presence of
stratification $\rho_0 \propto e^{-z/H}$. In the absence of any perturbations it
is then easy to show that $\rd{P_0}{z} = -\rho_0 g$.

\subsection{Linear, Incompressible}

\subsubsection{Solution for Arbitrary Stratification}

First, we solve the incompressible case $c_s^2 \to \infty, \vec{\nabla} \cdot
\vec{u} = 0$ in the linear regime. For funsies, we solve for arbitrary
stratification first. The fluid equations to first order reduce to
\begin{equation}
    \begin{split}
        \pd{\rho_1}{t} + \p*{\vec{u}_1 \cdot \vec{\nabla}}\rho_0 &= 0,\\
        \vec{\nabla} \cdot \vec{u}_1 &= 0,\\
        \pd{\vec{u}_1}{t} &= -\frac{\vec{\nabla}P_1}{\rho_0}
            - P_1 \vec{\nabla}\frac{1}{\rho_0}
    \end{split}\label{eq:lin_incomp}
\end{equation}
We expect there to be some $z$ dependence in the amplitude, so we substitute
variables of form $e^{i(kx - \omega t)}$ and do not specify the $z$ dependence.
This gives us
\begin{equation}
    \begin{split}
        -i\omega \rho_1 - u_{1z}\rd{\rho_0}{z} &= 0,\\
        iku_{1x} + \rd{u_{1z}}{z} &= 0,\\
        -iw u_{1x} + \frac{ik_x P_1}{\rho_0} &= 0,\\
        -iw u_{1z} + \frac{1}{\rho_0}\pd{P_1}{z} +
            \frac{\rho_1}{\rho_0^2}\rd{P_0}{z} &= 0.
    \end{split}
\end{equation}
We substitute $N^2 = -\frac{g}{\rho_0}\rd{\rho_0}{z}$ and $\rd{P_0}{z} = -\rho
g$ to obtain
\begin{subequations}
    \begin{align}
        -i\omega \rho_1 - u_{1z}\frac{\rho_0N^2}{g} &= 0,\label{eq:lin.cont}\\
        iku_{1x} + \rd{u_{1z}}{z} &= 0,\label{eq:lin.div}\\
        -iw u_{1x} + \frac{ik_x P_1}{\rho_0} &= 0,\label{eq:lin.mom_x}\\
        -iw u_{1z} + \frac{1}{\rho_0}\pd{P_1}{z} +
            \frac{\rho_1g}{\rho_0} &= 0.\label{eq:lin.mom_z}
    \end{align}
\end{subequations}
Eliminating $u_{1x}$ by substituting~\eqref{eq:lin.div}
into~\eqref{eq:lin.mom_x} and $\rho_1$ by substituting~\eqref{eq:lin.cont}
into~\eqref{eq:lin.mom_z} give
\begin{subequations}
    \begin{align}
        i\omega \rd{u_{1z}}{z} + \frac{k_x^2 P_1}{\rho_0} &= 0,
            \label{eq:lin.eq1}\\
        \p*{\omega^2 - N^2 }u_{1z} + \frac{i\omega}{\rho_0}\rd{P_1}{z} &= 0.
            \label{eq:lin.eq2}
    \end{align}
\end{subequations}
Finally, we multiply~\eqref{eq:lin.eq1} with $\rho_0$ and differentiate
$\mathrm{d}z$ and combine with~\eqref{eq:lin.eq2} to give
\begin{equation}
    \rtd{u_{1z}}{z} + \frac{1}{\rho_0}\rd{\rho_0}{z}\rd{u_{1z}}{z}
        + k_x^2\p*{\frac{N^2}{\omega^2} - 1}u_{1z} = 0.\label{eq:lin.incomp.gen}
\end{equation}

\subsubsection{Introduce Stratification}

With stratification $\rho \propto e^{-z/H}$ \autoref{eq:lin.incomp.gen} clearly
has exponential solutions $e^{\kappa z}$ for
\begin{equation}
    \kappa^2 - \frac{\kappa}{H} + k_x^2\p*{\frac{N^2}{\omega^2} - 1} = 0.
\end{equation}

We permit complex $\kappa = -\frac{1}{2H} + ik_z$, and from the above clearly
\begin{align}
    k_z^2 &= -\frac{1}{4H^2} + k_x^2\p*{\frac{N^2}{\omega^2} - 1},\nonumber\\
    \omega^2 &= \frac{N^2k_x^2}{k_x^2 + k_z^2 + \frac{1}{4H^2}}.
\end{align}

\subsection{Linear Regime, Compressible}

\subsection{Boundary Conditions}

We must bound this to a finite domain. We choose periodic boundary conditions in
$x$ with total length $L_x$, $x \in \s*{-\frac{L}{2}, \frac{L}{2}}$. We choose
$z$ dimension to have total length $L_z$, $z \in [0, L_z]$.

To set up the boundary condition at $z = 0$, we recall that gravity waves in an
atmosphere with $e^{-z/H}$ profile have form
\begin{equation}
    u_{1z} \propto e^{\frac{z}{2H}}e^{i\p*{k_xx + k_zz - \omega t}},
        \label{eq:bc_below}
\end{equation}
where ($N^2 \equiv \frac{g}{H}$ is the Brunt-V\"ais\"al\"a frequency in an
incompressible, stratified atmosphere)
\begin{equation}
    \omega^2 = \frac{N^2 k_x^2}{k_x^2 + k_z^2 + \frac{1}{4H^2}}.
        \label{eq:omega}
\end{equation}
Thus, at constant $z = 0$ we must have
\begin{equation}
    \vec{u}_{1z}(z = 0) \propto e^{i(k_xx - \omega t)}\label{eq:bc_u1z}.
\end{equation}

The boundary condition at $z = L_z$ is much harder to determine. It is clear
that $\vec{u}_1(z = \infty) = \rho_1(z = \infty) = 0$, and for $L_z \gg H$ this
would be a reasonable approximation, if simply because we expect the majority of
the wave to dissipate via turbulent dissipation as $z$ reaches many $H$. We will
simply choose the BC to be many multiples of $H$. We can solve with both a
Dirichlet and Neumann BC and compare the two solutions; if the solutions
differ significantly then we must choose a larger $L_z$. These are the only two
solutions that can be implemented where we do not need the phase of the linear
wave, which we lose during the nonlinear breaking region, to relate the function
and derivative at the boundary.

For the boundary conditions on $\rho$, we note that in the linear regime it
should just have a phase offset from $u_{1z}$, thus we choose $\rho(z = 0)
\propto e^{\frac{z}{2H}}ie^{i\p*{k_xx + k_zz - \omega t}}$ and a similar
treatment at $z = L_z$, taking both a Dirichlet and Neumann BC\@.

\subsection{Simulation}

We begin our simulation with $\rho_1 = 0, \vec{u}_1 = 0$ strictly within the
domain of simulation. We will borrow some values from Sutherland's paper and use
$k_z = 2\;\mathrm{km^{-1}}$ then define $k_x = -0.4k_z, H = 10 / k_z, A = 0.05 /
k_z, L_z = 300 / k_z, L_x = 20 / k_z$, We also use $\mu \approx 29, T =
273\;\mathrm{K}, \rho_0 = 1\;\mathrm{kg/m^3}, P_0 = \frac{\rho_0 k_BT}{\mu m_p},
g = 10\;\mathrm{m/s^2}$.

\end{document}

