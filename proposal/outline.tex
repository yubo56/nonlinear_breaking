    \documentclass[11pt,
        usenames, % allows access to some tikz colors
        dvipsnames % more colors: https://en.wikibooks.org/wiki/LaTeX/Colors
    ]{article}
    \usepackage{
        amsmath,
        amssymb,
        fouriernc, % fourier font w/ new century book
        fancyhdr, % page styling
        lastpage, % footer fanciness
        hyperref, % various links
        setspace, % line spacing
        amsthm, % newtheorem and proof environment
        mathtools, % \Aboxed for boxing inside aligns, among others
        float, % Allow [H] figure env alignment
        enumerate, % Allow custom enumerate numbering
        graphicx, % allow includegraphics with more filetypes
        wasysym, % \smiley!
        upgreek, % \upmu for \mum macro
        listings, % writing TrueType fonts and including code prettily
        tikz, % drawing things
        booktabs, % \bottomrule instead of hline apparently
        cancel % can cancel things out!
    }
    \usepackage[margin=1in]{geometry} % page geometry
    \usepackage[
        labelfont=bf, % caption names are labeled in bold
        font=scriptsize % smaller font for captions
    ]{caption}
    \usepackage[font=scriptsize]{subcaption} % subfigures

    \newcommand*{\scinot}[2]{#1\times10^{#2}}
    \newcommand*{\bra}[1]{\left<#1\right|}
    \newcommand*{\ket}[1]{\left|#1\right>}
    \newcommand*{\dotp}[2]{\left<#1\,\middle|\,#2\right>}
    \newcommand*{\rd}[2]{\frac{\mathrm{d}#1}{\mathrm{d}#2}}
    \newcommand*{\pd}[2]{\frac{\partial#1}{\partial#2}}
    \newcommand*{\rtd}[2]{\frac{\mathrm{d}^2#1}{\mathrm{d}#2^2}}
    \newcommand*{\ptd}[2]{\frac{\partial^2 #1}{\partial#2^2}}
    \newcommand*{\md}[2]{\frac{\mathrm{D}#1}{\mathrm{D}#2}}
    \newcommand*{\norm}[1]{\left|\left|#1\right|\right|}
    \newcommand*{\abs}[1]{\left|#1\right|}
    \newcommand*{\pvec}[1]{\vec{#1}^{\,\prime}}
    \newcommand*{\svec}[1]{\vec{#1}\;\!}
    \newcommand*{\bm}[1]{\boldsymbol{\mathbf{#1}}}
    \newcommand*{\expvalue}[1]{\left<#1\right>}
    \newcommand*{\ang}[0]{\text{\AA}}
    \newcommand*{\mum}[0]{\upmu \mathrm{m}}
    \newcommand*{\at}[1]{\left.#1\right|}

    \newtheorem{theorem}{Theorem}[section]

    \let\Re\undefined
    \let\Im\undefined
    \DeclareMathOperator{\Res}{Res}
    \DeclareMathOperator{\Re}{Re}
    \DeclareMathOperator{\Im}{Im}
    \DeclareMathOperator{\Log}{Log}
    \DeclareMathOperator{\Arg}{Arg}
    \DeclareMathOperator{\Tr}{Tr}
    \DeclareMathOperator{\E}{E}
    \DeclareMathOperator{\Var}{Var}
    \DeclareMathOperator*{\argmin}{argmin}
    \DeclareMathOperator*{\argmax}{argmax}
    \DeclareMathOperator{\sgn}{sgn}
    \DeclareMathOperator{\diag}{diag\;}

    \DeclarePairedDelimiter\p{\lparen}{\rparen}
    \DeclarePairedDelimiter\s{\lbrack}{\rbrack}
    \DeclarePairedDelimiter\z{\lbrace}{\rbrace}

    % \everymath{\displaystyle} % biggify limits of inline sums and integrals
    \tikzstyle{circ} % usage: \node[circ, placement] (label) {text};
        = [draw, circle, fill=white, node distance=3cm, minimum height=2em]
    \definecolor{commentgreen}{rgb}{0,0.6,0}
    \lstset{
        basicstyle=\ttfamily\footnotesize,
        frame=single,
        numbers=left,
        showstringspaces=false,
        keywordstyle=\color{blue},
        stringstyle=\color{purple},
        commentstyle=\color{commentgreen},
        morecomment=[l][\color{magenta}]{\#}
    }

\begin{document}

\def\Snospace~{\S{}} % hack to remove the space left after autorefs
\renewcommand*{\sectionautorefname}{\Snospace}
\renewcommand*{\appendixautorefname}{\Snospace}
\renewcommand*{\figureautorefname}{Fig.}
\renewcommand*{\equationautorefname}{Eq.}
\renewcommand*{\tableautorefname}{Tab.}

\section*{Timelines}

\begin{itemize}
    \item 12/31/17---NDSEG \url{http://www.ndsegfellowships.org/application}
        \begin{itemize}
            \item \emph{Please submit a proposal relevant to the DoD Agency
                (AIRFORCE, ARMY, ONR)/field that interest you.} (Pitch as
                ``science of extreme materials, results that can be useful in
                understanding turbulence''?)
            \item 3 page proposal.
        \end{itemize}

    \item 01/10/17---NNSA SSGF \url{
        https://www.krellinst.org/ssgf/about-doe-nnsa-ssgf/fields-study}
        \begin{itemize}
            \item Promote stewardship science (National Nuclear Security
                Administration).
            \item Fields of study: \textbf{High Energy Density Physics,
                Materials under Extreme Conditions and Hydrodynamics}.
            \item Ask to propose \emph{program of study}, I picked:
                \begin{itemize}
                    \item Advanced Plasmas
                    \item Physics of BHs, WDs, stars
                    \item Theory stellar structure/evolution
                    \item Applications of Parallel Computers
                    \item Computational fluid dynamics
                    \item Turbulence/turbulent flows (not sure this one)
                \end{itemize}
            \item Essays:
                \begin{itemize}
                    \item 500w area of research
                    \item 300w intellectual excitement
                    \item 300w ``how your research will directly compile to
                        stockpile stewardship in your area'' Advice from
                        theoretical astrophysicist: ``\emph{I pretty much argued
                        that SNe are labs for many different types of extreme
                        physics and can help inform ground-based experiments.
                        Also many of the same principles in my simulations could
                        be applied to other things. Depending on what you study
                        it shouldn't be too hard to justify (especially if
                        you're into plasmas, high energy particles, fluid
                        dynamics, etc.)}''
                \end{itemize}
        \end{itemize}

    \item 01/17/17---DOE CSGF \url{
        https://www.krellinst.org/csgf/about-doe-csgf/fields-study}
        \begin{itemize}
            \item Math + Computers to study fields e.g.\ astrophysics, etc.,
                promote interdisciplinary, require broad program of study.
            \item Possible proposal: \textbf{Propose to develop new software? Or
                just say that we'll be ready to, just in case.}
            \item Program of Study:
                \begin{itemize}
                    \item (Science/Eng) Physics BH, WD, NS
                    \item (Science/Eng) Computational Fluid Dynamics
                    \item (Math/Stats) Applied Dynamical Systems
                    \item (Math/Stats) Functional Analysis
                    \item (Computer Science) Applications of Parallel Computers
                    \item (Computer Science) Advanced Programming Languages
                \end{itemize}
            \item Essays:
                \begin{itemize}
                    \item 2250c Field of Interest, role of CS
                    \item 2250c use of CS + math in proposed research
                    \item 2250c justify program of study.
                \end{itemize}
        \end{itemize}
    \item 02/01/17---NSPIRES NESSF \url{
        https://nspires.nasaprs.com/external/solicitations/summary.do?method=init&solId={1A7F0A8C-1126-EA39-6584-10DBD500C8AC}&path=open
        }
        \begin{itemize}
            \item \textbf{Adviser starts}, check instructions at \href{
                https://nspires.nasaprs.com/external/viewrepositorydocument/cmdocumentid=539448/solicitationId=\%7B1A7F0A8C-1126-EA39-6584-10DBD500C8AC\%7D/viewSolicitationDocument=1/NESSF\%202017-2018\%20Proposal\%20Submission\%20Instructions.pdf}{here}
            \item Proposal limit 6 pages
        \end{itemize}
\end{itemize}

\newpage

\title{Nonlinear Tidal Dissipation in White Dwarfs}
\author{Yubo Su}
\date{}

\maketitle

\section{Introduction}

\subsection{Background}

\begin{itemize}
    \item \emph{Steal a bunch from Fuller et al}
    \item WD-WD systems are important, produce sdO stars, giant stars,
        Ia supernovae (Fuller Paper II citations)
    \item WD-BH systems, tidal disruption produces observable flares
        (https://arxiv.org/pdf/1701.08162.pdf)
    \item WD binaries are in LISA sensitivity band, GW astronomy is powerful
        tool evidenced by the recent LIGO success (FullerII)
    \item Tidal dissipation can contribute significantly to the brightnesses of
        these binaries (FullerII)
\end{itemize}

\subsection{Previous Work}

\begin{itemize}
    \item Fuller + Lai determined in WD binaries, interval gravity waves are
        excited that propagate towards the surface and nonlinear dissipate in
        some WDs.
    \item Predicted dissipation luminosity dominates black body dissipation,
        significant contribution to energy dynamics. (FullerII)
    \item State of the art is WKB (FullerIV)
    \item Sensitive probe of WD conditions/models, e.g.\ discern whether certain
        white dwarfs are undergoing strong tidal heating (FullerIV).
    \item Require numerical simulations to determine dissipation profile.
\end{itemize}

\section{Proposed Research}

\textbf{Propose to study nonlinear wave breaking in white dwarf binaries via
numerical simulation}.

\begin{itemize}
    \item Research Plan
        \begin{itemize}
            \item For various WD models + compositions, run simulations to
                determine $E, \vec{L}$ dissipation distribution.
            \item Start with spectral code (Dedalus), existing literature shows
                that handles instability-driven turbulence with high accuracy
                and speed.
            \item Have the computational expertise to modify existing
                codes/build new code if existing ones are insufficient.
            \item Determine whether observed events e.g.\ Brown et al 2011 match
                our predictions for any models, other observational signatures.
            \item Study model output for simpler models to develop analytical
                tools.
        \end{itemize}
    \item Further applications
        \begin{itemize}
            \item Forming hot Jupiters (Michelle paper)
            \item \textbf{LIGO detecting kilonovae means many energetic
                phenomena to study, any chance to apply similar approach?}
        \end{itemize}
\end{itemize}


\end{document}

