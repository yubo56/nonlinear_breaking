    \documentclass[11pt,
        usenames, % allows access to some tikz colors
        dvipsnames % more colors: https://en.wikibooks.org/wiki/LaTeX/Colors
    ]{article}
    \usepackage{
        amsmath,
        amssymb,
        fouriernc, % fourier font w/ new century book
        fancyhdr, % page styling
        lastpage, % footer fanciness
        hyperref, % various links
        setspace, % line spacing
        amsthm, % newtheorem and proof environment
        mathtools, % \Aboxed for boxing inside aligns, among others
        float, % Allow [H] figure env alignment
        enumerate, % Allow custom enumerate numbering
        graphicx, % allow includegraphics with more filetypes
        wasysym, % \smiley!
        upgreek, % \upmu for \mum macro
        listings, % writing TrueType fonts and including code prettily
        tikz, % drawing things
        booktabs, % \bottomrule instead of hline apparently
        cancel % can cancel things out!
    }
    \usepackage[margin=1in]{geometry} % page geometry
    \usepackage[
        labelfont=bf, % caption names are labeled in bold
        font=scriptsize % smaller font for captions
    ]{caption}
    \usepackage[font=scriptsize]{subcaption} % subfigures

    \newcommand*{\scinot}[2]{#1\times10^{#2}}
    \newcommand*{\bra}[1]{\left<#1\right|}
    \newcommand*{\ket}[1]{\left|#1\right>}
    \newcommand*{\dotp}[2]{\left<#1\,\middle|\,#2\right>}
    \newcommand*{\rd}[2]{\frac{\mathrm{d}#1}{\mathrm{d}#2}}
    \newcommand*{\pd}[2]{\frac{\partial#1}{\partial#2}}
    \newcommand*{\rtd}[2]{\frac{\mathrm{d}^2#1}{\mathrm{d}#2^2}}
    \newcommand*{\ptd}[2]{\frac{\partial^2 #1}{\partial#2^2}}
    \newcommand*{\md}[2]{\frac{\mathrm{D}#1}{\mathrm{D}#2}}
    \newcommand*{\norm}[1]{\left|\left|#1\right|\right|}
    \newcommand*{\abs}[1]{\left|#1\right|}
    \newcommand*{\pvec}[1]{\vec{#1}^{\,\prime}}
    \newcommand*{\svec}[1]{\vec{#1}\;\!}
    \newcommand*{\bm}[1]{\boldsymbol{\mathbf{#1}}}
    \newcommand*{\expvalue}[1]{\left<#1\right>}
    \newcommand*{\ang}[0]{\text{\AA}}
    \newcommand*{\mum}[0]{\upmu \mathrm{m}}
    \newcommand*{\at}[1]{\left.#1\right|}

    \newtheorem{theorem}{Theorem}[section]

    \let\Re\undefined
    \let\Im\undefined
    \DeclareMathOperator{\Res}{Res}
    \DeclareMathOperator{\Re}{Re}
    \DeclareMathOperator{\Im}{Im}
    \DeclareMathOperator{\Log}{Log}
    \DeclareMathOperator{\Arg}{Arg}
    \DeclareMathOperator{\Tr}{Tr}
    \DeclareMathOperator{\E}{E}
    \DeclareMathOperator{\Var}{Var}
    \DeclareMathOperator*{\argmin}{argmin}
    \DeclareMathOperator*{\argmax}{argmax}
    \DeclareMathOperator{\sgn}{sgn}
    \DeclareMathOperator{\diag}{diag\;}

    \DeclarePairedDelimiter\p{\lparen}{\rparen}
    \DeclarePairedDelimiter\s{\lbrack}{\rbrack}
    \DeclarePairedDelimiter\z{\lbrace}{\rbrace}

    % \everymath{\displaystyle} % biggify limits of inline sums and integrals
    \tikzstyle{circ} % usage: \node[circ, placement] (label) {text};
        = [draw, circle, fill=white, node distance=3cm, minimum height=2em]
    \definecolor{commentgreen}{rgb}{0,0.6,0}
    \lstset{
        basicstyle=\ttfamily\footnotesize,
        frame=single,
        numbers=left,
        showstringspaces=false,
        keywordstyle=\color{blue},
        stringstyle=\color{purple},
        commentstyle=\color{commentgreen},
        morecomment=[l][\color{magenta}]{\#}
    }

\begin{document}

\def\Snospace~{\S{}} % hack to remove the space left after autorefs
\renewcommand*{\sectionautorefname}{\Snospace}
\renewcommand*{\appendixautorefname}{\Snospace}
\renewcommand*{\figureautorefname}{Fig.}
\renewcommand*{\equationautorefname}{Eq.}
\renewcommand*{\tableautorefname}{Tab.}

\onehalfspacing

\textbf{Yubo Su Personal Statement}\\[\parskip]

As an aspiring theoretical astrophysics professor, my goals consist of
developing both a strong grasp of pertinent numerical and analytical techniques
and a broad understanding of all science that can be observed in astrophysical
systems. Having this extra understanding will allow me to contribute to various
fields using the natural laboratories that astronomical objects supply, probing
physics inacessible to traditional experiment. To this end, the NDSEG fellowship
along with its associated benefits are a great fit for my goals, and I would
look forward to contributing to the NDSEG community and the scientific community
at large.

My immediate priority in becoming a professor is to seek highly versatile,
interdisciplinary research to best prepare myself for future work. Such research
will build up a core skillset that can be applied to various fields. My proposed
research, numerical studies of nonlinear wave breaking in white dwarfs, is an
example of such research that combines high energy physics, fluid dynamics and
numerics. A second core component of building a transferrable skillset is
collaborating with scientists with diverse areas of expertise. The NDSEG
practicum will be a great opportunity both to meet research leaders in relevant
but different fields and to work alongside fellow growing scientists.

Ultimately, I seek to become a research professor developing simple, tractable
mobels of behaviors in astrophysical systems as probes of conditions
beyond man-made laboratories. I believe such models are invaluable for turning
astronomical observations into constraints useful for other fields of study. My
proposed work accomplishes this by turning observed white dwarf luminosity and
orbital parameters into constraints on the behavior of matter under extreme
conditions. My preference for simpler models is rooted in my excitement to
teach; intuitive models are more pedagogically suitable. Many of my best-taught
classes were transformative experiences for me, and I hope as part of my job to
instill the same clarity as did my professors.

My academic career to date has revolved around my goal of becoming an
interdisciplinary research and teaching faculty member developing simplified
models of intractable behavior. At Caltech, I pursued a double major and a
broad courseload that prepared me to apply many skills flexibly in my research.
I then worked for a year at a software startup, where I brought in many of my
problem solving skills from my physics background, and am now bringing industry
best practice to my research efforts at Cornell. I will therefore be well
prepared to bilaterally share knowledge between a DoD practicum and Cornell
research groups. Now, I have begun research with the versatile Professor Dong
Lai whose research centers on the same principle of distilling intractable
problems into intuitive, simplified models. My present track lines up well with
my desired goals, and I am primed to both contribute to and learn from an NDSEG
fellowship.

% short term goals: perform highly interdisciplinary research that will enable
% me to contribute to a variety of fields and bring novel tools across
% specialties.
%   - build core "swiss army knife" to apply to variety of problems
%   - practice learning new ideas, i.e. breadth ~= depth, cross-pollination
% long term goals: astrophysics research professor studying them as natural
% laboratories for simple, tractable models of behavior inaccessible to
% experiment
%   - strong grasp of wide breadth of literature to understand what mechasims
%   can be at play
%   - passion for teaching, information dissemination
% goals developed from desire to study complicated problems with easily grasped
% explanations
%   - implications beyond target problem
%   - demonstration of principle of prescient analysis, outreach inspiration
% I have already begun to lay foundation for this
%   - research with multidisciplinary Dong Lai, strong emphasis on simplified
%   problems
%   - bringing new tools to table; double major, software best practice
% fellowship fits into goals of understanding a wide breadth of material at a
% strong enough level for application to unrelated problems

\end{document}

