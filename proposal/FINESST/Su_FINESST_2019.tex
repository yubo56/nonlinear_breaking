    \documentclass[12pt,
        usenames, % allows access to some tikz colors
        dvipsnames % more colors: https://en.wikibooks.org/wiki/LaTeX/Colors
    ]{article}
    \usepackage{
        amsmath,
        amssymb,
        times,
        fancyhdr, % page styling
        lastpage, % footer fanciness
        hyperref, % various links
        setspace, % line spacing
        amsthm, % newtheorem and proof environment
        mathtools, % \Aboxed for boxing inside aligns, among others
        float, % Allow [H] figure env alignment
        enumerate, % Allow custom enumerate numbering
        graphicx, % allow includegraphics with more filetypes
        wasysym, % \smiley!
        upgreek, % \upmu for \mum macro
        listings, % writing TrueType fonts and including code prettily
        tikz, % drawing things
        booktabs, % \bottomrule instead of hline apparently
        cancel % can cancel things out!
    }
    \usepackage[margin=1in]{geometry} % page geometry
    \usepackage[
        labelfont=bf, % caption names are labeled in bold
        font=scriptsize % smaller font for captions
    ]{caption}
    \usepackage[font=scriptsize]{subcaption} % subfigures

    \newcommand*{\scinot}[2]{#1\times10^{#2}}
    \newcommand*{\dotp}[2]{\left<#1\,\middle|\,#2\right>}
    \newcommand*{\rd}[2]{\frac{\mathrm{d}#1}{\mathrm{d}#2}}
    \newcommand*{\pd}[2]{\frac{\partial#1}{\partial#2}}
    \newcommand*{\rtd}[2]{\frac{\mathrm{d}^2#1}{\mathrm{d}#2^2}}
    \newcommand*{\ptd}[2]{\frac{\partial^2 #1}{\partial#2^2}}
    \newcommand*{\md}[2]{\frac{\mathrm{D}#1}{\mathrm{D}#2}}
    \newcommand*{\pvec}[1]{\vec{#1}^{\,\prime}}
    \newcommand*{\svec}[1]{\vec{#1}\;\!}
    \newcommand*{\bm}[1]{\boldsymbol{\mathbf{#1}}}
    \newcommand*{\ang}[0]{\;\text{\AA}}
    \newcommand*{\mum}[0]{\;\upmu \mathrm{m}}
    \newcommand*{\at}[1]{\left.#1\right|}

    \newtheorem{theorem}{Theorem}[section]

    \let\Re\undefined
    \let\Im\undefined
    \DeclareMathOperator{\Res}{Res}
    \DeclareMathOperator{\Re}{Re}
    \DeclareMathOperator{\Im}{Im}
    \DeclareMathOperator{\Log}{Log}
    \DeclareMathOperator{\Arg}{Arg}
    \DeclareMathOperator{\Tr}{Tr}
    \DeclareMathOperator{\E}{E}
    \DeclareMathOperator{\Var}{Var}
    \DeclareMathOperator*{\argmin}{argmin}
    \DeclareMathOperator*{\argmax}{argmax}
    \DeclareMathOperator{\sgn}{sgn}
    \DeclareMathOperator{\diag}{diag\;}

    \DeclarePairedDelimiter\bra{\langle}{\rvert}
    \DeclarePairedDelimiter\ket{\lvert}{\rangle}
    \DeclarePairedDelimiter\abs{\lvert}{\rvert}
    \DeclarePairedDelimiter\ev{\langle}{\rangle}
    \DeclarePairedDelimiter\p{\lparen}{\rparen}
    \DeclarePairedDelimiter\s{\lbrack}{\rbrack}
    \DeclarePairedDelimiter\z{\lbrace}{\rbrace}

    % \everymath{\displaystyle} % biggify limits of inline sums and integrals
    \tikzstyle{circ} % usage: \node[circ, placement] (label) {text};
        = [draw, circle, fill=white, node distance=3cm, minimum height=2em]
    \definecolor{commentgreen}{rgb}{0,0.6,0}
    \lstset{
        basicstyle=\ttfamily\footnotesize,
        frame=single,
        numbers=left,
        showstringspaces=false,
        keywordstyle=\color{blue},
        stringstyle=\color{purple},
        commentstyle=\color{commentgreen},
        morecomment=[l][\color{magenta}]{\#}
    }
    \usepackage[
        backend=bibtex8,
        style=nature]{biblatex}
% \defbibenvironment{bibliography}
%   {\noindent}
%   {\unspace}
%   {\printtext[labelnumberwidth]{%
%      \printfield{labelprefix}%
%      \printfield{labelnumber}}%
%    \addspace}
% \renewbibmacro*{finentry}{\finentry\addspace}
\bibliography{Su_FINESST_2019}
\AtEveryBibitem{\clearfield{title}}
\begin{document}

% \def\Snospace~{\S{}} % hack to remove the space left after autorefs
\renewcommand*{\sectionautorefname}{Section}
% \renewcommand*{\appendixautorefname}{\Snospace}
% \renewcommand*{\figureautorefname}{Fig.}
% \renewcommand*{\equationautorefname}{Eq.}
% \renewcommand*{\tableautorefname}{Tab.}

\singlespacing

\pagestyle{fancy}
\rfoot{Yubo Su}
\rhead{}
\cfoot{\thepage/\pageref{LastPage}}

\title{Nonlinear Tidal Dissipation in Binary White Dwarfs}
\author{Yubo Su}
\date{}

\maketitle

\section{Background and Proposed Research}\label{s:1}

\subsection{White Dwarf Binaries}

Compact white dwarf (WD) binary systems, with orbital periods in the range of
minutes to hours, are important for a range of astrophysical problems. They are
the most important sources of gravitational waves (GWs) for the Laser
Interferometric Space Antenna (LISA)\cite{lisa}. They are also thought to
produce interesting optical transients such as underluminous
supernovae\cite{underlum}, Ca-rich fast transients\cite{carich}, and tidal
novae\cite{tidal_novae}. Most importantly, they have been proposed as the likely
progenitors of type Ia supernovae (e.g.~\cite{Ia0,webbink} or more
recently\cite{Ia1,Ia2}). While presently only a few tens of compact WD binaries
are known\cite{lsst_wd}, \emph{Gaia} (currently gathering data) is expected to
expand the catalog to a few hundreds\cite{lsst_wd} (results based on
\emph{Gaia}'s second data release have already begun to
appear\cite{gaiaDD,gaiaDD2}), and the Large Synoptic Survey Telescope (LSST,
first light scheduled for 2020) will likely detect a few thousand
more\cite{lsst_wd}. These observations will significantly advance the
understanding of WD binaries and their evolution. My proposed theoretical and
computational research is well-timed to take advantage of these new advances.

In spite of the broad importance of WD binaries, the evolution of these systems
prior to their final mergers is not well understood. Much of this uncertainty
comes from our imprecise understanding of tidal interactions, which play an
important role during a compact WD binary's inspiral\cite{fullerII}. Previous
studies have shown that these interactions manifest as tidal excitation of
internal gravity waves (IGW), waves in the WD fluid restored by the buoyancy
force due to density stratification\cite{fullerI}. As these waves propagate
outwards towards the WD surface, they grow in amplitude until they break, as do
ocean waves on a shore, and transfer both energy and angular momentum from the
binary orbit to the outer envelope of the WD\cite{fullerI,fullerII}.

Previous works have found that the dissipation of IGW can generate significantly
more energy than thermal radiation from the isolated WD surface and is thus a
major contributor to the WD energy budget\cite{fullerII,fullerIV}. However,
these works parameterized the wave breaking process in an ad hoc manner. The
details of dissipation, namely the location and spatial extent of the wave
breaking, affect the observable outcome: dissipation near the surface of the WD
can be efficiently radiated away and simply brightens the WD, while dissipation
deep in the WD envelope causes an energy buildup that results in energetic
flares\cite{tidal_novae}. Works in other fields based on numerical simulations
show that strongly nonlinear wave breaking process exhibits new behavior that
cannot be described by linear and weakly nonlinear
theory\cite{winters1994,barker_ogilvie}. Such fully nonlinear numerical
simulations have not been performed for WDs.

\subsection{Goals of Proposed Research}

Characterizing the location and spatial extent of tidal dissipation in WD
binaries will require numerical simulation to capture the turbulent cascade to
small scales that causes wave breaking. \textbf{I propose to study the dynamical
effects of tidal dissipation via nonlinear Internal Gravity Wave (IGW) breaking
in binary WDs.} There are three specific goals of my proposed research:
\begin{itemize}
    \item Characterize the location, spatial extent, and other properties of
        wave breaking in realistic WD models via direct numerical simulation.
        The location and spatial extent of wave breaking will furnish a simple
        yet effective parameterization of tidal dissipation in a range of WD
        models. \autoref{s:2} describes the steps required to acheive this goal.

    \item Predict signatures of tidal dissipation over a wide range of possible
        WD systems. In particular, I will study the impact of tidal heating
        on the luminosity of WDs in binaries and explore the possibility of
        producing of observable flares. \autoref{s:3} details my plan.

    \item Compute modified GW templates for LISA that account for changes in the
        phase evolution of the orbit due to tidal dissipation. \autoref{s:4}
        elaborates on how I will perform this computation.
\end{itemize}

\section{Nonlinear Tidal Dissipation}\label{s:2}

\subsection{Background and Preliminary Work}

The current understanding of tidal synchronization in WD binaries is laid out
in~\cite{fullerII}: tidal forces from the companion excites IGWs in the deep
envelope of the WDs. These IGW propagate outwards and undergo wave breaking in
the outer envelope of the WD, locally depositing angular momentum and
synchronizing the WD spin to the binary orbit. A similar process also
operates in binaries consisting of early type stars\cite{zahn75,gn89},
the only major difference being in the specifics of wave
excitation\footnote{While IGWs in massive stars are excited at their
radiative-convective boundaries, the excitation of IGWs in WDs is more gradual
and is associated with sharp composition changes in the stellar
envelope\cite{fullerII}.}. Nevertheless, direct numerical simulation of the wave
breaking process has not been performed in either of these systems. Since wave
breaking is a strongly nonlinear phenomenon, where a larger wave breaks down
into many smaller-scale waves, numerical simulation is paramount to an accurate
understanding of the tidal dissipation process.

IGW breaking has been studied in atmospheric sciences. The wave
breaking process proceeds as follows: Initially, as the IGW reaches nonlinear
amplitudes, it breaks down via the parametric subharmonic instability and
transfers energy and angular momentum from the wave to the mean flow of the
fluid\cite{drazin}; after the mean flow velocity reaches the
horizontal phase velocity of the IGW, a critical layer forms. Analytical
calculations show that the IGW is nearly completely absorbed at this critical
layer in the linear approximation and endows the atmosphere with a mean
horizontal flow\cite{booker_bretherton,hazel}. However, when this mean flow
absorption was numerically studied including full nonlinear interactions, new
phenomena not described by the linear theory (reflection off the critical layer
and sharpening of the mean flow) were observed\cite{jones_num,winters1994}; this
nonlinear behavior significantly affectedthe evolution of the atmosphere over
time. \emph{This highlights the importance of numerical simulation in capturing
the wave breaking process.}

To gain insight into the tidal dissipation process, I began by adapting the
spectral hydrodynamics code Dedalus\cite{dedalus} to study IGW breaking in a 2D
isothermal, stratified atmosphere. A spectral code like Dedalus is ideal for
simulating complex hydrodynamical phenomena, as spectral methods have no
inherent numerical viscosity and so can more accurately resolve the nonlinear
cascade to small length scales in wave breaking.

Working with Dr.\ Daniel Lecoanet (a post-doc at Princeton University, and one
of the authors of the Dedalus code) and my advisor, Prof.\ Dong Lai, I have
simulated the nonlinear evolution of an upward-propagating IGW wavetrain excited
at the bottom of the atmosphere. My simulations show the waves breaking and
depositing horizontal momentum in the fluid, causing the fluid to acquire an
average horizontal flow, consistent with previous studies\cite{fullerII}. I have
derived simple formulae for the location and spatial extent of the dissipation
zone where the IGW is absorbed by the fluid. More interestingly, my simulation
reveals a partial reflection of the IGW at the critical layer\cite{me}, a
phenomenon not considered in the current astrophysical literature but consistent
with the aforementioned results\cite{winters1994}. A sample simulation is
presented in \autoref{fig:nl_fluxes}. I am preparing these results for
publication\cite{me}.

\begin{figure}[!h]
    \centering
    \includegraphics[width=0.8\textwidth]{nl_fluxes.png}
    \caption{The evolution of the average horizontal flow velocity of the fluid
    $U_0$ (in units of $c_{ph,x}$ the horizontal phase velocity of the IGW) and
    horizontal momentum flux contained in the IGW $S_{px}(z)$ (in units of the
    total excited flux $S_0$) for one of my simulations. Different lines
    correspond to different times in the simulation. Times are measured in units
    of $N^{-1}$, the inverse of the Brunt-V\"ais\"al\"a frequency (or the
    buoyancy period), and heights in units of $H$ (the scale height of the
    density stratification). IGWs are excited at $z = 2H$ and propagate to
    higher $z$ before undergoing wave breaking. The sharp decrease in $S_{px}$
    is the location of the critical layer where the IGW breaks and drives $U_0$
    towards $c_{ph, x}$; it moves to lower $z$ over time. The apparent decrease
    in $S_{px}$ at later times indicates wave reflection off the critical
    layer.}\label{fig:nl_fluxes}
\end{figure}

\subsection{Proposed Work}

It is clear that full numerical modeling of IGW breaking is necessary to fully
characterize tidal dissipation. The first aim of my proposal is to
\textbf{extend my preliminary results concerning IGW breaking in stratified
atmospheres to characterize the dynamics of nonlinear IGW breaking in realistic
WD models.} Via numerical simulation, I will develop models for tidal
dissipation that can be used to study long-term WD evolution under tidal heating
without expensive hydrodynamical simulations. I will continue my work in the
following stages:
\begin{itemize}
    \item I will perform simulations examining the validity of my results in
        spherical geometries to capture tidal effects in WD binaries. I will
        continue to use Dedalus, which supports spherical coordinates. Although
        3D simulations are necessarily more complex than my 3D simulations of
        stratified atmospheres, the underlying dynamics of IGW breaking are the
        same.

    \item I will extend my simulations to realistic WD models and equations of
        state such as those in~\cite{brassard1992} as well as those generated by
        MESA\cite{MESA} (see \autoref{s:3} below), continuing to track the
        location and spatial extent of the dissipation layer as well as any new
        phenomena. As WDs vary widely in composition and effective temperature,
        studying representative WD models is vital to obtaining a robust
        characterization of tidal dissipation.
\end{itemize}

\section{Tidal Heating and Binary White Dwarf Evolution}\label{s:3}

\subsection{Background}

As discussed earlier, compact WD binaries may exhibit a range of transient
phenomena: tidal novae\cite{tidal_novae}, underluminous
supernovae\cite{underlum}, and Ca-rich fast transients\cite{carich} are all
hypothesized to arise in WD binary systems. Given that tidal heating can become
a significant contributor to a WD's total energy budget, a realistic model of
tidal dissipation is important to understanding the thermal evolution of WDs
during their binary inspiral.

In~\cite{tidal_novae} (hereafter FL), the authors used MESA\cite{MESA} to study
the production of tidal novae in binary WDs. A simple two-zone parameterization
where the tidal heat is deposited throughout the outer zone was used to model
tidal dissipation. It was found that cool WDs in sufficiently compact binaries
(orbital period $\lesssim 15$ minutes) may incur a thermonuclear detonation of
the hydrogen envelope. WD binaries with such short orbital periods have been
observed, e.g.\ SDSS J065133+284423 has a period of 12.75 minutes\cite{12min}.
FL fitted the observed properties of SDSS J065133+284423 to their tidal heating
models and finds evidence for tidal heating of the secondary WD in SDSS
J065133+284423.

\subsection{Proposed Work}

With the tidal dissipation models I will develop, I will be able to perform a
binary WD evolution study similar to that in FL but with a realistic tidal
dissipation profile (instead of their parameterized model). The second aim of my
proposal is to \textbf{use my tidal dissipation profiles to simulate a binary WD
undergoing tidal heating and make comparisons to observational data.} The
location of the tidal heating is a key ingredient in determining whether
the deposited tidal energy can be efficiently transfered away or whether the WD
experiences sudden detonations. As such, an improved understanding of the
properties of tidal dissipation is important to proper forecasting of binary
WDs' thermodynamic evolution. Moreover, while only a few sufficiently compact WD
binaries were available at the time of writing of FL, \emph{Gaia} data releases
2 and 3 will provide many more compact WD binaries to examine for observational
manifestations of tidal heating.

Using MESA, I will evolve various WD models undergoing tidal heating. From these
studies, I will extract the increased temperature of the WDs and make
comparisons to observational data, in particular to WDs in new \emph{Gaia} data
releases. I will also identify the occurrence rate and observational properties
of any predicted optical transients such as tidal novae and attempt to identify
them among existing detected events. Predictions of the occurence rates of such
phenomena are vital to guiding future observations. Finally, comparison of
observational data from known WD binaries to my theoretical predictions could
yield new constraints on the interior properties of WDs (e.g.\ crystallization).

\section{Tidal Dissipation and LISA}\label{s:4}

As discussed in \autoref{s:1}, WD binaries are an important source of GW
radiation for LISA\@. LISA will attain optimal sensitivity at frequencies
$10^{-4}$--$10^{-1}\;\mathrm{Hz}$\cite{LISA_band}. Exactly in this frequency
range, tidal effects act to synchronize the spin of the WDs to the binary orbit
and transfer energy from the orbit into the WDs. While the decay of the binary
orbit is still mostly driven by GW radiation, the tidal energy dissipation rate
grows to $\sim10^{-2}$ the GW luminosity\cite{fullerII,fullerIV}. An effect of
such a magnitude causes the phase of the emitted GWs to deviate significantly
from the point-mass binary prediction; the emitted wave may exhibit ``missing
cycles'' due to tidal effects\cite{fullerII}. GW astronomy uses matched
filtering, where a library of template waveforms is matched against instrument
data, to identify GW signals. As such, the accuracy and completeness of the
template library is of utmost importance.

The final aim of my proposal is \textbf{to use my tidal dissipation model to
compute WD binary GW waveforms including tidal dissipation for use in the LISA
detection pipeline}. This aim is much less computationally expensive than it
appears: LISA-band WD binaries can be well described using the Newtonian
dynamics of two co-orbiting point masses\cite{DWD_pointmass}. The resultant GW
emission can then be accurately computed using the weak gravity quadrupole
approximation (see e.g.~\cite{peters,lsst_wd}). Under these two approximations,
the GW waveform can be computed analytically without resorting to numerical
relativity simulations at all. Thus, I will compute GW waveforms accounting for
the additional phase evolution due to tidal dissipation. I will publish my
corrected waveforms for use by LISA and the GW community.

\section{Project Timeline}

During the first year of work and first half of the second year, I will complete
calculations of tidal dissipation models. I anticipate that the extension of my
2D plane-parallel work to 3D spherical geometries will be complete within the
first half year, while the extension to realistic WD models will take up to a
year. I expect that these two results together will produce two peer-reviewed
publications in addition to the one currently in preparation.

During the following year, I will use my tidal dissipation model to perform MESA
simulations of tidally heated WDs and compare to observational data. I expect my
MESA simulations and extracting appropriate observables to take about half a
year. I then intend to spend another half year analyzing observational data for
potential signatures of tidal heating or tidal novae. I expect both of these
phases will produce peer-reviewed publications.

Finally, in the last six months I will compute WD binary GW templates for use by
the LISA community. This work will also produce one peer-reviewed publication.

\section{Relevance to NASA Objectives}

This project is extremely relevant to the NASA Astrophysics research program.
My work directly relates to (i) the interactions of particles under the extreme
conditions found in astrophysical situations, (ii) how complex systems create
and shape the structure and composition of the universe on all scales, and (iii)
the development of new techniques that can be applied to future major missions.

My work will improve understanding of possible energetic phenomena that can
occur in compact object binaries. My results concerning angular momentum
transfer via IGW breaking will be applicable to astrophysical systems beyond WD
binaries, for instance angular momentum transfer in massive
stars\cite{l_trans_rev}. I will devise new ways of analyzing astrophysically
interesting systems in \emph{Gaia} and LSST\@. Finally, my GW templates will be
important for LISA GW detection efforts.

My work has direct relevance to NASA missions in the detectability of
astrophysical transients related to WD binaries. For instance, tidal novae are
theorized to have a similar observational signature to dwarf novae which have
been observed with Chandra and the Hubble Space Telescope, among others.

\clearpage

\printbibliography

\end{document}
