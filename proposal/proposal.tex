    \documentclass[12pt,
        usenames, % allows access to some tikz colors
        dvipsnames % more colors: https://en.wikibooks.org/wiki/LaTeX/Colors
    ]{article}
    \usepackage{
        amsmath,
        amssymb,
        fancyhdr, % page styling
        lastpage, % footer fanciness
        hyperref, % various links
        setspace, % line spacing
        amsthm, % newtheorem and proof environment
        mathtools, % \Aboxed for boxing inside aligns, among others
        float, % Allow [H] figure env alignment
        enumerate, % Allow custom enumerate numbering
        graphicx, % allow includegraphics with more filetypes
        wasysym, % \smiley!
        upgreek, % \upmu for \mum macro
        listings, % writing TrueType fonts and including code prettily
        tikz, % drawing things
        times,
        booktabs % \bottomrule instead of hline apparently
    }
    \usepackage[margin=0.5in, top=1in, bottom=1in]{geometry} % page geometry
    \usepackage[
        labelfont=bf, % caption names are labeled in bold
        font=scriptsize % smaller font for captions
    ]{caption}
    \usepackage[font=scriptsize]{subcaption} % subfigures

    \newcommand*{\scinot}[2]{#1\times10^{#2}}
    \newcommand*{\bra}[1]{\left<#1\right|}
    \newcommand*{\ket}[1]{\left|#1\right>}
    \newcommand*{\dotp}[2]{\left<#1\,\middle|\,#2\right>}
    \newcommand*{\rd}[2]{\frac{\mathrm{d}#1}{\mathrm{d}#2}}
    \newcommand*{\pd}[2]{\frac{\partial#1}{\partial#2}}
    \newcommand*{\rtd}[2]{\frac{\mathrm{d}^2#1}{\mathrm{d}#2^2}}
    \newcommand*{\ptd}[2]{\frac{\partial^2 #1}{\partial#2^2}}
    \newcommand*{\md}[2]{\frac{\mathrm{D}#1}{\mathrm{D}#2}}
    \newcommand*{\norm}[1]{\left|\left|#1\right|\right|}
    \newcommand*{\abs}[1]{\left|#1\right|}
    \newcommand*{\pvec}[1]{\vec{#1}^{\,\prime}}
    \newcommand*{\svec}[1]{\vec{#1}\;\!}
    \newcommand*{\bm}[1]{\boldsymbol{\mathbf{#1}}}
    \newcommand*{\expvalue}[1]{\left<#1\right>}
    \newcommand*{\ang}[0]{\text{\AA}}
    \newcommand*{\mum}[0]{\upmu \mathrm{m}}

    \newtheorem{theorem}{Theorem}[section]

    \let\Re\undefined
    \let\Im\undefined
    \DeclareMathOperator{\Res}{Res}
    \DeclareMathOperator{\Re}{Re}
    \DeclareMathOperator{\Im}{Im}
    \DeclareMathOperator{\Log}{Log}
    \DeclareMathOperator{\Arg}{Arg}
    \DeclareMathOperator{\Tr}{Tr}
    \DeclareMathOperator{\E}{E}
    \DeclareMathOperator{\Var}{Var}
    \DeclareMathOperator*{\argmin}{argmin}
    \DeclareMathOperator*{\argmax}{argmax}
    \DeclareMathOperator{\sgn}{sgn}
    \DeclareMathOperator{\diag}{diag}

    \everymath{\displaystyle} % biggify limits of inline sums and integrals
    \tikzstyle{circ} % usage: \node[circ, placement] (label) {text};
        = [draw, circle, fill=white, node distance=3cm, minimum height=2em]
    \definecolor{commentgreen}{rgb}{0,0.6,0}
    \lstset{
        basicstyle=\ttfamily\footnotesize,
        frame=single,
        numbers=left,
        showstringspaces=false,
        keywordstyle=\color{blue},
        stringstyle=\color{purple},
        commentstyle=\color{commentgreen},
        morecomment=[l][\color{magenta}]{\#}
    }

\begin{document}

\def\Snospace~{\S{}} % hack to remove the space left after autorefs
\renewcommand*{\sectionautorefname}{\Snospace}
\renewcommand*{\appendixautorefname}{\Snospace}
\renewcommand*{\figureautorefname}{Fig.}
\renewcommand*{\equationautorefname}{Eq.}
\renewcommand*{\tableautorefname}{Tab.}

\onehalfspacing

% helpful links:
% http://www.alexhunterlang.com/nsf-fellowship#TOC-Examples-of-Successful-Essays

\section{Proposed Research}

In the last decade, astrophysical systems with one white dwarf and another
body (e.g.\ a black hole or another white dwarf), referred to collectively as WD
binaries, have become increasingly important for several topics in astrophysics.
Mergers of WD-WD binaries create a variety of astrophysical systems, the most
important of which are type Ia supernovae, the standard candle of observational
astronomy. WD-BH (black hole) systems are expected to produce detectable flares
that can be observed to yield constraints on currently unknown physics
(https://arxiv.org/pdf/1701.08162.pdf).

\emph{Tidal interactions}, the effect of an uneven gravitational field produced
by one body on the other, are the key component in such systems. In WD binaries,
tidal interactions are expected to excite internal gravity waves in the WD\@.
These then propagate towards the surface and dissipate by wave breaking, as ocean
waves do on a beach, among other mechanisms.
Prior work shows that this wave breaking can play a dominant role in determining
the internal structure of the WD\@. Nonetheless, previous treatments of WD
binaries adopt incomplete, semi-analytical treatments to estimate the location
and strength of wave-breaking induced dissipation. For such fluid systems,
numerical simulation is the only way to obtain quantitatively accurate results.
% (Fuller 2012d)

\textbf{We propose to study nonlinear wave breaking of internal gravity waves in
WDs using numerical hydrodynamic codes.} There are already a wide assortment of
suitable hydrodynamical codes
% (Dedalus\footnote{
% http://iopscience.iop.org/article/10.1088/0004-%
% 637X/796/1/17}, Athena\footnote{e.g.\
% http://iopscience.iop.org/article/10.1088/0004-%
% 637X/741/1/57})
capable of solving the fully nonlinear hydrodynamical equations, we expect to be
able to apply such codes with slight modifications to solve the proposed
problem. Should existing codes prove inapplicable, the applicant is well
equipped to develop suitable software from scratch (\autoref{s:plan}). We aim
to extend our findings to observational signatures for a variety of compositions
and uncertain material properties.

The results of our study as GW astronomy continues to improve; inaccuracies are
especially important to mitigate with the high sensitivity requirements
involved. Conversely, a gravitational wave detection could yield significant
information on WD dynamics that, thanks to their ubiquity, could have
ramifications throughout astronomy. Equally importantly, our approach would be
highly transferable to other problems such as tidal heating during exoplanet
formation, another area of crucial and open research.

A tentative timeline for research follows:
\begin{itemize}
    \item 2--4 months: Set up hydrodynamic code, calibrate against simplified,
        well-understood fiducial models.
    \item 4--10 months: Apply hydrodynamic code to systems found in literature.
    \item 10--14 months: Generate observational counterparts to predicted
        phenomena.
    \item 14--24 months: Explore related interesting astrophysical systems, seek
        analytical descriptions of any discovered phenomena.
\end{itemize}

\section{Plan of Study}\label{s:plan}

Cornell University is the ideal institution to pursue the proposed research.
Professor Dong Lai, the applicant's adviser, has authored numerous papers on WD
tidal dissipation and is a foremost expert in the subject. Other faculty and
researchers at Cornell are equally well-suited for the project. Notably,
Professor Saul Teukolsky is a leading researcher in numerical techniques in
astrophysics for the past five decades.

The applicant earned two B.S.\ from the California Institute of Technology in
Physics and Computer science and is well equipped for the proposed problem with
teaching proficiency in differential equations and completed coursework in
computational physics, astrophysical fluid dynamics, GPU programming and others.
Future relevant coursework includes astrophysical processes, celestial
mechanics and physics of compact objects, all of which are scheduled to be
completed within six months of recipt of the fellowship funding. These courses
will enable both a complete treatment of relevant physics and accurate, fast
programming of numerical codes.

The proposed study is integral to the long-term career goals of the applicant,
a professorship in astrophysics at a research institution. Proficiency in
numerical techniques and software development is extremely important in modern
astrophysical studies. White dwarfs both are important to observational
astronomy and involve many astrophysical processes, so such a project would be
relevant to many members of the academic community and open avenues for
collaboration.

% \begin{itemize}
%     \item Lots of WKB treatments
%     % TODO radiative diffusion vs turbulent dissipation?
%     %   - if in stars can power energic outbursts, same for WD?
%     % Planet formation, WD inspirals + waveforms (important b/c Fuller et al P4)
%     %   - heating should have visible signatures in flares
%     \item

%     \item \url{http://journals.ametsoc.org/doi/pdf/10.1175/1520-%
%         0469\%281983\%29040\%3C2497\%3ATIOGWB\%3E2.0.CO\%3B2} Wave breaking is
%         an essential part of circulation % holten et al 1983

%     \item \url{https://arxiv.org/pdf/1211.0624.pdf}, % fuller 2012, vick 2016
%         \url{https://arxiv.org/abs/1612.07316} Show necessary to get heating as
%         function of depth

%     \item \url{http://journals.ametsoc.org/doi/abs/10.1175/JAS-D-11%
%         -097.1} has already studied to some extent in atmospheres, weakly
%         nonlinear regime
%         % sutherland 2011

%     \item Tidal heating in planet formation too!
% \end{itemize}

\end{document}

