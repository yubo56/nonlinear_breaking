    \documentclass[11pt,
        usenames, % allows access to some tikz colors
        dvipsnames % more colors: https://en.wikibooks.org/wiki/LaTeX/Colors
    ]{article}
    \usepackage[numbers]{natbib}
    \usepackage{
        amsmath,
        amssymb,
        fouriernc,
        fancyhdr, % page styling
        lastpage, % footer fanciness
        hyperref, % various links
        setspace, % line spacing
        amsthm, % newtheorem and proof environment
        mathtools, % \Aboxed for boxing inside aligns, among others
        float, % Allow [H] figure env alignment
        enumerate, % Allow custom enumerate numbering
        graphicx, % allow includegraphics with more filetypes
        wasysym, % \smiley!
        upgreek, % \upmu for \mum macro
        listings, % writing TrueType fonts and including code prettily
        tikz, % drawing things
        times,
        booktabs % \bottomrule instead of hline apparently
    }
    \usepackage[margin=0.8in, top=1in, bottom=1in]{geometry} % page geometry
    \usepackage[
        labelfont=bf, % caption names are labeled in bold
        font=scriptsize % smaller font for captions
    ]{caption}
    \usepackage[font=scriptsize]{subcaption} % subfigures

    \newcommand*{\scinot}[2]{#1\times10^{#2}}
    \newcommand*{\bra}[1]{\left<#1\right|}
    \newcommand*{\ket}[1]{\left|#1\right>}
    \newcommand*{\dotp}[2]{\left<#1\,\middle|\,#2\right>}
    \newcommand*{\rd}[2]{\frac{\mathrm{d}#1}{\mathrm{d}#2}}
    \newcommand*{\pd}[2]{\frac{\partial#1}{\partial#2}}
    \newcommand*{\rtd}[2]{\frac{\mathrm{d}^2#1}{\mathrm{d}#2^2}}
    \newcommand*{\ptd}[2]{\frac{\partial^2 #1}{\partial#2^2}}
    \newcommand*{\md}[2]{\frac{\mathrm{D}#1}{\mathrm{D}#2}}
    \newcommand*{\norm}[1]{\left|\left|#1\right|\right|}
    \newcommand*{\abs}[1]{\left|#1\right|}
    \newcommand*{\pvec}[1]{\vec{#1}^{\,\prime}}
    \newcommand*{\svec}[1]{\vec{#1}\;\!}
    \newcommand*{\bm}[1]{\boldsymbol{\mathbf{#1}}}
    \newcommand*{\expvalue}[1]{\left<#1\right>}
    \newcommand*{\ang}[0]{\text{\AA}}
    \newcommand*{\mum}[0]{\upmu \mathrm{m}}

    \newtheorem{theorem}{Theorem}[section]

    \let\Re\undefined
    \let\Im\undefined
    \DeclareMathOperator{\Res}{Res}
    \DeclareMathOperator{\Re}{Re}
    \DeclareMathOperator{\Im}{Im}
    \DeclareMathOperator{\Log}{Log}
    \DeclareMathOperator{\Arg}{Arg}
    \DeclareMathOperator{\Tr}{Tr}
    \DeclareMathOperator{\E}{E}
    \DeclareMathOperator{\Var}{Var}
    \DeclareMathOperator*{\argmin}{argmin}
    \DeclareMathOperator*{\argmax}{argmax}
    \DeclareMathOperator{\sgn}{sgn}
    \DeclareMathOperator{\diag}{diag}

    \tikzstyle{circ} % usage: \node[circ, placement] (label) {text};
        = [draw, circle, fill=white, node distance=3cm, minimum height=2em]
    \definecolor{commentgreen}{rgb}{0,0.6,0}
    \lstset{
        basicstyle=\ttfamily\footnotesize,
        frame=single,
        numbers=left,
        showstringspaces=false,
        keywordstyle=\color{blue},
        stringstyle=\color{purple},
        commentstyle=\color{commentgreen},
        morecomment=[l][\color{magenta}]{\#}
    }

\begin{document}

\def\Snospace~{\S{}} % hack to remove the space left after autorefs
\renewcommand*{\sectionautorefname}{\Snospace}
\renewcommand*{\appendixautorefname}{\Snospace}
\renewcommand*{\figureautorefname}{Fig.}
\renewcommand*{\equationautorefname}{Eq.}
\renewcommand*{\tableautorefname}{Tab.}

\onehalfspacing

% helpful links:
% http://www.alexhunterlang.com/nsf-fellowship#TOC-Examples-of-Successful-Essays

\title{Nonlinear Tidal Dissipation in White Dwarfs}
\author{Yubo Su}
\date{}

\maketitle

\section{Introduction: White Dwarf Binaries}

White dwarfs (WDs) are luminous, long-lasting products of late-stage stellar
evolution that are held up by electron degeneracy pressure, a quantum mechanical
effect arising from subjecting electrons to immense pressures. They are some of
the densest objects in the known universe, behind only neutron stars (NS) and
black holes (BH), fitting a solar mass into an Earth-sized sphere. Furthermore,
they exhibit various chemical compositions, each of which is subject to
different physics, and are found in a variety of interesting systems. As such,
WDs are a unique and diverse window into matter under extreme conditions.

WDs are commonly found in binary orbits with companion objects, in which the two
orbit their center of mass under their mutual gravitational attraction. The
companion object ranges from another WD to a supermassive BH (SMBH),
hypothesized to be at the center of galaxies and have mass $10^5$--$10^6$ solar
masses. All of these binary systems are very important to astrophysics. WD-WD
binaries are most important for being thought to generate \emph{Type Ia
supernovae} in which one of the two WDs reaches a critical mass through
accreting surrounding matter and collapses under its own gravity, producing a
shock wave that tears the WD apart and releases immense amounts of energy. Type
Ia supernovae are unique in astrophysics for their near-constancy throughout
space and time, consistently occuring at the WD critical mass, and have been
used to discover numerous cosmological phenomena such as dark energy.

WD-BH systems are also interesting subjects of study. Studies indicate that
as WDs orbit the extreme gravity surrounding BHs, they will produce observable
flares induced by gravitational tidal forces\cite{flares}. A WD orbiting a SMBH
would produce gravitational waves, waves carried by distortions in space-time
and generated according to Einstein's general relativity by accelerating massive
objects, that are expected to be detected by the space-based \emph{Laser
Interferometer Space Antenna (LISA)} when deployed\cite{lisa}. Gravitational
wave astronomy is an increasingly exciting field as the \emph{Laser
Interferometer Gravitational-Wave Observatory (LIGO)} continues to make
detections since its first in late 2015. As gravitational wave astronomy relies
on accurate predictions of the expected signals, it is important to build as
accurate models as possible before observation runs begin.

\subsection{Tidal Dissipation}

The excitation of \emph{internal gravity waves} in the WD by the tidal forces of
the companion is an effect seen in all of the aforementioned systems. Internal
gravity waves, not to be confused with the gravitational waves discussed above,
are internal displacements in the WD that oscillate and propagate due to a
restoring buoyancy force. As these waves propagate outward from where they are
excited, they are expected to grow in amplitude until they break, as do ocean
waves on a shore, and deposit both energy and angular momentum inside the WD\@.

Previous work predicts that this dissipation mechanism can generate
significantly more energy than thermal radiation from the WD surface alone and
are thus a significant contribution to the WD energy budget\cite{fullerII}. The
exact radial dissipation profile is of enormous interest since it is both
sensitive to WD properties and can produce drastically different observable
outcomes. One proposed outcome is a \emph{tidal nova}, in which heating in the
WD's degenerate hydrogen layer is significant enough to trigger runaway nuclear
fusion and a significant, observable explosion\cite{tidal_novae}. Understanding
whether such phenomena occur requires understanding how energy is distributed
internally inside a WD\@.

Internal gravity wave breaking is a nonlinear hydrodynamic phenomenon. Such
phenomena are known to require numerical simulation to study. It is therefore
paramount to begin numerical study to build dissipation models inside WDs to
characterise what phenomena can be observed in which WD models and
characteristics.

\section{Proposed Research}

\textbf{We propose to study tidal dissipation via nonlinear gravity wave
breaking in white dwarfs via numerical simulation.}

Our research will consist of computing for various WD models and compositions
the energy and angular momentum dissipation profiles inside WDs. Once such
profiles are obtained, we intend to add these profiles to existing stellar
evolution codes to study the dynamical effects of tidal dissipation. We will
attempt to find a compact or even analytical representation of our numerical
work to greatly simplify and accelerate such integration. Finally, we will make
observational predictions with our results and compare our predictions with
existing observed WDs. Any software and results would be made in accordance with
best practice and publicly available for the entire community.

\subsection{Qualifications}

The proposing researcher (YS) and his adviser (DL) are uniquely qualified to
pursue such studies. YS has ample research experience in numerical simulation
and is continuing to pursue academic study in hydrodynamics, turbulence and
computation. He also completed a double Bachelors degree in physics and computer
science from the California Institute of Technology. Finally, he has worked in
the software industry for a year and continues to follow discussions of best
practice far ahead of those in academia and apply them to his own work.

DL is a co-author on much of the latest literature on WD binaries and is an
expert on many related subjects such as fluid dynamics in extreme matter and
studies of other compact object systems. Cornell University also houses Saul
Teukolsky's research group, one of the pre-eminent numerical relativity groups
in the world, which will be frequently consulted for numerical best practices.

In terms of software, YS proposes to only implement what is not currently
available in the community. Current literature indicates that the spectral
hydrodynamic code Dedalus is able to handle instability-driven turbulence with
high accuracy and speed, and our preliminary work seems to indicate that Dedalus
will be sufficiently accurate to simulate wave breaking
dissipation\cite{dedalus}. Modules for Experiments in Stellar Astrophysics
(MESA) is a proven stellar evolution code that will likely need only be extended
to incorporate the discovered dissipation models\cite{MESA}. Nevertheless, both
codes will require adaptation to be applied to the present problem (YS has
already contributed an improvement to Dedalus), a task for which YS is
well-equiped.

Should no existing software prove adequate, we are more than able and willing to
develop a new code and contribute it to the community. Notably, no
GPU-accelerated spectral hydrodynamic code has yet been released, which would be
of great value to researchers, and YS is well-versed in GPU programming.
Nevertheless, in the interest of the proposed science objectives codes will be
developed on an as-needed basis.

\subsection{Applications, Interest to the DoD}

A robust pipeline coupling nonlinear wave breaking to system evolution
would be of interest spanning all fields of physics. Barring BHs, all
astrophysical objects exhibit internal waves that can break as they propagate
towards the surface. Notable present interests in the astrophysical community
include tidal disruption, planet formation and stars. Even in planetary and
atmospheric sciences, wave breaking in atmospheres would share many common
techniques with the proposed research.

The proposed research is extremely relevant to Department of Defense (DoD)
fields of interest. WDs and its energetic phenomena we propose to study e.g.\
novae of all sorts are natural laboratories of extreme physics that have
direct consequences for understanding of plasmas and other technologies integral
to propulsion in aeronautics. A theory of nonlinear wave breaking would have
far-reaching consequences in turbulence study and atmospheric sciences. Finally,
the observation of our predictions could shed light on atmospheric optics,
whether as a use for calibration or understanding contamination effects.

\bibliographystyle{plainnat}
\renewcommand{\bibname}{References}
{\scriptsize \bibliography{proposal}}

\end{document}

