    \documentclass[11pt,
        usenames, % allows access to some tikz colors
        dvipsnames % more colors: https://en.wikibooks.org/wiki/LaTeX/Colors
    ]{article}
    \usepackage{
        amsmath,
        amssymb,
        fouriernc, % fourier font w/ new century book
        fancyhdr, % page styling
        lastpage, % footer fanciness
        hyperref, % various links
        setspace, % line spacing
        amsthm, % newtheorem and proof environment
        mathtools, % \Aboxed for boxing inside aligns, among others
        float, % Allow [H] figure env alignment
        enumerate, % Allow custom enumerate numbering
        graphicx, % allow includegraphics with more filetypes
        wasysym, % \smiley!
        upgreek, % \upmu for \mum macro
        listings, % writing TrueType fonts and including code prettily
        tikz, % drawing things
        booktabs, % \bottomrule instead of hline apparently
        cancel % can cancel things out!
    }
    \usepackage[margin=1in]{geometry} % page geometry
    \usepackage[
        labelfont=bf, % caption names are labeled in bold
        font=scriptsize % smaller font for captions
    ]{caption}
    \usepackage[font=scriptsize]{subcaption} % subfigures

    \newcommand*{\scinot}[2]{#1\times10^{#2}}
    \newcommand*{\dotp}[2]{\left<#1\,\middle|\,#2\right>}
    \newcommand*{\rd}[2]{\frac{\mathrm{d}#1}{\mathrm{d}#2}}
    \newcommand*{\pd}[2]{\frac{\partial#1}{\partial#2}}
    \newcommand*{\rtd}[2]{\frac{\mathrm{d}^2#1}{\mathrm{d}#2^2}}
    \newcommand*{\ptd}[2]{\frac{\partial^2 #1}{\partial#2^2}}
    \newcommand*{\md}[2]{\frac{\mathrm{D}#1}{\mathrm{D}#2}}
    \newcommand*{\pvec}[1]{\vec{#1}^{\,\prime}}
    \newcommand*{\svec}[1]{\vec{#1}\;\!}
    \newcommand*{\bm}[1]{\boldsymbol{\mathbf{#1}}}
    \newcommand*{\ang}[0]{\;\text{\AA}}
    \newcommand*{\mum}[0]{\;\upmu \mathrm{m}}
    \newcommand*{\at}[1]{\left.#1\right|}

    \newtheorem{theorem}{Theorem}[section]

    \let\Re\undefined
    \let\Im\undefined
    \DeclareMathOperator{\Res}{Res}
    \DeclareMathOperator{\Re}{Re}
    \DeclareMathOperator{\Im}{Im}
    \DeclareMathOperator{\Log}{Log}
    \DeclareMathOperator{\Arg}{Arg}
    \DeclareMathOperator{\Tr}{Tr}
    \DeclareMathOperator{\E}{E}
    \DeclareMathOperator{\Var}{Var}
    \DeclareMathOperator*{\argmin}{argmin}
    \DeclareMathOperator*{\argmax}{argmax}
    \DeclareMathOperator{\sgn}{sgn}
    \DeclareMathOperator{\diag}{diag\;}

    \DeclarePairedDelimiter\bra{\langle}{\rvert}
    \DeclarePairedDelimiter\ket{\lvert}{\rangle}
    \DeclarePairedDelimiter\abs{\lvert}{\rvert}
    \DeclarePairedDelimiter\ev{\langle}{\rangle}
    \DeclarePairedDelimiter\p{\lparen}{\rparen}
    \DeclarePairedDelimiter\s{\lbrack}{\rbrack}
    \DeclarePairedDelimiter\z{\lbrace}{\rbrace}

    % \everymath{\displaystyle} % biggify limits of inline sums and integrals
    \tikzstyle{circ} % usage: \node[circ, placement] (label) {text};
        = [draw, circle, fill=white, node distance=3cm, minimum height=2em]
    \definecolor{commentgreen}{rgb}{0,0.6,0}
    \lstset{
        basicstyle=\ttfamily\footnotesize,
        frame=single,
        numbers=left,
        showstringspaces=false,
        keywordstyle=\color{blue},
        stringstyle=\color{purple},
        commentstyle=\color{commentgreen},
        morecomment=[l][\color{magenta}]{\#}
    }
    \usepackage[
        backend=bibtex8,
        style=nature]{biblatex}

\bibliography{Su_GRFP}
% \AtEveryBibitem{\clearfield{title}}
\begin{document}

% NOTE: this is not the version I submitted, I wrote it on gdocs

\def\Snospace~{\S{}} % hack to remove the space left after autorefs
\renewcommand*{\sectionautorefname}{\Snospace}
\renewcommand*{\appendixautorefname}{\Snospace}
\renewcommand*{\figureautorefname}{Fig.}
\renewcommand*{\equationautorefname}{Eq.}
\renewcommand*{\tableautorefname}{Tab.}

\onehalfspacing

\pagestyle{fancy}
\rfoot{Yubo Su}
\rhead{}
\cfoot{\thepage/\pageref{LastPage}}

\title{Nonlinear Tidal Dissipation in Binary White Dwarfs}
\author{Yubo Su}
\date{\today}

\maketitle


\section{White Dwarf Binaries}

White dwarfs (WDs) are remnants of stellar evolution for stars with mass
$\lesssim 8M_{\odot}$. They are some of the densest objects in the universe,
fitting a solar mass into an Earth-sized sphere. They are supported against
gravity by degeneracy pressure, a quantum mechanical effect arising from
subjecting matter to immense pressure. As such, WDs are a unique window into
matter under extreme conditions.

WDs are commonly found in binary systems in which two objects orbit their center
of mass under mutual gravitational attraction. The companion object ranges from
another WD to a supermassive black hole, and all of these binary systems are
very important to astrophysics. In particular, merging WD-WD binaries are
thought to generate Type Ia supernovae (SNe Ia). Because SNe Ia are highly
luminous and can be seen at large distances, they have been used as ``standard
candles'' to probe the expansion rate of the universe (e.g.\ in 1998 SNe Ia
observations provided the first evidence for dark energy\cite{DE}). WD binaries
also produce gravitational waves (GW), periodic warping of space-time predicted
by Einstein's theory of general relativity. The first detection of GWs by the
Laser Interferometer Gravitational-Wave Observatory (LIGO) in 2015 heralded a
new era of GW astrophysics. Although LIGO cannot detect the low-frequency GWs
expected from WD binaries, the space-based Laser Interferometer Space Antenna
(LISA) will detect such GWs in the coming decades\cite{lisa}. As GW astronomy
relies on accurate predictions of the expected gravitational waveforms, it is
important to build highly precise models of the WD binary inspiral before LISA
observation runs begin (scheduled for $\sim 2030$).

\subsection{Tidal Dissipation and Internal Gravity Waves}

Tidal dissipation in binary WDs can affect the pre-merger state of the WD and
have a major impact on the gravitational waves produced by the binary. Such
dissipation arises from the excitation of internal gravity waves (IGWs) in the
deep envelope of the WD by the gravitational force of its companion and their
subsequent dissipation in the outer envelope\cite{fullerII}. IGWs, not to be
confused with the gravitational waves above, are internal displacements in the
WD fluid that oscillate and propagate due to a restoring buoyancy force. These
waves are analogous to ocean tides on Earth raised by the Moon and the Sun,
except that, since WDs do not have sharp surfaces, these waves are internal to
the WD\@. As these waves propagate outwards from where they are excited, they
grow in amplitude until they break, as do ocean waves on a shore, and deposit
both energy and angular momentum from the binary orbit in the outer envelope of
the WD\@.

Previous works have shown that this dissipation mechanism can generate
significantly more energy than thermal radiation from the WD surface alone and
is thus a major contributor to the WD energy budget\cite{fullerII,fullerIV}.
However, these works treated nonlinear wave dissipation by a simple
parameterization. Related works in other fields find that properly including
nonlinear effects produces drastically different IGW dissipation
behavior\cite{winters1994}. Such fully nonlinear studies have not been performed
for astrophysical binary systems but are important for the thermal and orbital
evolution of WD binaries undergoing tidal heating. Characterizing this nonlinear
tidal dissipation will require extensive numerical simulation to capture the
turbulent cascade to small scales that drives IGW dissipation. This is the goal
of my proposed research.

\section{Proposed Research}

I propose to study the dynamical effects of tidal dissipation via nonlinear IGW
breaking in binary WDs using both analytical techniques and numerical
simulations. As WDs vary widely in composition and effective temperature, I will
study select WD models to capture a wide range of possible phenomena. My
research will proceed in the following stages:

I will perform simulations of nonlinear IGW breaking for various WD models. In
the last year, I have adapted the spectral hydrodynamic code
Dedalus\cite{dedalus} to simulate the simplest WD models. I have excited IGWs in
the deep envelope and observed nonlinear dissipation as the wave propagates into
the outer envelope. I am in the process of characterizing this dissipation and
extending my simulation to other, more realistic WD models. From these
simulations, I will compute energy and angular momentum dissipation as a
function of time and depth in the envelope for each WD model.

I will apply my dissipation profiles to study tidally heated WDs. Modules for
Experiments in Stellar Astrophysics (MESA) is a proven stellar evolution code
that is readily extensible to include tidal heating\cite{MESA}. Packaging my dissipation
profiles for each WD model as a MESA module would enable simulations of the
binary WDs’ internal structures evolving under tidal heating. From these
simulations, I will extract the WD luminosity over time.

I will compare the simulated binary WD luminosity to observational data to
constrain WD properties. For instance, the WD binary SDSS JJ065133+284423
(period 13 minutes) may undergo strong tidal heating and exhibit energetic
behavior such as tidal novae\cite{12min, tidal_novae}. The Large Synoptic Survey
Telescope (LSST, expected 2021) is expected to detect a few thousand more WD
binaries\cite{lsst_wd}. I will forecast properties of these samples due to tidal
heating.

Tidal dissipation can also affect the orbital evolution of the WD binary and
hence its GW emission during inspiral. I will calculate the signature of tidal
dissipation on these GWs for each WD model. I will publish my corrected GW
waveforms for use by LISA and the GW community.

Intellectual Merit: As discussed, WD binaries are important for being potential
progenitors of Type Ia supernovae, for being future LISA GW sources and for
producing various transients that would be detectable. My work will characterize
a key ingredient of WD binary evolution that may explain observed properties of
such binaries. Additionally, the nonlinear IGW dissipation model I will build
has applications beyond WDs binaries. Within astrophysics, IGWs are a key
angular momentum transport mechanism in stellar interiors: they are believed to
fill a crucial gap in stellar evolution theory\cite{l_trans_rev}. Beyond
astrophysics, IGWs are also studied in terrestrial sciences such as oceanography
and atmospheric sciences. A robust understanding of nonlinear IGW dissipation
would thus have consequences for problems ranging from planetary and stellar
evolution to weather patterns and biosphere dynamics on Earth.

Broader Impacts: With the support of the NSF GRFP, I will grow as an effective
researcher. I will present my results at conferences and to collaborators in the
US and abroad. I will generate videos for use in future Cornell outreach
efforts. As hydrodynamical simulation produces visually spectacular videos, my
research will be an effective tool in inspiring the public.

\printbibliography

\end{document}
