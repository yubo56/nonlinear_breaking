    \documentclass[11pt,
        usenames, % allows access to some tikz colors
        dvipsnames % more colors: https://en.wikibooks.org/wiki/LaTeX/Colors
    ]{article}
    \usepackage{
        amsmath,
        amssymb,
        fouriernc, % fourier font w/ new century book
        hyperref, % various links
        setspace, % line spacing
        amsthm, % newtheorem and proof environment
        mathtools, % \Aboxed for boxing inside aligns, among others
        float, % Allow [H] figure env alignment
        enumerate, % Allow custom enumerate numbering
        graphicx, % allow includegraphics with more filetypes
        wasysym, % \smiley!
        upgreek, % \upmu for \mum macro
        listings, % writing TrueType fonts and including code prettily
        tikz, % drawing things
        booktabs, % \bottomrule instead of hline apparently
        cancel % can cancel things out!
    }
    \usepackage[margin=1in]{geometry} % page geometry
    \usepackage[
        labelfont=bf, % caption names are labeled in bold
        font=scriptsize % smaller font for captions
    ]{caption}
    \usepackage[font=scriptsize]{subcaption} % subfigures

    \newcommand*{\scinot}[2]{#1\times10^{#2}}
    \newcommand*{\dotp}[2]{\left<#1\,\middle|\,#2\right>}
    \newcommand*{\rd}[2]{\frac{\mathrm{d}#1}{\mathrm{d}#2}}
    \newcommand*{\pd}[2]{\frac{\partial#1}{\partial#2}}
    \newcommand*{\rtd}[2]{\frac{\mathrm{d}^2#1}{\mathrm{d}#2^2}}
    \newcommand*{\ptd}[2]{\frac{\partial^2 #1}{\partial#2^2}}
    \newcommand*{\md}[2]{\frac{\mathrm{D}#1}{\mathrm{D}#2}}
    \newcommand*{\pvec}[1]{\vec{#1}^{\,\prime}}
    \newcommand*{\svec}[1]{\vec{#1}\;\!}
    \newcommand*{\bm}[1]{\boldsymbol{\mathbf{#1}}}
    \newcommand*{\ang}[0]{\;\text{\AA}}
    \newcommand*{\mum}[0]{\;\upmu \mathrm{m}}
    \newcommand*{\at}[1]{\left.#1\right|}

    \newtheorem{theorem}{Theorem}[section]

    \let\Re\undefined
    \let\Im\undefined
    \DeclareMathOperator{\Res}{Res}
    \DeclareMathOperator{\Re}{Re}
    \DeclareMathOperator{\Im}{Im}
    \DeclareMathOperator{\Log}{Log}
    \DeclareMathOperator{\Arg}{Arg}
    \DeclareMathOperator{\Tr}{Tr}
    \DeclareMathOperator{\E}{E}
    \DeclareMathOperator{\Var}{Var}
    \DeclareMathOperator*{\argmin}{argmin}
    \DeclareMathOperator*{\argmax}{argmax}
    \DeclareMathOperator{\sgn}{sgn}
    \DeclareMathOperator{\diag}{diag\;}

    \DeclarePairedDelimiter\bra{\langle}{\rvert}
    \DeclarePairedDelimiter\ket{\lvert}{\rangle}
    \DeclarePairedDelimiter\abs{\lvert}{\rvert}
    \DeclarePairedDelimiter\ev{\langle}{\rangle}
    \DeclarePairedDelimiter\p{\lparen}{\rparen}
    \DeclarePairedDelimiter\s{\lbrack}{\rbrack}
    \DeclarePairedDelimiter\z{\lbrace}{\rbrace}

    % \everymath{\displaystyle} % biggify limits of inline sums and integrals
    \tikzstyle{circ} % usage: \node[circ, placement] (label) {text};
        = [draw, circle, fill=white, node distance=3cm, minimum height=2em]
    \definecolor{commentgreen}{rgb}{0,0.6,0}
    \lstset{
        basicstyle=\ttfamily\footnotesize,
        frame=single,
        numbers=left,
        showstringspaces=false,
        keywordstyle=\color{blue},
        stringstyle=\color{purple},
        commentstyle=\color{commentgreen},
        morecomment=[l][\color{magenta}]{\#}
    }

\begin{document}

\def\Snospace~{\S{}} % hack to remove the space left after autorefs
\renewcommand*{\sectionautorefname}{\Snospace}
\renewcommand*{\appendixautorefname}{\Snospace}
\renewcommand*{\figureautorefname}{Fig.}
\renewcommand*{\equationautorefname}{Eq.}
\renewcommand*{\tableautorefname}{Tab.}

\onehalfspacing

\section{Problem}

To the end of simulating breaking of internal gravity waves (IGW) in white
dwarfs (WDs), we begin with the toy problem of IGW breaking in a uniformly
stratified atmosphere in 2D. To begin with the simplest treatment, we consider
an incompressible fluid characterized by four dynamical variables $\rho, P, u_x,
u_z$.

\subsection{Physical Description}

We consider perturbations, not necessarily small, in a medium that at
equilibrium is at rest and has uniform density stratification
\begin{equation}
    \rho_0(x, z) \propto e^{-z/H}.
\end{equation}

For an incompressible fluid, the density of the fluid $\rho\p*{x, z, t}$
decomposes into this background stratification $\rho_0(x, z)$ and a small
perturbation $\rho_1\p*{x, z, t} \ll \rho_0$ ($\rho_1 \ll \rho_0$ is a necessary
condition for an incompressible treatment). Permitting a similar decomposition for
$P(x, z, t) = P_0(x, z) + P_1(x, z, t)$ and using hydrostatic equilibrium
$\rd{P_0(x, z)}{z} = -\rho g$ gives fluid equations
\begin{subequations}\label{eq:nonlin_feq}
    \begin{align}
        \vec{\nabla} \cdot \vec{u}_1 &= 0,\\
        \pd{\rho_1}{t} - u_{1z}\frac{\rho_0}{H} &=
            -\p*{\vec{u}_1 \cdot \vec{\nabla}} \rho_1,\\
        \pd{\vec{u}_1}{t} + \frac{\vec{\nabla}P_1}{\rho_0}
            + \frac{\rho_1}{\rho_0}\vec{g} &= -\p*{\vec{u}_1 \cdot \vec{\nabla}}
            \vec{u}_1.
    \end{align}
\end{subequations}
I notate $\vec{u} \equiv \vec{u_1}$ to identify that it is a perturbation
variable even though there is no distinction between the two without a
background flow.

\subsection{Linear Description}

If the perturbation is sufficiently small $u_1 \ll v_{ph}$ the phase velocity,
then we drop terms nonlinear in the perturbation quantities. The linearized
equations admit wave-like solutions described by
\begin{subequations}\label{eq:lin_sol}
    \begin{align}
        u_{1z} &= Ae^{z/2H}e^{i\p*{\vec{k} \cdot \vec{r} - \omega t}},\\
        \pd{u_{1x}}{x} &= -\pd{u_{1z}}{z},\\
        \pd{\rho_1}{t} &= u_{1z}\frac{\rho_0}{H},\\
        \pd{P}{x} &= -\rho_0 \pd{u_{1x}}{t},\\
        \omega^2 &= \frac{N^2k_x^2}{k_x^2 + k_z^2 + \frac{1}{4H^2}}.
    \end{align}
\end{subequations}
Note $N^2 = g/H$ the buoyancy frequency here.

\section{Linear Numerical Simulation using \texttt{Dedalus}}

In the WD problem, the IGW is generated deep within the WD and propagates
outwards. Since we are only concerned with its breaking behavior, we initially
seek to excite waves in a way agnostic of how the waves are generated. Put
another way, we don't want to generate the wave, we just want to see how the
wave evolves as it nears nonlinear amplitudes. We begin with the linear problem
to see whether our mechanism produces expected results.

The code described in this section is attached as \texttt{strat.py}, a minimum
working example that saves plots to \texttt{plots/} on the fly.

\subsection{Numerical Setup}

To this end, we simulate the linearized fluid equations
\begin{subequations}\label{eq:lin_feq}
    \begin{align}
        \vec{\nabla} \cdot \vec{u}_1 &= 0,\\
        \pd{\rho_1}{t} - u_{1z}\frac{\rho_0}{H} &= 0,\\
        \pd{\vec{u}_1}{t} + \frac{\vec{\nabla}P_1}{\rho_0}
            + \frac{\rho_1}{\rho_0}\vec{g} &= 0.
    \end{align}
\end{subequations}

The domain of our simulation is $x \in [0, H], z \in [0, 4H]$. We use a Fourier
basis in the $x$ and a Chebyshev in the $z$. We try with just $16$ modes in the
$x$ and $64$ modes in the $z$. Since we choose parameters of the problem such
that $\frac{2\pi}{k_x} = H, \frac{2\pi}{k_z} = -\frac{4H}{\pi}$\footnote{As
seen later, $k_z$ never explicitly enters into the problem description; instead,
we specify $\omega(k_x, k_z)$ using the desired $k_x, k_z$ values.}, this should
be plenty of resolution to resolve the IGW (I chose $k_z$ as an irrational value
to ensure any undesired reflections are far from any normal modes of the box).
The timescale is set such that $N = 1$.

\subsection{Reflection Suppression}

To simulate a purely outgoing wave, we adopt a damping zone near the top
boundary to suppress reflection. Specifically, we use
\begin{subequations}\label{eq:lin_feq_sponge}
    \begin{align}
        \vec{\nabla} \cdot \vec{u}_1 &= 0,\\
        \pd{\rho_1}{t} + f(z)\rho_1 - u_{1z}\frac{\rho_0}{H} &= 0,\\
        \pd{\vec{u}_1}{t} + f(z)\rho_1 + \frac{\vec{\nabla}P_1}{\rho_0}
            + \frac{\rho_1}{\rho_0}\vec{g} &= 0,
    \end{align}
\end{subequations}
where
\begin{equation}
    f(z) = f_0 \frac{\max\p*{z - z_0, 0}^2}{\p*{z_{\max} - z_0}^2}.
\end{equation}
We use $f_0 = 2, z_0 = 0.7 z_{\max}$, so the damping turns on quadratically
starting at $z_0 = 0.7z_{\max}$.

\subsection{Boundary Conditions}

To excite the wave, we adopt bottom boundary condition (BC)
\begin{equation}
    \at{\pd{u_{1z}}{z}}_{z = 0} = -Ak_z\cos\p*{k_xx - \omega t + \frac{1}{2H}}.
\end{equation}
The motivation for this comes from differentiating $\pd{u_{1z}}{z}$ in
\autoref{eq:lin_sol}. The reason our BC isn't just $u_{1z}(z = 0)$ is because the
initial condition $u_x(t = 0) = u_{1z}(t = 0) = 0$ is not divergence free.

Since we have the reflection suppression term above, it turns out we need fix
the $u_{1z}$ gauge, otherwise the $k_x = 0$ mode is numerically ill-conditioned
and causes \texttt{Dedalus} to blow up. Thus, we enforce $u_{1z}(k_x = 0) = 0$.

The remaining boundary condition (there are two $\partial_z$ terms in
\autoref{eq:lin_feq}) need only fix the pressure gauge, so we choose $P_1(k_x =
0) = 0$ and a throwaway $u_{1x}(k_x \neq 0) = 0$.

\subsection{Initial Conditions}

We initialize everything $\rho_1 = P_1 = u_{1x} = u_{1z} = 0$ everywhere.

\subsection{Results}

A snapshot of the simulation produced by \texttt{strat.py} is shown below in
\autoref{fig:lin_plot}. We see reasonable agreement between the analytical
solution and the simulation data as the wave propagates. More plots can be found
in the \texttt{plots/} folder.
\begin{figure}[t]
    \centering
    \begin{subfigure}{0.7\textwidth}
        \centering
        \includegraphics[width=\textwidth]{plots/t_20.png}
    \end{subfigure}

    \begin{subfigure}{0.7\textwidth}
        \centering
        \includegraphics[width=\textwidth]{plots/t_70.png}
    \end{subfigure}
    \caption{For line plots, red line indicates beginning of damping zone, blue
    the analytical solution and orange the simulation data. Variables are
    $u_{1z} = uz, u_{1x} = ux$, $F_z$ the energy flux, $E$ the energy, $P_1 = P,
    \rho_1 = rho$. Line plots of dynamical variables are slices through $x = 0$.
    }\label{fig:lin_plot}
\end{figure}

\section{Nonlinear Numerical Simulation}

\subsection{Low Amplitude}

To move to the nonlinear problem, we first introduce the nonlinear terms as in
\autoref{eq:nonlin_feq} but use small $A$ such that $u_{1z} \sim
Ae^{z_{\max}/2H} \ll v_{ph}$. All other problem setup is kept the same. The code
is in \texttt{strat\_nonlin\_lowA.py}. We expect agreement with the analytical
solution to the linear problem.

\subsection{Non-negligible amplitude: Instability}

We now increase the amplitude, and a numerical instability seems to develop. We
explore this with \texttt{strat\_nonlin.py}. An instability seems to grow near
$z = 0$.

\section{Attempted Solutions}

It seems that the observed instability has to do with the interface driving. A
few possible solutions have been considered:
\begin{itemize}
    \item I've considered other types of boundary conditions, such as
        $\ptd{u_{1z}}{z}$ or $\pd{P}{z}$, but all of these seem to produce
        similar behavior, instability near $z = 0$.

    \item Start the simulation with a small amplitude and use $z_{\max}$ very
        large, relying on the $u_{1z} \sim e^{-z/2H}$ to bring the wave to
        breaking amplitudes. This will require a much larger number of spatial
        modes to resolve the vertical wavelengths though and will be very slow.

    \item Bulk forcing has also been considered, where we use a driving term in
        the momentum equation $\pd{\vec{u}_1}{t}$ to generate IGW, then use
        reflection-suppressing damping zones on both sides of the $z$ domain.
        This seems to draw focus away from the focus on the breaking portion.

        I was able to get bulk forcing to work, however, and the results seem
        reasonable. It would be a less desirable entry point though, since we're
        trying to study just the breaking irrespective of the forcing mechanism.
\end{itemize}

\end{document}

